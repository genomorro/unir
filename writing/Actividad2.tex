\documentclass[12pt,a4paper,table]{article}

    \usepackage[breakable]{tcolorbox}
    \usepackage{parskip} % Stop auto-indenting (to mimic markdown behaviour)
    

    % Basic figure setup, for now with no caption control since it's done
    % automatically by Pandoc (which extracts ![](path) syntax from Markdown).
    \usepackage{graphicx}
    % Maintain compatibility with old templates. Remove in nbconvert 6.0
    \let\Oldincludegraphics\includegraphics
    % Ensure that by default, figures have no caption (until we provide a
    % proper Figure object with a Caption API and a way to capture that
    % in the conversion process - todo).
    \usepackage{caption}
    \DeclareCaptionFormat{nocaption}{}
    \captionsetup{format=nocaption,aboveskip=0pt,belowskip=0pt}

    \usepackage{float}
    \floatplacement{figure}{H} % forces figures to be placed at the correct location
    \usepackage{xcolor} % Allow colors to be defined
    \usepackage{enumerate} % Needed for markdown enumerations to work
    \usepackage{geometry} % Used to adjust the document margins
    \usepackage{amsmath} % Equations
    \usepackage{amssymb} % Equations
    \usepackage{textcomp} % defines textquotesingle
    % Hack from http://tex.stackexchange.com/a/47451/13684:
    \AtBeginDocument{%
        \def\PYZsq{\textquotesingle}% Upright quotes in Pygmentized code
    }
    \usepackage{upquote} % Upright quotes for verbatim code
    \usepackage{eurosym} % defines \euro

    \usepackage{iftex}
    \ifPDFTeX
        \usepackage[T1]{fontenc}
        \IfFileExists{alphabeta.sty}{
              \usepackage{alphabeta}
          }{
              \usepackage[mathletters]{ucs}
              \usepackage[utf8x]{inputenc}
          }
    \else
        \usepackage{fontspec}
        \usepackage{unicode-math}
    \fi

    \usepackage{fancyvrb} % verbatim replacement that allows latex
    \usepackage{grffile} % extends the file name processing of package graphics 
                         % to support a larger range
    \makeatletter % fix for old versions of grffile with XeLaTeX
    \@ifpackagelater{grffile}{2019/11/01}
    {
      % Do nothing on new versions
    }
    {
      \def\Gread@@xetex#1{%
        \IfFileExists{"\Gin@base".bb}%
        {\Gread@eps{\Gin@base.bb}}%
        {\Gread@@xetex@aux#1}%
      }
    }
    \makeatother
    \usepackage[Export]{adjustbox} % Used to constrain images to a maximum size
    \adjustboxset{max size={0.9\linewidth}{0.9\paperheight}}

    % The hyperref package gives us a pdf with properly built
    % internal navigation ('pdf bookmarks' for the table of contents,
    % internal cross-reference links, web links for URLs, etc.)
    \usepackage{hyperref}
    % The default LaTeX title has an obnoxious amount of whitespace. By default,
    % titling removes some of it. It also provides customization options.
    \usepackage{titling}
    \usepackage{longtable} % longtable support required by pandoc >1.10
    \usepackage{booktabs}  % table support for pandoc > 1.12.2
    \usepackage{array}     % table support for pandoc >= 2.11.3
    \usepackage{calc}      % table minipage width calculation for pandoc >= 2.11.1
    \usepackage[inline]{enumitem} % IRkernel/repr support (it uses the enumerate* environment)
    \usepackage[normalem]{ulem} % ulem is needed to support strikethroughs (\sout)
                                % normalem makes italics be italics, not underlines
    \usepackage{mathrsfs}
    

    
    % Colors for the hyperref package
    \definecolor{urlcolor}{rgb}{0,.145,.698}
    \definecolor{linkcolor}{rgb}{.71,0.21,0.01}
    \definecolor{citecolor}{rgb}{.12,.54,.11}

    % ANSI colors
    \definecolor{ansi-black}{HTML}{3E424D}
    \definecolor{ansi-black-intense}{HTML}{282C36}
    \definecolor{ansi-red}{HTML}{E75C58}
    \definecolor{ansi-red-intense}{HTML}{B22B31}
    \definecolor{ansi-green}{HTML}{00A250}
    \definecolor{ansi-green-intense}{HTML}{007427}
    \definecolor{ansi-yellow}{HTML}{DDB62B}
    \definecolor{ansi-yellow-intense}{HTML}{B27D12}
    \definecolor{ansi-blue}{HTML}{208FFB}
    \definecolor{ansi-blue-intense}{HTML}{0065CA}
    \definecolor{ansi-magenta}{HTML}{D160C4}
    \definecolor{ansi-magenta-intense}{HTML}{A03196}
    \definecolor{ansi-cyan}{HTML}{60C6C8}
    \definecolor{ansi-cyan-intense}{HTML}{258F8F}
    \definecolor{ansi-white}{HTML}{C5C1B4}
    \definecolor{ansi-white-intense}{HTML}{A1A6B2}
    \definecolor{ansi-default-inverse-fg}{HTML}{FFFFFF}
    \definecolor{ansi-default-inverse-bg}{HTML}{000000}

    % common color for the border for error outputs.
    \definecolor{outerrorbackground}{HTML}{FFDFDF}

    % commands and environments needed by pandoc snippets
    % extracted from the output of `pandoc -s`
    \providecommand{\tightlist}{%
      \setlength{\itemsep}{0pt}\setlength{\parskip}{0pt}}
    \DefineVerbatimEnvironment{Highlighting}{Verbatim}{commandchars=\\\{\}}
    % Add ',fontsize=\small' for more characters per line
    \newenvironment{Shaded}{}{}
    \newcommand{\KeywordTok}[1]{\textcolor[rgb]{0.00,0.44,0.13}{\textbf{{#1}}}}
    \newcommand{\DataTypeTok}[1]{\textcolor[rgb]{0.56,0.13,0.00}{{#1}}}
    \newcommand{\DecValTok}[1]{\textcolor[rgb]{0.25,0.63,0.44}{{#1}}}
    \newcommand{\BaseNTok}[1]{\textcolor[rgb]{0.25,0.63,0.44}{{#1}}}
    \newcommand{\FloatTok}[1]{\textcolor[rgb]{0.25,0.63,0.44}{{#1}}}
    \newcommand{\CharTok}[1]{\textcolor[rgb]{0.25,0.44,0.63}{{#1}}}
    \newcommand{\StringTok}[1]{\textcolor[rgb]{0.25,0.44,0.63}{{#1}}}
    \newcommand{\CommentTok}[1]{\textcolor[rgb]{0.38,0.63,0.69}{\textit{{#1}}}}
    \newcommand{\OtherTok}[1]{\textcolor[rgb]{0.00,0.44,0.13}{{#1}}}
    \newcommand{\AlertTok}[1]{\textcolor[rgb]{1.00,0.00,0.00}{\textbf{{#1}}}}
    \newcommand{\FunctionTok}[1]{\textcolor[rgb]{0.02,0.16,0.49}{{#1}}}
    \newcommand{\RegionMarkerTok}[1]{{#1}}
    \newcommand{\ErrorTok}[1]{\textcolor[rgb]{1.00,0.00,0.00}{\textbf{{#1}}}}
    \newcommand{\NormalTok}[1]{{#1}}
    
    % Additional commands for more recent versions of Pandoc
    \newcommand{\ConstantTok}[1]{\textcolor[rgb]{0.53,0.00,0.00}{{#1}}}
    \newcommand{\SpecialCharTok}[1]{\textcolor[rgb]{0.25,0.44,0.63}{{#1}}}
    \newcommand{\VerbatimStringTok}[1]{\textcolor[rgb]{0.25,0.44,0.63}{{#1}}}
    \newcommand{\SpecialStringTok}[1]{\textcolor[rgb]{0.73,0.40,0.53}{{#1}}}
    \newcommand{\ImportTok}[1]{{#1}}
    \newcommand{\DocumentationTok}[1]{\textcolor[rgb]{0.73,0.13,0.13}{\textit{{#1}}}}
    \newcommand{\AnnotationTok}[1]{\textcolor[rgb]{0.38,0.63,0.69}{\textbf{\textit{{#1}}}}}
    \newcommand{\CommentVarTok}[1]{\textcolor[rgb]{0.38,0.63,0.69}{\textbf{\textit{{#1}}}}}
    \newcommand{\VariableTok}[1]{\textcolor[rgb]{0.10,0.09,0.49}{{#1}}}
    \newcommand{\ControlFlowTok}[1]{\textcolor[rgb]{0.00,0.44,0.13}{\textbf{{#1}}}}
    \newcommand{\OperatorTok}[1]{\textcolor[rgb]{0.40,0.40,0.40}{{#1}}}
    \newcommand{\BuiltInTok}[1]{{#1}}
    \newcommand{\ExtensionTok}[1]{{#1}}
    \newcommand{\PreprocessorTok}[1]{\textcolor[rgb]{0.74,0.48,0.00}{{#1}}}
    \newcommand{\AttributeTok}[1]{\textcolor[rgb]{0.49,0.56,0.16}{{#1}}}
    \newcommand{\InformationTok}[1]{\textcolor[rgb]{0.38,0.63,0.69}{\textbf{\textit{{#1}}}}}
    \newcommand{\WarningTok}[1]{\textcolor[rgb]{0.38,0.63,0.69}{\textbf{\textit{{#1}}}}}
    
    
    % Define a nice break command that doesn't care if a line doesn't already
    % exist.
    \def\br{\hspace*{\fill} \\* }
    % Math Jax compatibility definitions
    \def\gt{>}
    \def\lt{<}
    \let\Oldtex\TeX
    \let\Oldlatex\LaTeX
    \renewcommand{\TeX}{\textrm{\Oldtex}}
    \renewcommand{\LaTeX}{\textrm{\Oldlatex}}
    % Document parameters
    % Document title
    \title{Actividad 2: Análisis sintáctico}
    
    
    
    
    
% Pygments definitions
\makeatletter
\def\PY@reset{\let\PY@it=\relax \let\PY@bf=\relax%
    \let\PY@ul=\relax \let\PY@tc=\relax%
    \let\PY@bc=\relax \let\PY@ff=\relax}
\def\PY@tok#1{\csname PY@tok@#1\endcsname}
\def\PY@toks#1+{\ifx\relax#1\empty\else%
    \PY@tok{#1}\expandafter\PY@toks\fi}
\def\PY@do#1{\PY@bc{\PY@tc{\PY@ul{%
    \PY@it{\PY@bf{\PY@ff{#1}}}}}}}
\def\PY#1#2{\PY@reset\PY@toks#1+\relax+\PY@do{#2}}

\@namedef{PY@tok@w}{\def\PY@tc##1{\textcolor[rgb]{0.73,0.73,0.73}{##1}}}
\@namedef{PY@tok@c}{\let\PY@it=\textit\def\PY@tc##1{\textcolor[rgb]{0.24,0.48,0.48}{##1}}}
\@namedef{PY@tok@cp}{\def\PY@tc##1{\textcolor[rgb]{0.61,0.40,0.00}{##1}}}
\@namedef{PY@tok@k}{\let\PY@bf=\textbf\def\PY@tc##1{\textcolor[rgb]{0.00,0.50,0.00}{##1}}}
\@namedef{PY@tok@kp}{\def\PY@tc##1{\textcolor[rgb]{0.00,0.50,0.00}{##1}}}
\@namedef{PY@tok@kt}{\def\PY@tc##1{\textcolor[rgb]{0.69,0.00,0.25}{##1}}}
\@namedef{PY@tok@o}{\def\PY@tc##1{\textcolor[rgb]{0.40,0.40,0.40}{##1}}}
\@namedef{PY@tok@ow}{\let\PY@bf=\textbf\def\PY@tc##1{\textcolor[rgb]{0.67,0.13,1.00}{##1}}}
\@namedef{PY@tok@nb}{\def\PY@tc##1{\textcolor[rgb]{0.00,0.50,0.00}{##1}}}
\@namedef{PY@tok@nf}{\def\PY@tc##1{\textcolor[rgb]{0.00,0.00,1.00}{##1}}}
\@namedef{PY@tok@nc}{\let\PY@bf=\textbf\def\PY@tc##1{\textcolor[rgb]{0.00,0.00,1.00}{##1}}}
\@namedef{PY@tok@nn}{\let\PY@bf=\textbf\def\PY@tc##1{\textcolor[rgb]{0.00,0.00,1.00}{##1}}}
\@namedef{PY@tok@ne}{\let\PY@bf=\textbf\def\PY@tc##1{\textcolor[rgb]{0.80,0.25,0.22}{##1}}}
\@namedef{PY@tok@nv}{\def\PY@tc##1{\textcolor[rgb]{0.10,0.09,0.49}{##1}}}
\@namedef{PY@tok@no}{\def\PY@tc##1{\textcolor[rgb]{0.53,0.00,0.00}{##1}}}
\@namedef{PY@tok@nl}{\def\PY@tc##1{\textcolor[rgb]{0.46,0.46,0.00}{##1}}}
\@namedef{PY@tok@ni}{\let\PY@bf=\textbf\def\PY@tc##1{\textcolor[rgb]{0.44,0.44,0.44}{##1}}}
\@namedef{PY@tok@na}{\def\PY@tc##1{\textcolor[rgb]{0.41,0.47,0.13}{##1}}}
\@namedef{PY@tok@nt}{\let\PY@bf=\textbf\def\PY@tc##1{\textcolor[rgb]{0.00,0.50,0.00}{##1}}}
\@namedef{PY@tok@nd}{\def\PY@tc##1{\textcolor[rgb]{0.67,0.13,1.00}{##1}}}
\@namedef{PY@tok@s}{\def\PY@tc##1{\textcolor[rgb]{0.73,0.13,0.13}{##1}}}
\@namedef{PY@tok@sd}{\let\PY@it=\textit\def\PY@tc##1{\textcolor[rgb]{0.73,0.13,0.13}{##1}}}
\@namedef{PY@tok@si}{\let\PY@bf=\textbf\def\PY@tc##1{\textcolor[rgb]{0.64,0.35,0.47}{##1}}}
\@namedef{PY@tok@se}{\let\PY@bf=\textbf\def\PY@tc##1{\textcolor[rgb]{0.67,0.36,0.12}{##1}}}
\@namedef{PY@tok@sr}{\def\PY@tc##1{\textcolor[rgb]{0.64,0.35,0.47}{##1}}}
\@namedef{PY@tok@ss}{\def\PY@tc##1{\textcolor[rgb]{0.10,0.09,0.49}{##1}}}
\@namedef{PY@tok@sx}{\def\PY@tc##1{\textcolor[rgb]{0.00,0.50,0.00}{##1}}}
\@namedef{PY@tok@m}{\def\PY@tc##1{\textcolor[rgb]{0.40,0.40,0.40}{##1}}}
\@namedef{PY@tok@gh}{\let\PY@bf=\textbf\def\PY@tc##1{\textcolor[rgb]{0.00,0.00,0.50}{##1}}}
\@namedef{PY@tok@gu}{\let\PY@bf=\textbf\def\PY@tc##1{\textcolor[rgb]{0.50,0.00,0.50}{##1}}}
\@namedef{PY@tok@gd}{\def\PY@tc##1{\textcolor[rgb]{0.63,0.00,0.00}{##1}}}
\@namedef{PY@tok@gi}{\def\PY@tc##1{\textcolor[rgb]{0.00,0.52,0.00}{##1}}}
\@namedef{PY@tok@gr}{\def\PY@tc##1{\textcolor[rgb]{0.89,0.00,0.00}{##1}}}
\@namedef{PY@tok@ge}{\let\PY@it=\textit}
\@namedef{PY@tok@gs}{\let\PY@bf=\textbf}
\@namedef{PY@tok@gp}{\let\PY@bf=\textbf\def\PY@tc##1{\textcolor[rgb]{0.00,0.00,0.50}{##1}}}
\@namedef{PY@tok@go}{\def\PY@tc##1{\textcolor[rgb]{0.44,0.44,0.44}{##1}}}
\@namedef{PY@tok@gt}{\def\PY@tc##1{\textcolor[rgb]{0.00,0.27,0.87}{##1}}}
\@namedef{PY@tok@err}{\def\PY@bc##1{{\setlength{\fboxsep}{\string -\fboxrule}\fcolorbox[rgb]{1.00,0.00,0.00}{1,1,1}{\strut ##1}}}}
\@namedef{PY@tok@kc}{\let\PY@bf=\textbf\def\PY@tc##1{\textcolor[rgb]{0.00,0.50,0.00}{##1}}}
\@namedef{PY@tok@kd}{\let\PY@bf=\textbf\def\PY@tc##1{\textcolor[rgb]{0.00,0.50,0.00}{##1}}}
\@namedef{PY@tok@kn}{\let\PY@bf=\textbf\def\PY@tc##1{\textcolor[rgb]{0.00,0.50,0.00}{##1}}}
\@namedef{PY@tok@kr}{\let\PY@bf=\textbf\def\PY@tc##1{\textcolor[rgb]{0.00,0.50,0.00}{##1}}}
\@namedef{PY@tok@bp}{\def\PY@tc##1{\textcolor[rgb]{0.00,0.50,0.00}{##1}}}
\@namedef{PY@tok@fm}{\def\PY@tc##1{\textcolor[rgb]{0.00,0.00,1.00}{##1}}}
\@namedef{PY@tok@vc}{\def\PY@tc##1{\textcolor[rgb]{0.10,0.09,0.49}{##1}}}
\@namedef{PY@tok@vg}{\def\PY@tc##1{\textcolor[rgb]{0.10,0.09,0.49}{##1}}}
\@namedef{PY@tok@vi}{\def\PY@tc##1{\textcolor[rgb]{0.10,0.09,0.49}{##1}}}
\@namedef{PY@tok@vm}{\def\PY@tc##1{\textcolor[rgb]{0.10,0.09,0.49}{##1}}}
\@namedef{PY@tok@sa}{\def\PY@tc##1{\textcolor[rgb]{0.73,0.13,0.13}{##1}}}
\@namedef{PY@tok@sb}{\def\PY@tc##1{\textcolor[rgb]{0.73,0.13,0.13}{##1}}}
\@namedef{PY@tok@sc}{\def\PY@tc##1{\textcolor[rgb]{0.73,0.13,0.13}{##1}}}
\@namedef{PY@tok@dl}{\def\PY@tc##1{\textcolor[rgb]{0.73,0.13,0.13}{##1}}}
\@namedef{PY@tok@s2}{\def\PY@tc##1{\textcolor[rgb]{0.73,0.13,0.13}{##1}}}
\@namedef{PY@tok@sh}{\def\PY@tc##1{\textcolor[rgb]{0.73,0.13,0.13}{##1}}}
\@namedef{PY@tok@s1}{\def\PY@tc##1{\textcolor[rgb]{0.73,0.13,0.13}{##1}}}
\@namedef{PY@tok@mb}{\def\PY@tc##1{\textcolor[rgb]{0.40,0.40,0.40}{##1}}}
\@namedef{PY@tok@mf}{\def\PY@tc##1{\textcolor[rgb]{0.40,0.40,0.40}{##1}}}
\@namedef{PY@tok@mh}{\def\PY@tc##1{\textcolor[rgb]{0.40,0.40,0.40}{##1}}}
\@namedef{PY@tok@mi}{\def\PY@tc##1{\textcolor[rgb]{0.40,0.40,0.40}{##1}}}
\@namedef{PY@tok@il}{\def\PY@tc##1{\textcolor[rgb]{0.40,0.40,0.40}{##1}}}
\@namedef{PY@tok@mo}{\def\PY@tc##1{\textcolor[rgb]{0.40,0.40,0.40}{##1}}}
\@namedef{PY@tok@ch}{\let\PY@it=\textit\def\PY@tc##1{\textcolor[rgb]{0.24,0.48,0.48}{##1}}}
\@namedef{PY@tok@cm}{\let\PY@it=\textit\def\PY@tc##1{\textcolor[rgb]{0.24,0.48,0.48}{##1}}}
\@namedef{PY@tok@cpf}{\let\PY@it=\textit\def\PY@tc##1{\textcolor[rgb]{0.24,0.48,0.48}{##1}}}
\@namedef{PY@tok@c1}{\let\PY@it=\textit\def\PY@tc##1{\textcolor[rgb]{0.24,0.48,0.48}{##1}}}
\@namedef{PY@tok@cs}{\let\PY@it=\textit\def\PY@tc##1{\textcolor[rgb]{0.24,0.48,0.48}{##1}}}

\def\PYZbs{\char`\\}
\def\PYZus{\char`\_}
\def\PYZob{\char`\{}
\def\PYZcb{\char`\}}
\def\PYZca{\char`\^}
\def\PYZam{\char`\&}
\def\PYZlt{\char`\<}
\def\PYZgt{\char`\>}
\def\PYZsh{\char`\#}
\def\PYZpc{\char`\%}
\def\PYZdl{\char`\$}
\def\PYZhy{\char`\-}
\def\PYZsq{\char`\'}
\def\PYZdq{\char`\"}
\def\PYZti{\char`\~}
% for compatibility with earlier versions
\def\PYZat{@}
\def\PYZlb{[}
\def\PYZrb{]}
\makeatother


    % For linebreaks inside Verbatim environment from package fancyvrb. 
    \makeatletter
        \newbox\Wrappedcontinuationbox 
        \newbox\Wrappedvisiblespacebox 
        \newcommand*\Wrappedvisiblespace {\textcolor{red}{\textvisiblespace}} 
        \newcommand*\Wrappedcontinuationsymbol {\textcolor{red}{\llap{\tiny$\m@th\hookrightarrow$}}} 
        \newcommand*\Wrappedcontinuationindent {3ex } 
        \newcommand*\Wrappedafterbreak {\kern\Wrappedcontinuationindent\copy\Wrappedcontinuationbox} 
        % Take advantage of the already applied Pygments mark-up to insert 
        % potential linebreaks for TeX processing. 
        %        {, <, #, %, $, ' and ": go to next line. 
        %        _, }, ^, &, >, - and ~: stay at end of broken line. 
        % Use of \textquotesingle for straight quote. 
        \newcommand*\Wrappedbreaksatspecials {% 
            \def\PYGZus{\discretionary{\char`\_}{\Wrappedafterbreak}{\char`\_}}% 
            \def\PYGZob{\discretionary{}{\Wrappedafterbreak\char`\{}{\char`\{}}% 
            \def\PYGZcb{\discretionary{\char`\}}{\Wrappedafterbreak}{\char`\}}}% 
            \def\PYGZca{\discretionary{\char`\^}{\Wrappedafterbreak}{\char`\^}}% 
            \def\PYGZam{\discretionary{\char`\&}{\Wrappedafterbreak}{\char`\&}}% 
            \def\PYGZlt{\discretionary{}{\Wrappedafterbreak\char`\<}{\char`\<}}% 
            \def\PYGZgt{\discretionary{\char`\>}{\Wrappedafterbreak}{\char`\>}}% 
            \def\PYGZsh{\discretionary{}{\Wrappedafterbreak\char`\#}{\char`\#}}% 
            \def\PYGZpc{\discretionary{}{\Wrappedafterbreak\char`\%}{\char`\%}}% 
            \def\PYGZdl{\discretionary{}{\Wrappedafterbreak\char`\$}{\char`\$}}% 
            \def\PYGZhy{\discretionary{\char`\-}{\Wrappedafterbreak}{\char`\-}}% 
            \def\PYGZsq{\discretionary{}{\Wrappedafterbreak\textquotesingle}{\textquotesingle}}% 
            \def\PYGZdq{\discretionary{}{\Wrappedafterbreak\char`\"}{\char`\"}}% 
            \def\PYGZti{\discretionary{\char`\~}{\Wrappedafterbreak}{\char`\~}}% 
        } 
        % Some characters . , ; ? ! / are not pygmentized. 
        % This macro makes them "active" and they will insert potential linebreaks 
        \newcommand*\Wrappedbreaksatpunct {% 
            \lccode`\~`\.\lowercase{\def~}{\discretionary{\hbox{\char`\.}}{\Wrappedafterbreak}{\hbox{\char`\.}}}% 
            \lccode`\~`\,\lowercase{\def~}{\discretionary{\hbox{\char`\,}}{\Wrappedafterbreak}{\hbox{\char`\,}}}% 
            \lccode`\~`\;\lowercase{\def~}{\discretionary{\hbox{\char`\;}}{\Wrappedafterbreak}{\hbox{\char`\;}}}% 
            \lccode`\~`\:\lowercase{\def~}{\discretionary{\hbox{\char`\:}}{\Wrappedafterbreak}{\hbox{\char`\:}}}% 
            \lccode`\~`\?\lowercase{\def~}{\discretionary{\hbox{\char`\?}}{\Wrappedafterbreak}{\hbox{\char`\?}}}% 
            \lccode`\~`\!\lowercase{\def~}{\discretionary{\hbox{\char`\!}}{\Wrappedafterbreak}{\hbox{\char`\!}}}% 
            \lccode`\~`\/\lowercase{\def~}{\discretionary{\hbox{\char`\/}}{\Wrappedafterbreak}{\hbox{\char`\/}}}% 
            \catcode`\.\active
            \catcode`\,\active 
            \catcode`\;\active
            \catcode`\:\active
            \catcode`\?\active
            \catcode`\!\active
            \catcode`\/\active 
            \lccode`\~`\~ 	
        }
    \makeatother

    \let\OriginalVerbatim=\Verbatim
    \makeatletter
    \renewcommand{\Verbatim}[1][1]{%
        %\parskip\z@skip
        \sbox\Wrappedcontinuationbox {\Wrappedcontinuationsymbol}%
        \sbox\Wrappedvisiblespacebox {\FV@SetupFont\Wrappedvisiblespace}%
        \def\FancyVerbFormatLine ##1{\hsize\linewidth
            \vtop{\raggedright\hyphenpenalty\z@\exhyphenpenalty\z@
                \doublehyphendemerits\z@\finalhyphendemerits\z@
                \strut ##1\strut}%
        }%
        % If the linebreak is at a space, the latter will be displayed as visible
        % space at end of first line, and a continuation symbol starts next line.
        % Stretch/shrink are however usually zero for typewriter font.
        \def\FV@Space {%
            \nobreak\hskip\z@ plus\fontdimen3\font minus\fontdimen4\font
            \discretionary{\copy\Wrappedvisiblespacebox}{\Wrappedafterbreak}
            {\kern\fontdimen2\font}%
        }%
        
        % Allow breaks at special characters using \PYG... macros.
        \Wrappedbreaksatspecials
        % Breaks at punctuation characters . , ; ? ! and / need catcode=\active 	
        \OriginalVerbatim[#1,codes*=\Wrappedbreaksatpunct]%
    }
    \makeatother

    % Exact colors from NB
    \definecolor{incolor}{HTML}{303F9F}
    \definecolor{outcolor}{HTML}{D84315}
    \definecolor{cellborder}{HTML}{CFCFCF}
    \definecolor{cellbackground}{HTML}{F7F7F7}
    
    % prompt
    \makeatletter
    \newcommand{\boxspacing}{\kern\kvtcb@left@rule\kern\kvtcb@boxsep}
    \makeatother
    \newcommand{\prompt}[4]{
        {\ttfamily\llap{{\color{#2}[#3]:\hspace{3pt}#4}}\vspace{-\baselineskip}}
    }
    

    
    % Prevent overflowing lines due to hard-to-break entities
    \sloppy 
    % Setup hyperref package
    \hypersetup{
      breaklinks=true,  % so long urls are correctly broken across lines
      colorlinks=true,
      urlcolor=urlcolor,
      linkcolor=linkcolor,
      citecolor=citecolor,
      }
    % Slightly bigger margins than the latex defaults
    
    \geometry{verbose,tmargin=1in,bmargin=1in,lmargin=1in,rmargin=1in}
    
    %% BEGIN: UNIR
    \usepackage[spanish,mexico]{babel}
    \usepackage[sfdefault,lf]{carlito}
    \makeatletter
    \let\newtitle\@title
    \makeatother
    \usepackage{amsmath}
    \usepackage{multirow}
    \definecolor{UnirLight}{HTML}{E6F4F9}
    \definecolor{UnirDark}{HTML}{0098CD}
    \arrayrulecolor{UnirDark}
    \usepackage{titlesec}
    \titleformat*{\section}{\color{UnirDark}\normalsize\bfseries}
    \titleformat*{\subsection}{\color{UnirDark}\normalsize\bfseries}
    \titleformat*{\subsubsection}{\color{UnirDark}\normalsize\bfseries}
    \usepackage{fancyhdr}
    \pagestyle{fancy}
    \renewcommand{\headrulewidth}{0pt}
    \headheight=60pt
    \setlength{\footskip}{64pt}
    \linespread{1.5}
    \lhead{}
    \chead{
    \begin{tabular}{|c|l|c|}
     \hline
     \rowcolor{UnirLight}
     \textcolor{UnirDark}{Asignatura} & \textcolor{UnirDark}{Datos del alumno} & \textcolor{UnirDark}{Fecha} \\
     \hline
     & Bernal Castillo Aldo Alberto & \\
     \textbf{Aprendizaje Automático} & Domínguez Espinoza Edgar Uriel & \today \\
     & Valdivia Medina Frank Edil & \\
     \hline      
    \end{tabular}}
    \rhead{}
    \lfoot{}
    \cfoot{}
    \rfoot{\makebox(70,56)[t]{\textcolor{UnirDark}{Actividades}}
        \colorbox{UnirDark}{
            \makebox(10,56)[t]{
                \textcolor{white}{\thepage}}}}
    \usepackage[color={[gray]{0.5}}, angle=90,fontsize=9pt,anchor=lb,pos={0.03\paperwidth,0.95\paperheight}]{draftwatermark}
    \SetWatermarkText{{\copyright} Universidad Internacional de La Rioja en México (UNIR)}
    \hypersetup{
      pdfauthor={Edgar Uriel Domínguez Espinoza},
      pdftitle={Análisis sintáctico},
      pdfkeywords={nlp,cky,pcky,análisis sintáctico},
      pdfsubject={Procesamiento del lenguaje natural},
      pdfcreator={Emacs 28.1}, 
      pdflang={Spanish}}
    \usepackage[round]{natbib}
    \usepackage{tikz}
    \usepackage{tikz-qtree}
    %% END: UNIR    


\begin{document}
    
    
    

    
    \hypertarget{implementaciuxf3n-del-algoritmo-pcky}{%
%%       \section{Implementación del algoritmo PCKY}
      \textcolor{UnirDark}{\Large\bfseries\newtitle}
\label{implementaciuxf3n-del-algoritmo-pcky}}

\hypertarget{libreruxedas-a-utilizar}{%
\subsection*{Librerías a utilizar}\label{libreruxedas-a-utilizar}}

Para la implementación del algoritmo CKY solo serán necesarias tres
paquetes adicionales, el primero (\texttt{itemgetter}) permite seleccionar un elemento de una
lista a partir del índice, el segundo (\texttt{pprint}) mostrará de manera más clara los
resultados de los árboles y finalmente, pandas que permitirá mostrar la
matriz CKY en forma tabular, como un dataframe.

%%     \begin{tcolorbox}[breakable, size=fbox, boxrule=1pt, pad at break*=1mm,colback=cellbackground, colframe=cellborder]
%% \prompt{In}{incolor}{ }{\boxspacing}
%% \begin{Verbatim}[commandchars=\\\{\}]
%% \PY{k+kn}{from} \PY{n+nn}{operator} \PY{k+kn}{import} \PY{n}{itemgetter}
%% \PY{k+kn}{from} \PY{n+nn}{pprint} \PY{k+kn}{import} \PY{n}{pprint}
%% \PY{k+kn}{import} \PY{n+nn}{pandas} \PY{k}{as} \PY{n+nn}{pd}
%% \end{Verbatim}
%% \end{tcolorbox}

    \hypertarget{datos-de-entrada}{%
\subsection*{Datos de entrada}\label{datos-de-entrada}}

La variable \texttt{grammar\_file} que aparece en el código contendrá las
reglas gramaticales escritas en un archivo de texto. En este caso se han
escrito las reglas propias del ejercicio, pero pueden variar (ver notebook adjunto).

%%     \begin{tcolorbox}[breakable, size=fbox, boxrule=1pt, pad at break*=1mm,colback=cellbackground, colframe=cellborder]
%% \prompt{In}{incolor}{ }{\boxspacing}
%% \begin{Verbatim}[commandchars=\\\{\}]
%% \PY{n}{file} \PY{o}{=} \PY{n+nb}{open}\PY{p}{(}\PY{l+s+s2}{\PYZdq{}}\PY{l+s+s2}{ds/grammar.txt}\PY{l+s+s2}{\PYZdq{}}\PY{p}{,}\PY{l+s+s1}{\PYZsq{}}\PY{l+s+s1}{r}\PY{l+s+s1}{\PYZsq{}}\PY{p}{)}
%% \PY{n}{grammar\PYZus{}file} \PY{o}{=} \PY{n}{file}\PY{o}{.}\PY{n}{read}\PY{p}{(}\PY{p}{)}
%% \end{Verbatim}
%% \end{tcolorbox}

    La función \texttt{read\_rules()} (ver notebook adjunto) recibe como argumento un texto con
reglas gramaticales y probabilidades para convierlas en tres
diccionarios que permiten buscar los datos necesarios para completar los
algoritmos CKY y PCKY.

Se crea el lexicón, de forma común, en lingüística se dice que un
elemento del lexicón en la entrada de un diccionario, entonces se define
como tal en el código.

Se expresan las reglas gramaticales y léxicas en forma de tuplas como
claves de una probabilidad que se usará para obtener las probabilidades
de cada árbol obtenido por el algoritmo. Se eligen tuplas porque son
fáciles de manejar, similar a los lenguajes Lisp.

%%     \begin{tcolorbox}[breakable, size=fbox, boxrule=1pt, pad at break*=1mm,colback=cellbackground, colframe=cellborder]
%% \prompt{In}{incolor}{ }{\boxspacing}
%% \begin{Verbatim}[commandchars=\\\{\}]
%% \PY{k}{def} \PY{n+nf}{read\PYZus{}rules}\PY{p}{(}\PY{n}{grammar\PYZus{}file}\PY{p}{)}\PY{p}{:}
%%     \PY{n}{gr\PYZus{}temp\PYZus{}rules} \PY{o}{=} \PY{n+nb}{list}\PY{p}{(}\PY{p}{)}
%%     \PY{n}{lexicon\PYZus{}rules} \PY{o}{=} \PY{n+nb}{dict}\PY{p}{(}\PY{p}{)}
%%     \PY{n}{probabilities} \PY{o}{=} \PY{n+nb}{dict}\PY{p}{(}\PY{p}{)}
%%     \PY{n}{grammar\PYZus{}rules} \PY{o}{=} \PY{n+nb}{dict}\PY{p}{(}\PY{p}{)}
%%     \PY{k}{for} \PY{n}{line} \PY{o+ow}{in} \PY{n}{grammar\PYZus{}file}\PY{o}{.}\PY{n}{strip}\PY{p}{(}\PY{p}{)}\PY{o}{.}\PY{n}{split}\PY{p}{(}\PY{l+s+s2}{\PYZdq{}}\PY{l+s+se}{\PYZbs{}n}\PY{l+s+s2}{\PYZdq{}}\PY{p}{)}\PY{p}{:}
%%         \PY{n}{parent}\PY{p}{,} \PY{n}{children} \PY{o}{=} \PY{n}{line}\PY{o}{.}\PY{n}{strip}\PY{p}{(}\PY{p}{)}\PY{o}{.}\PY{n}{split}\PY{p}{(}\PY{l+s+s2}{\PYZdq{}}\PY{l+s+s2}{\PYZhy{}\PYZgt{}}\PY{l+s+s2}{\PYZdq{}}\PY{p}{)}
%%         \PY{n}{parent} \PY{o}{=} \PY{n}{parent}\PY{o}{.}\PY{n}{strip}\PY{p}{(}\PY{p}{)}
%%         \PY{n}{children} \PY{o}{=} \PY{n}{children}\PY{o}{.}\PY{n}{split}\PY{p}{(}\PY{p}{)}
%%         \PY{k}{if} \PY{n+nb}{len}\PY{p}{(}\PY{n}{children}\PY{p}{)} \PY{o}{==} \PY{l+m+mi}{2}\PY{p}{:}
%%             \PY{k}{if} \PY{n}{parent} \PY{o+ow}{not} \PY{o+ow}{in} \PY{n}{lexicon\PYZus{}rules}\PY{p}{:}
%%                 \PY{n}{lexicon\PYZus{}rules}\PY{p}{[}\PY{n}{parent}\PY{p}{]} \PY{o}{=} \PY{n+nb}{set}\PY{p}{(}\PY{p}{)}
%%             \PY{n}{lexicon\PYZus{}rules}\PY{p}{[}\PY{n}{parent}\PY{p}{]}\PY{o}{.}\PY{n}{add}\PY{p}{(}\PY{n}{children}\PY{p}{[}\PY{l+m+mi}{0}\PY{p}{]}\PY{p}{)}
%%             \PY{n}{probabilities}\PY{p}{[}\PY{n}{parent}\PY{p}{,}\PY{n}{children}\PY{p}{[}\PY{l+m+mi}{0}\PY{p}{]}\PY{p}{]} \PY{o}{=} \PY{n+nb}{float}\PY{p}{(}\PY{n}{children}\PY{p}{[}\PY{l+m+mi}{1}\PY{p}{]}\PY{p}{)}
%%         \PY{k}{else}\PY{p}{:}
%%             \PY{n}{rule} \PY{o}{=} \PY{p}{(}\PY{n}{parent}\PY{p}{,} \PY{n+nb}{tuple}\PY{p}{(}\PY{n}{children}\PY{p}{[}\PY{p}{:}\PY{l+m+mi}{2}\PY{p}{]}\PY{p}{)}\PY{p}{)}
%%             \PY{n}{probabilities}\PY{p}{[}\PY{n}{rule}\PY{p}{]} \PY{o}{=} \PY{n+nb}{float}\PY{p}{(}\PY{n}{children}\PY{p}{[}\PY{l+m+mi}{2}\PY{p}{]}\PY{p}{)}
%%             \PY{n}{gr\PYZus{}temp\PYZus{}rules}\PY{o}{.}\PY{n}{append}\PY{p}{(}\PY{n}{rule}\PY{p}{)}
%%     \PY{k}{for} \PY{p}{(}\PY{n}{parent}\PY{p}{,} \PY{p}{(}\PY{n}{lc}\PY{p}{,} \PY{n}{rc}\PY{p}{)}\PY{p}{)} \PY{o+ow}{in} \PY{n}{gr\PYZus{}temp\PYZus{}rules}\PY{p}{:}
%%         \PY{k}{if} \PY{p}{(}\PY{n}{lc}\PY{p}{,} \PY{n}{rc}\PY{p}{)} \PY{o+ow}{not} \PY{o+ow}{in} \PY{n}{grammar\PYZus{}rules}\PY{p}{:}
%%             \PY{n}{grammar\PYZus{}rules}\PY{p}{[}\PY{n}{lc}\PY{p}{,} \PY{n}{rc}\PY{p}{]} \PY{o}{=} \PY{n+nb}{list}\PY{p}{(}\PY{p}{)}
%%         \PY{n}{grammar\PYZus{}rules}\PY{p}{[}\PY{n}{lc}\PY{p}{,}\PY{n}{rc}\PY{p}{]}\PY{o}{.}\PY{n}{append}\PY{p}{(}\PY{n}{parent}\PY{p}{)}
%%     \PY{k}{return} \PY{n}{grammar\PYZus{}rules}\PY{p}{,}\PY{n}{lexicon\PYZus{}rules}\PY{p}{,}\PY{n}{probabilities}
%% \end{Verbatim}
%% \end{tcolorbox}

    \hypertarget{algoritmos-cky-y-pcky}{%
\subsection*{Algoritmos CKY y PCKY}\label{algoritmos-cky-y-pcky}}

Ahora se implementa el algoritmo CKY, mismo que recibe como entrada una
oración en forma de cadena de texto, un diccionario con reglas
gramaticales y un lexicón adecuado a la gramática. Opcionalmente, se
puede definir el argumento de entrada \texttt{verbose} al valor
\texttt{True}, de esta manera se mostrará en pantalla cada uno de los
pasos que sigue el algoritmo para crear la matriz correspondiente. Esta
matriz también tendrá forma de diccionario (ver notebook anexo).

%%     \begin{tcolorbox}[breakable, size=fbox, boxrule=1pt, pad at break*=1mm,colback=cellbackground, colframe=cellborder]
%% \prompt{In}{incolor}{ }{\boxspacing}
%% \begin{Verbatim}[commandchars=\\\{\}]
%% \PY{k}{def} \PY{n+nf}{cky\PYZus{}parser}\PY{p}{(}\PY{n}{sentence}\PY{p}{,} \PY{n}{grammar}\PY{p}{,} \PY{n}{lexicon}\PY{p}{,} \PY{n}{verbose}\PY{o}{=}\PY{k+kc}{False}\PY{p}{)}\PY{p}{:}
%%     \PY{n}{words} \PY{o}{=} \PY{n}{sentence}\PY{o}{.}\PY{n}{lower}\PY{p}{(}\PY{p}{)}\PY{o}{.}\PY{n}{split}\PY{p}{(}\PY{p}{)}
%%     \PY{k}{if} \PY{n}{verbose}\PY{p}{:}
%%         \PY{n+nb}{print}\PY{p}{(}\PY{n}{words}\PY{p}{)}
%%     \PY{n}{matrix\PYZus{}cky} \PY{o}{=} \PY{n+nb}{dict}\PY{p}{(}\PY{p}{)}
%%     \PY{k}{for} \PY{n}{i} \PY{o+ow}{in} \PY{n+nb}{range}\PY{p}{(}\PY{n+nb}{len}\PY{p}{(}\PY{n}{words}\PY{p}{)}\PY{p}{)}\PY{p}{:}
%%         \PY{n}{ordinate} \PY{o}{=} \PY{l+m+mi}{0}
%%         \PY{k}{if} \PY{n}{verbose}\PY{p}{:}
%%             \PY{n+nb}{print}\PY{p}{(}\PY{l+s+s2}{\PYZdq{}}\PY{l+s+s2}{0: i=}\PY{l+s+s2}{\PYZdq{}}\PY{p}{,}\PY{n}{i}\PY{p}{)}
%%         \PY{k}{for} \PY{n}{abscissa} \PY{o+ow}{in} \PY{n+nb}{range}\PY{p}{(}\PY{n}{i}\PY{o}{+}\PY{l+m+mi}{1}\PY{p}{,} \PY{n+nb}{len}\PY{p}{(}\PY{n}{words}\PY{p}{)}\PY{o}{+}\PY{l+m+mi}{1}\PY{p}{)}\PY{p}{:}
%%             \PY{n}{matrix\PYZus{}cky}\PY{p}{[}\PY{p}{(}\PY{n}{ordinate}\PY{p}{,} \PY{n}{abscissa}\PY{p}{)}\PY{p}{]} \PY{o}{=} \PY{n+nb}{list}\PY{p}{(}\PY{p}{)}
%%             \PY{k}{if} \PY{n}{verbose}\PY{p}{:}
%%                 \PY{n+nb}{print}\PY{p}{(}\PY{l+s+s2}{\PYZdq{}}\PY{l+s+s2}{1: abscissa=}\PY{l+s+s2}{\PYZdq{}}\PY{p}{,}\PY{n}{abscissa}\PY{p}{,}\PY{l+s+s2}{\PYZdq{}}\PY{l+s+s2}{ordinate=}\PY{l+s+s2}{\PYZdq{}}\PY{p}{,}\PY{n}{ordinate}\PY{p}{,}\PY{l+s+s2}{\PYZdq{}}\PY{l+s+s2}{a\PYZhy{}o=}\PY{l+s+s2}{\PYZdq{}}\PY{p}{,}\PY{n}{abscissa}\PY{o}{\PYZhy{}}\PY{n}{ordinate}\PY{p}{)}
%%             \PY{k}{if}\PY{p}{(}\PY{n}{abscissa}\PY{o}{\PYZhy{}}\PY{n}{ordinate} \PY{o}{==} \PY{l+m+mi}{1}\PY{p}{)}\PY{p}{:}
%%                 \PY{k}{for} \PY{n}{key} \PY{o+ow}{in} \PY{n}{lexicon}\PY{p}{:}
%%                     \PY{k}{if} \PY{n}{verbose}\PY{p}{:}
%%                         \PY{n+nb}{print}\PY{p}{(}\PY{l+s+s2}{\PYZdq{}}\PY{l+s+s2}{2: key=}\PY{l+s+s2}{\PYZdq{}}\PY{p}{,}\PY{n}{key}\PY{p}{,}\PY{l+s+s2}{\PYZdq{}}\PY{l+s+s2}{word=}\PY{l+s+s2}{\PYZdq{}}\PY{p}{,}\PY{n}{words}\PY{p}{[}\PY{n}{ordinate}\PY{p}{]}\PY{p}{)}
%%                     \PY{k}{if}\PY{p}{(}\PY{n}{words}\PY{p}{[}\PY{n}{ordinate}\PY{p}{]} \PY{o+ow}{in} \PY{n}{lexicon}\PY{p}{[}\PY{n}{key}\PY{p}{]}\PY{p}{)}\PY{p}{:}
%%                         \PY{n}{matrix\PYZus{}cky}\PY{p}{[}\PY{p}{(}\PY{n}{ordinate}\PY{p}{,}\PY{n}{abscissa}\PY{p}{)}\PY{p}{]}\PY{o}{.}\PY{n}{append}\PY{p}{(}
%%                             \PY{p}{(}\PY{n}{key}\PY{p}{,}\PY{l+m+mi}{0}\PY{p}{,}\PY{n}{words}\PY{p}{[}\PY{n}{ordinate}\PY{p}{]}\PY{p}{,}\PY{n}{words}\PY{p}{[}\PY{n}{ordinate}\PY{p}{]}\PY{p}{)}\PY{p}{)}
%%                         \PY{k}{if} \PY{n}{verbose}\PY{p}{:}
%%                             \PY{n+nb}{print}\PY{p}{(}\PY{l+s+s2}{\PYZdq{}}\PY{l+s+s2}{2:}\PY{l+s+s2}{\PYZdq{}}\PY{p}{,}\PY{n}{matrix\PYZus{}cky}\PY{p}{)}
%%             \PY{k}{elif}\PY{p}{(}\PY{n}{abscissa}\PY{o}{\PYZhy{}}\PY{n}{ordinate} \PY{o}{\PYZgt{}} \PY{l+m+mi}{1}\PY{p}{)}\PY{p}{:}
%%                 \PY{k}{if} \PY{n}{verbose}\PY{p}{:}
%%                     \PY{n+nb}{print}\PY{p}{(}\PY{l+s+s2}{\PYZdq{}}\PY{l+s+s2}{3:}\PY{l+s+s2}{\PYZdq{}}\PY{p}{,}\PY{l+s+s2}{\PYZdq{}}\PY{l+s+s2}{(}\PY{l+s+s2}{\PYZdq{}}\PY{p}{,}\PY{n}{ordinate}\PY{p}{,}\PY{l+s+s2}{\PYZdq{}}\PY{l+s+s2}{,}\PY{l+s+s2}{\PYZdq{}}\PY{p}{,}\PY{n}{abscissa}\PY{p}{,}\PY{l+s+s2}{\PYZdq{}}\PY{l+s+s2}{)}\PY{l+s+s2}{\PYZdq{}}\PY{p}{)}
%%                 \PY{k}{for} \PY{n}{index} \PY{o+ow}{in} \PY{n+nb}{range}\PY{p}{(}\PY{n}{abscissa}\PY{o}{\PYZhy{}}\PY{n}{ordinate}\PY{o}{\PYZhy{}}\PY{l+m+mi}{1}\PY{p}{)}\PY{p}{:}
%%                     \PY{n}{left} \PY{o}{=} \PY{n}{matrix\PYZus{}cky}\PY{p}{[}\PY{p}{(}\PY{n}{ordinate}\PY{p}{,}\PY{n}{abscissa}\PY{o}{\PYZhy{}}\PY{l+m+mi}{1}\PY{o}{\PYZhy{}}\PY{n}{index}\PY{p}{)}\PY{p}{]}
%%                     \PY{n}{down} \PY{o}{=} \PY{n}{matrix\PYZus{}cky}\PY{p}{[}\PY{p}{(}\PY{n}{abscissa}\PY{o}{\PYZhy{}}\PY{l+m+mi}{1}\PY{o}{\PYZhy{}}\PY{n}{index}\PY{p}{,}\PY{n}{abscissa}\PY{p}{)}\PY{p}{]}
%%                     \PY{k}{if} \PY{n}{verbose}\PY{p}{:}
%%                         \PY{n+nb}{print}\PY{p}{(}\PY{l+s+s2}{\PYZdq{}}\PY{l+s+s2}{4: index=}\PY{l+s+s2}{\PYZdq{}}\PY{p}{,}\PY{n}{index}\PY{p}{)}
%%                         \PY{n+nb}{print}\PY{p}{(}\PY{l+s+s2}{\PYZdq{}}\PY{l+s+s2}{4: left=}\PY{l+s+s2}{\PYZdq{}}\PY{p}{,}\PY{n}{left}\PY{p}{,}\PY{l+s+s2}{\PYZdq{}}\PY{l+s+se}{\PYZbs{}n}\PY{l+s+s2}{  }\PY{l+s+s2}{\PYZdq{}}\PY{p}{,}\PY{l+s+s2}{\PYZdq{}}\PY{l+s+s2}{down=}\PY{l+s+s2}{\PYZdq{}}\PY{p}{,}\PY{n}{down}\PY{p}{)}
%%                     \PY{k}{if} \PY{o+ow}{not} \PY{n}{left} \PY{o+ow}{or} \PY{o+ow}{not} \PY{n}{down}\PY{p}{:}
%%                         \PY{k}{if} \PY{n}{verbose}\PY{p}{:}
%%                             \PY{n+nb}{print}\PY{p}{(}\PY{l+s+s2}{\PYZdq{}}\PY{l+s+s2}{5: Nothing to do}\PY{l+s+s2}{\PYZdq{}}\PY{p}{)}
%%                     \PY{k}{else}\PY{p}{:}
%%                         \PY{k}{for} \PY{n}{a} \PY{o+ow}{in} \PY{n}{left}\PY{p}{:}
%%                             \PY{k}{for} \PY{n}{b} \PY{o+ow}{in} \PY{n}{down}\PY{p}{:}
%%                                 \PY{k}{if}\PY{p}{(}\PY{p}{(}\PY{n}{a}\PY{p}{[}\PY{l+m+mi}{0}\PY{p}{]}\PY{p}{,}\PY{n}{b}\PY{p}{[}\PY{l+m+mi}{0}\PY{p}{]}\PY{p}{)} \PY{o+ow}{in} \PY{n}{grammar}\PY{p}{)}\PY{p}{:}
%%                                     \PY{n}{matrix\PYZus{}cky}\PY{p}{[}\PY{p}{(}\PY{n}{ordinate}\PY{p}{,}\PY{n}{abscissa}\PY{p}{)}\PY{p}{]}\PY{o}{.}\PY{n}{append}\PY{p}{(}
%%                                         \PY{p}{(}\PY{n}{grammar}\PY{p}{[}\PY{p}{(}\PY{n}{a}\PY{p}{[}\PY{l+m+mi}{0}\PY{p}{]}\PY{p}{,}\PY{n}{b}\PY{p}{[}\PY{l+m+mi}{0}\PY{p}{]}\PY{p}{)}\PY{p}{]}\PY{p}{[}\PY{l+m+mi}{0}\PY{p}{]}\PY{p}{,}
%%                                         \PY{n}{abscissa}\PY{o}{\PYZhy{}}\PY{l+m+mi}{1}\PY{o}{\PYZhy{}}\PY{n}{index}\PY{p}{,}\PY{n}{a}\PY{p}{[}\PY{l+m+mi}{0}\PY{p}{]}\PY{p}{,}\PY{n}{b}\PY{p}{[}\PY{l+m+mi}{0}\PY{p}{]}\PY{p}{)}\PY{p}{)}
%%                                     \PY{k}{if} \PY{n}{verbose}\PY{p}{:}
%%                                         \PY{n+nb}{print}\PY{p}{(}\PY{l+s+s2}{\PYZdq{}}\PY{l+s+s2}{6: add=}\PY{l+s+s2}{\PYZdq{}}\PY{p}{,}\PY{n}{grammar}\PY{p}{[}\PY{p}{(}\PY{n}{a}\PY{p}{[}\PY{l+m+mi}{0}\PY{p}{]}\PY{p}{,}\PY{n}{b}\PY{p}{[}\PY{l+m+mi}{0}\PY{p}{]}\PY{p}{)}\PY{p}{]}\PY{p}{[}\PY{l+m+mi}{0}\PY{p}{]}\PY{p}{,}\PY{n}{abscissa}\PY{o}{\PYZhy{}}\PY{l+m+mi}{1}\PY{o}{\PYZhy{}}\PY{n}{index}\PY{p}{,}\PY{n}{a}\PY{p}{[}\PY{l+m+mi}{0}\PY{p}{]}\PY{p}{,}\PY{n}{b}\PY{p}{[}\PY{l+m+mi}{0}\PY{p}{]}\PY{p}{)}
%%                                 \PY{k}{else}\PY{p}{:}
%%                                     \PY{k}{if} \PY{n}{verbose}\PY{p}{:}
%%                                         \PY{n+nb}{print}\PY{p}{(}\PY{l+s+s2}{\PYZdq{}}\PY{l+s+s2}{6: Nothing to do}\PY{l+s+s2}{\PYZdq{}}\PY{p}{)}
%%             \PY{k}{if} \PY{n}{verbose}\PY{p}{:}
%%                 \PY{n+nb}{print}\PY{p}{(}\PY{n}{matrix\PYZus{}cky}\PY{p}{)}
%%                 \PY{n+nb}{print}\PY{p}{(}\PY{l+s+s2}{\PYZdq{}}\PY{l+s+s2}{9: Next Ordinate}\PY{l+s+s2}{\PYZdq{}}\PY{p}{)}
%%             \PY{n}{ordinate}\PY{o}{+}\PY{o}{=}\PY{l+m+mi}{1}
%%     \PY{k}{return} \PY{n}{matrix\PYZus{}cky}
%% \end{Verbatim}
%% \end{tcolorbox}

    Enseguida se crean dos funciones, la primera, \texttt{get\_prob()}, se
usará para obtener las probabilidades del diccionario de probabilidades,
la segunda, \texttt{update\_prob()}, se usará para actualizar las
probabilidades desde los nodos intermedios hasta el nodo padre (ver notebook anexo).

%%     \begin{tcolorbox}[breakable, size=fbox, boxrule=1pt, pad at break*=1mm,colback=cellbackground, colframe=cellborder]
%% \prompt{In}{incolor}{ }{\boxspacing}
%% \begin{Verbatim}[commandchars=\\\{\}]
%% \PY{k}{def} \PY{n+nf}{get\PYZus{}prob}\PY{p}{(}\PY{n}{node}\PY{p}{,} \PY{n}{prob\PYZus{}dict}\PY{p}{)}\PY{p}{:}
%%     \PY{k}{return} \PY{n}{prob\PYZus{}dict}\PY{p}{[}\PY{n}{node}\PY{p}{]}

%% \PY{k}{def} \PY{n+nf}{update\PYZus{}prob}\PY{p}{(}\PY{n}{left}\PY{p}{,} \PY{n}{down}\PY{p}{,} \PY{n}{actual}\PY{p}{,} \PY{n}{prob\PYZus{}dict}\PY{p}{)}\PY{p}{:}
%%     \PY{n}{actual\PYZus{}prob} \PY{o}{=} \PY{n}{get\PYZus{}prob}\PY{p}{(}\PY{p}{(}\PY{n}{actual}\PY{p}{,}\PY{p}{(}\PY{n}{left}\PY{p}{[}\PY{l+m+mi}{0}\PY{p}{]}\PY{p}{,}\PY{n}{down}\PY{p}{[}\PY{l+m+mi}{0}\PY{p}{]}\PY{p}{)}\PY{p}{)}\PY{p}{,}\PY{n}{prob\PYZus{}dict}\PY{p}{)}
%%     \PY{n}{prob} \PY{o}{=} \PY{n}{left}\PY{p}{[}\PY{l+m+mi}{4}\PY{p}{]}\PY{o}{*}\PY{n}{down}\PY{p}{[}\PY{l+m+mi}{4}\PY{p}{]}\PY{o}{*}\PY{n}{actual\PYZus{}prob}
%%     \PY{k}{return} \PY{n}{prob}
%% \end{Verbatim}
%% \end{tcolorbox}

    Se actualiza \texttt{cky\_parser()}. Ahora \texttt{pcky\_parser()}
agregará la información de la probabilidad a cada lista agregada a
\texttt{matrix\_pcky}, con esto se completa el algoritmo solicitado por la práctica.

    \begin{tcolorbox}[breakable, size=fbox, boxrule=1pt, pad at break*=1mm,colback=cellbackground, colframe=cellborder]
\prompt{In}{incolor}{ }{\boxspacing}
\begin{Verbatim}[commandchars=\\\{\}]
\PY{k}{def} \PY{n+nf}{pcky\PYZus{}parser}\PY{p}{(}\PY{n}{sentence}\PY{p}{,} \PY{n}{grammar}\PY{p}{,} \PY{n}{lexicon}\PY{p}{,} \PY{n}{probabilities\PYZus{}table}\PY{p}{,} \PY{n}{verbose}\PY{o}{=}\PY{k+kc}{False}\PY{p}{)}\PY{p}{:}
    \PY{n}{words} \PY{o}{=} \PY{n}{sentence}\PY{o}{.}\PY{n}{lower}\PY{p}{(}\PY{p}{)}\PY{o}{.}\PY{n}{split}\PY{p}{(}\PY{p}{)}
    \PY{k}{if} \PY{n}{verbose}\PY{p}{:}
        \PY{n+nb}{print}\PY{p}{(}\PY{n}{words}\PY{p}{)}
    \PY{n}{matrix\PYZus{}pcky} \PY{o}{=} \PY{n+nb}{dict}\PY{p}{(}\PY{p}{)}
    \PY{k}{for} \PY{n}{i} \PY{o+ow}{in} \PY{n+nb}{range}\PY{p}{(}\PY{n+nb}{len}\PY{p}{(}\PY{n}{words}\PY{p}{)}\PY{p}{)}\PY{p}{:}
        \PY{n}{ordinate} \PY{o}{=} \PY{l+m+mi}{0}
        \PY{k}{if} \PY{n}{verbose}\PY{p}{:}
            \PY{n+nb}{print}\PY{p}{(}\PY{l+s+s2}{\PYZdq{}}\PY{l+s+s2}{0: i=}\PY{l+s+s2}{\PYZdq{}}\PY{p}{,}\PY{n}{i}\PY{p}{)}
        \PY{k}{for} \PY{n}{abscissa} \PY{o+ow}{in} \PY{n+nb}{range}\PY{p}{(}\PY{n}{i}\PY{o}{+}\PY{l+m+mi}{1}\PY{p}{,}\PY{n+nb}{len}\PY{p}{(}\PY{n}{words}\PY{p}{)}\PY{o}{+}\PY{l+m+mi}{1}\PY{p}{)}\PY{p}{:}
            \PY{n}{matrix\PYZus{}pcky}\PY{p}{[}\PY{p}{(}\PY{n}{ordinate}\PY{p}{,} \PY{n}{abscissa}\PY{p}{)}\PY{p}{]} \PY{o}{=} \PY{n+nb}{list}\PY{p}{(}\PY{p}{)}
            \PY{k}{if} \PY{n}{verbose}\PY{p}{:}
                \PY{n+nb}{print}\PY{p}{(}\PY{l+s+s2}{\PYZdq{}}\PY{l+s+s2}{1: abscissa=}\PY{l+s+s2}{\PYZdq{}}\PY{p}{,}\PY{n}{abscissa}\PY{p}{,}\PY{l+s+s2}{\PYZdq{}}\PY{l+s+s2}{ordinate=}\PY{l+s+s2}{\PYZdq{}}\PY{p}{,}\PY{n}{ordinate}\PY{p}{,}\PY{l+s+s2}{\PYZdq{}}\PY{l+s+s2}{a\PYZhy{}o=}\PY{l+s+s2}{\PYZdq{}}\PY{p}{,}\PY{n}{abscissa}\PY{o}{\PYZhy{}}\PY{n}{ordinate}\PY{p}{)}
            \PY{k}{if}\PY{p}{(}\PY{n}{abscissa}\PY{o}{\PYZhy{}}\PY{n}{ordinate} \PY{o}{==} \PY{l+m+mi}{1}\PY{p}{)}\PY{p}{:}
                \PY{k}{for} \PY{n}{key} \PY{o+ow}{in} \PY{n}{lexicon}\PY{p}{:}
                    \PY{k}{if} \PY{n}{verbose}\PY{p}{:}
                        \PY{n+nb}{print}\PY{p}{(}\PY{l+s+s2}{\PYZdq{}}\PY{l+s+s2}{2: key=}\PY{l+s+s2}{\PYZdq{}}\PY{p}{,}\PY{n}{key}\PY{p}{,}\PY{l+s+s2}{\PYZdq{}}\PY{l+s+s2}{word=}\PY{l+s+s2}{\PYZdq{}}\PY{p}{,}\PY{n}{words}\PY{p}{[}\PY{n}{ordinate}\PY{p}{]}\PY{p}{)}
                    \PY{k}{if}\PY{p}{(}\PY{n}{words}\PY{p}{[}\PY{n}{ordinate}\PY{p}{]} \PY{o+ow}{in} \PY{n}{lexicon}\PY{p}{[}\PY{n}{key}\PY{p}{]}\PY{p}{)}\PY{p}{:}
                        \PY{n}{matrix\PYZus{}pcky}\PY{p}{[}\PY{p}{(}\PY{n}{ordinate}\PY{p}{,}\PY{n}{abscissa}\PY{p}{)}\PY{p}{]}\PY{o}{.}\PY{n}{append}\PY{p}{(}
                            \PY{p}{(}\PY{n}{key}\PY{p}{,}\PY{l+m+mi}{0}\PY{p}{,}\PY{n}{words}\PY{p}{[}\PY{n}{ordinate}\PY{p}{]}\PY{p}{,}\PY{n}{words}\PY{p}{[}\PY{n}{ordinate}\PY{p}{]}\PY{p}{,}
                            \PY{n}{get\PYZus{}prob}\PY{p}{(}\PY{p}{(}\PY{n}{key}\PY{p}{,}\PY{n}{words}\PY{p}{[}\PY{n}{ordinate}\PY{p}{]}\PY{p}{)}\PY{p}{,}\PY{n}{probabilities\PYZus{}table}\PY{p}{)}\PY{p}{)}\PY{p}{)}
                        \PY{k}{if} \PY{n}{verbose}\PY{p}{:}
                            \PY{n+nb}{print}\PY{p}{(}\PY{l+s+s2}{\PYZdq{}}\PY{l+s+s2}{2:}\PY{l+s+s2}{\PYZdq{}}\PY{p}{,}\PY{n}{matrix\PYZus{}pcky}\PY{p}{)}
            \PY{k}{elif}\PY{p}{(}\PY{n}{abscissa}\PY{o}{\PYZhy{}}\PY{n}{ordinate} \PY{o}{\PYZgt{}} \PY{l+m+mi}{1}\PY{p}{)}\PY{p}{:}
                \PY{k}{if} \PY{n}{verbose}\PY{p}{:}
                    \PY{n+nb}{print}\PY{p}{(}\PY{l+s+s2}{\PYZdq{}}\PY{l+s+s2}{3:}\PY{l+s+s2}{\PYZdq{}}\PY{p}{,}\PY{l+s+s2}{\PYZdq{}}\PY{l+s+s2}{(}\PY{l+s+s2}{\PYZdq{}}\PY{p}{,}\PY{n}{ordinate}\PY{p}{,}\PY{l+s+s2}{\PYZdq{}}\PY{l+s+s2}{,}\PY{l+s+s2}{\PYZdq{}}\PY{p}{,}\PY{n}{abscissa}\PY{p}{,}\PY{l+s+s2}{\PYZdq{}}\PY{l+s+s2}{)}\PY{l+s+s2}{\PYZdq{}}\PY{p}{)}
                \PY{k}{for} \PY{n}{index} \PY{o+ow}{in} \PY{n+nb}{range}\PY{p}{(}\PY{n}{abscissa}\PY{o}{\PYZhy{}}\PY{n}{ordinate}\PY{o}{\PYZhy{}}\PY{l+m+mi}{1}\PY{p}{)}\PY{p}{:}
                    \PY{n}{left} \PY{o}{=} \PY{n}{matrix\PYZus{}pcky}\PY{p}{[}\PY{p}{(}\PY{n}{ordinate}\PY{p}{,}\PY{n}{abscissa}\PY{o}{\PYZhy{}}\PY{l+m+mi}{1}\PY{o}{\PYZhy{}}\PY{n}{index}\PY{p}{)}\PY{p}{]}
                    \PY{n}{down} \PY{o}{=} \PY{n}{matrix\PYZus{}pcky}\PY{p}{[}\PY{p}{(}\PY{n}{abscissa}\PY{o}{\PYZhy{}}\PY{l+m+mi}{1}\PY{o}{\PYZhy{}}\PY{n}{index}\PY{p}{,}\PY{n}{abscissa}\PY{p}{)}\PY{p}{]}
                    \PY{k}{if} \PY{n}{verbose}\PY{p}{:}
                        \PY{n+nb}{print}\PY{p}{(}\PY{l+s+s2}{\PYZdq{}}\PY{l+s+s2}{4: index=}\PY{l+s+s2}{\PYZdq{}}\PY{p}{,}\PY{n}{index}\PY{p}{)}
                        \PY{n+nb}{print}\PY{p}{(}\PY{l+s+s2}{\PYZdq{}}\PY{l+s+s2}{4: left=}\PY{l+s+s2}{\PYZdq{}}\PY{p}{,}\PY{n}{left}\PY{p}{,}\PY{l+s+s2}{\PYZdq{}}\PY{l+s+se}{\PYZbs{}n}\PY{l+s+s2}{  }\PY{l+s+s2}{\PYZdq{}}\PY{p}{,}\PY{l+s+s2}{\PYZdq{}}\PY{l+s+s2}{down=}\PY{l+s+s2}{\PYZdq{}}\PY{p}{,}\PY{n}{down}\PY{p}{)}
                    \PY{k}{if} \PY{o+ow}{not} \PY{n}{left} \PY{o+ow}{or} \PY{o+ow}{not} \PY{n}{down}\PY{p}{:}
                        \PY{k}{if} \PY{n}{verbose}\PY{p}{:}
                            \PY{n+nb}{print}\PY{p}{(}\PY{l+s+s2}{\PYZdq{}}\PY{l+s+s2}{5: Nothing to do}\PY{l+s+s2}{\PYZdq{}}\PY{p}{)}
                    \PY{k}{else}\PY{p}{:}
                        \PY{k}{for} \PY{n}{a} \PY{o+ow}{in} \PY{n}{left}\PY{p}{:}
                            \PY{k}{for} \PY{n}{b} \PY{o+ow}{in} \PY{n}{down}\PY{p}{:}
                                \PY{k}{if}\PY{p}{(}\PY{p}{(}\PY{n}{a}\PY{p}{[}\PY{l+m+mi}{0}\PY{p}{]}\PY{p}{,}\PY{n}{b}\PY{p}{[}\PY{l+m+mi}{0}\PY{p}{]}\PY{p}{)} \PY{o+ow}{in} \PY{n}{grammar}\PY{p}{)}\PY{p}{:}
                                    \PY{n}{matrix\PYZus{}pcky}\PY{p}{[}\PY{p}{(}\PY{n}{ordinate}\PY{p}{,}\PY{n}{abscissa}\PY{p}{)}\PY{p}{]}\PY{o}{.}\PY{n}{append}\PY{p}{(}
                                        \PY{p}{(}\PY{n}{grammar}\PY{p}{[}\PY{p}{(}\PY{n}{a}\PY{p}{[}\PY{l+m+mi}{0}\PY{p}{]}\PY{p}{,}\PY{n}{b}\PY{p}{[}\PY{l+m+mi}{0}\PY{p}{]}\PY{p}{)}\PY{p}{]}\PY{p}{[}\PY{l+m+mi}{0}\PY{p}{]}\PY{p}{,}
                                        \PY{n}{abscissa}\PY{o}{\PYZhy{}}\PY{l+m+mi}{1}\PY{o}{\PYZhy{}}\PY{n}{index}\PY{p}{,}\PY{n}{a}\PY{p}{[}\PY{l+m+mi}{0}\PY{p}{]}\PY{p}{,}\PY{n}{b}\PY{p}{[}\PY{l+m+mi}{0}\PY{p}{]}\PY{p}{,}
                                        \PY{n}{update\PYZus{}prob}\PY{p}{(}\PY{n}{a}\PY{p}{,}\PY{n}{b}\PY{p}{,}\PY{n}{grammar}\PY{p}{[}\PY{p}{(}\PY{n}{a}\PY{p}{[}\PY{l+m+mi}{0}\PY{p}{]}\PY{p}{,}\PY{n}{b}\PY{p}{[}\PY{l+m+mi}{0}\PY{p}{]}\PY{p}{)}\PY{p}{]}\PY{p}{[}\PY{l+m+mi}{0}\PY{p}{]}\PY{p}{,}\PY{n}{probabilities\PYZus{}table}\PY{p}{)}\PY{p}{)}
                                        \PY{p}{)}
                                    \PY{k}{if} \PY{n}{verbose}\PY{p}{:}
                                        \PY{n+nb}{print}\PY{p}{(}\PY{l+s+s2}{\PYZdq{}}\PY{l+s+s2}{6: add=}\PY{l+s+s2}{\PYZdq{}}\PY{p}{,}\PY{n}{grammar}\PY{p}{[}\PY{p}{(}\PY{n}{a}\PY{p}{[}\PY{l+m+mi}{0}\PY{p}{]}\PY{p}{,}\PY{n}{b}\PY{p}{[}\PY{l+m+mi}{0}\PY{p}{]}\PY{p}{)}\PY{p}{]}\PY{p}{[}\PY{l+m+mi}{0}\PY{p}{]}\PY{p}{,}\PY{n}{abscissa}\PY{o}{\PYZhy{}}\PY{l+m+mi}{1}\PY{o}{\PYZhy{}}\PY{n}{index}\PY{p}{,}\PY{n}{a}\PY{p}{[}\PY{l+m+mi}{0}\PY{p}{]}\PY{p}{,}\PY{n}{b}\PY{p}{[}\PY{l+m+mi}{0}\PY{p}{]}\PY{p}{)}
                                \PY{k}{else}\PY{p}{:}
                                    \PY{k}{if} \PY{n}{verbose}\PY{p}{:}
                                        \PY{n+nb}{print}\PY{p}{(}\PY{l+s+s2}{\PYZdq{}}\PY{l+s+s2}{6: Nothing to do}\PY{l+s+s2}{\PYZdq{}}\PY{p}{)}
            \PY{k}{if} \PY{n}{verbose}\PY{p}{:}
                \PY{n+nb}{print}\PY{p}{(}\PY{n}{matrix\PYZus{}pcky}\PY{p}{)}
                \PY{n+nb}{print}\PY{p}{(}\PY{l+s+s2}{\PYZdq{}}\PY{l+s+s2}{9: Next Ordinate}\PY{l+s+s2}{\PYZdq{}}\PY{p}{)}
            \PY{n}{ordinate}\PY{o}{+}\PY{o}{=}\PY{l+m+mi}{1}
    \PY{k}{return} \PY{n}{matrix\PYZus{}pcky}
\end{Verbatim}
\end{tcolorbox}

    \hypertarget{obtenciuxf3n-de-resultados}{%
\subsection*{Obtención de resultados}\label{obtenciuxf3n-de-resultados}}

La función \texttt{tree()} recorre una matriz generada por el algoritmo
CKY en busca de árboles a partir del valor de un nodo en particular,
tomando en cuenta una ruta a través de un índice (llamado backpointer en el algoritmo) creado por el
algoritmo.

    \begin{tcolorbox}[breakable, size=fbox, boxrule=1pt, pad at break*=1mm,colback=cellbackground, colframe=cellborder]
\prompt{In}{incolor}{ }{\boxspacing}
\begin{Verbatim}[commandchars=\\\{\}]
\PY{k}{def} \PY{n+nf}{tree} \PY{p}{(}\PY{n}{matrix}\PY{p}{,} \PY{n}{cell}\PY{p}{,} \PY{n}{parent}\PY{o}{=}\PY{l+s+s2}{\PYZdq{}}\PY{l+s+s2}{S}\PY{l+s+s2}{\PYZdq{}}\PY{p}{,} \PY{n}{index}\PY{o}{=}\PY{l+m+mi}{0}\PY{p}{,} \PY{n}{pcky}\PY{o}{=}\PY{k+kc}{False}\PY{p}{)}\PY{p}{:}
    \PY{n+nb}{list} \PY{o}{=} \PY{p}{[}\PY{p}{]}
    \PY{k}{if} \PY{n+nb}{len}\PY{p}{(}\PY{n}{matrix}\PY{p}{[}\PY{n}{cell}\PY{p}{]}\PY{p}{)} \PY{o}{==} \PY{l+m+mi}{0}\PY{p}{:}
        \PY{k}{return} \PY{n+nb}{list}
    \PY{k}{if} \PY{n}{matrix}\PY{p}{[}\PY{n}{cell}\PY{p}{]}\PY{p}{[}\PY{l+m+mi}{0}\PY{p}{]}\PY{p}{[}\PY{l+m+mi}{1}\PY{p}{]} \PY{o}{==} \PY{l+m+mi}{0}\PY{p}{:}
        \PY{k}{if} \PY{n}{matrix}\PY{p}{[}\PY{n}{cell}\PY{p}{]}\PY{p}{[}\PY{n}{index}\PY{p}{]}\PY{p}{[}\PY{l+m+mi}{0}\PY{p}{]} \PY{o}{==} \PY{n}{parent}\PY{p}{:}
            \PY{k}{if} \PY{n}{pcky}\PY{p}{:}
                \PY{k}{return} \PY{p}{[}\PY{p}{(}\PY{n}{matrix}\PY{p}{[}\PY{n}{cell}\PY{p}{]}\PY{p}{[}\PY{n}{index}\PY{p}{]}\PY{p}{[}\PY{l+m+mi}{0}\PY{p}{]}\PY{p}{,}\PY{n}{matrix}\PY{p}{[}\PY{n}{cell}\PY{p}{]}\PY{p}{[}\PY{n}{index}\PY{p}{]}\PY{p}{[}\PY{l+m+mi}{4}\PY{p}{]}\PY{p}{)}\PY{p}{,}
                         \PY{n}{matrix}\PY{p}{[}\PY{n}{cell}\PY{p}{]}\PY{p}{[}\PY{n}{index}\PY{p}{]}\PY{p}{[}\PY{l+m+mi}{2}\PY{p}{]}\PY{p}{]}
            \PY{k}{else}\PY{p}{:}
                \PY{k}{return} \PY{p}{[}\PY{n}{matrix}\PY{p}{[}\PY{n}{cell}\PY{p}{]}\PY{p}{[}\PY{n}{index}\PY{p}{]}\PY{p}{[}\PY{l+m+mi}{0}\PY{p}{]}\PY{p}{,} \PY{n}{matrix}\PY{p}{[}\PY{n}{cell}\PY{p}{]}\PY{p}{[}\PY{n}{index}\PY{p}{]}\PY{p}{[}\PY{l+m+mi}{2}\PY{p}{]}\PY{p}{]}
        \PY{k}{else}\PY{p}{:}
            \PY{k}{if} \PY{n}{index}\PY{o}{+}\PY{l+m+mi}{1} \PY{o}{\PYZlt{}} \PY{n+nb}{len}\PY{p}{(}\PY{n}{matrix}\PY{p}{[}\PY{n}{cell}\PY{p}{]}\PY{p}{)}\PY{p}{:}
                \PY{k}{if} \PY{n}{pcky}\PY{p}{:}
                    \PY{k}{return} \PY{n}{tree}\PY{p}{(}\PY{n}{matrix}\PY{p}{,}\PY{n}{cell}\PY{p}{,}\PY{n}{parent}\PY{p}{,}\PY{n}{index}\PY{o}{+}\PY{l+m+mi}{1}\PY{p}{,}\PY{n}{pcky}\PY{o}{=}\PY{k+kc}{True}\PY{p}{)}
                \PY{k}{else}\PY{p}{:}
                    \PY{k}{return} \PY{n}{tree}\PY{p}{(}\PY{n}{matrix}\PY{p}{,}\PY{n}{cell}\PY{p}{,}\PY{n}{parent}\PY{p}{,}\PY{n}{index}\PY{o}{+}\PY{l+m+mi}{1}\PY{p}{)}
            \PY{k}{else}\PY{p}{:}
                \PY{k}{return} \PY{n+nb}{list}
    \PY{k}{elif} \PY{p}{(}\PY{n}{matrix}\PY{p}{[}\PY{n}{cell}\PY{p}{]}\PY{p}{[}\PY{n}{index}\PY{p}{]}\PY{p}{[}\PY{l+m+mi}{0}\PY{p}{]}\PY{p}{)} \PY{o}{==} \PY{n}{parent}\PY{p}{:}
        \PY{k}{if} \PY{n}{pcky}\PY{p}{:}
            \PY{n+nb}{list}\PY{o}{.}\PY{n}{append}\PY{p}{(}\PY{p}{(}\PY{n}{matrix}\PY{p}{[}\PY{n}{cell}\PY{p}{]}\PY{p}{[}\PY{n}{index}\PY{p}{]}\PY{p}{[}\PY{l+m+mi}{0}\PY{p}{]}\PY{p}{,}\PY{n}{matrix}\PY{p}{[}\PY{n}{cell}\PY{p}{]}\PY{p}{[}\PY{n}{index}\PY{p}{]}\PY{p}{[}\PY{l+m+mi}{4}\PY{p}{]}\PY{p}{)}\PY{p}{)}
        \PY{k}{else}\PY{p}{:}
            \PY{n+nb}{list}\PY{o}{.}\PY{n}{append}\PY{p}{(}\PY{n}{matrix}\PY{p}{[}\PY{n}{cell}\PY{p}{]}\PY{p}{[}\PY{n}{index}\PY{p}{]}\PY{p}{[}\PY{l+m+mi}{0}\PY{p}{]}\PY{p}{)}
        \PY{n}{child} \PY{o}{=} \PY{p}{[}\PY{p}{]}
        \PY{k}{if} \PY{n}{pcky}\PY{p}{:}
            \PY{n}{child}\PY{o}{.}\PY{n}{append}\PY{p}{(}
                \PY{n}{tree}\PY{p}{(}\PY{n}{matrix}\PY{p}{,}
                     \PY{p}{(}\PY{n}{cell}\PY{p}{[}\PY{l+m+mi}{0}\PY{p}{]}\PY{p}{,}\PY{n}{matrix}\PY{p}{[}\PY{n}{cell}\PY{p}{]}\PY{p}{[}\PY{n}{index}\PY{p}{]}\PY{p}{[}\PY{l+m+mi}{1}\PY{p}{]}\PY{p}{)}\PY{p}{,}
                     \PY{n}{matrix}\PY{p}{[}\PY{n}{cell}\PY{p}{]}\PY{p}{[}\PY{n}{index}\PY{p}{]}\PY{p}{[}\PY{l+m+mi}{2}\PY{p}{]}\PY{p}{,}\PY{n}{pcky}\PY{o}{=}\PY{k+kc}{True}\PY{p}{)}\PY{p}{)}
        \PY{k}{else}\PY{p}{:}
            \PY{n}{child}\PY{o}{.}\PY{n}{append}\PY{p}{(}
                \PY{n}{tree}\PY{p}{(}\PY{n}{matrix}\PY{p}{,}
                     \PY{p}{(}\PY{n}{cell}\PY{p}{[}\PY{l+m+mi}{0}\PY{p}{]}\PY{p}{,}\PY{n}{matrix}\PY{p}{[}\PY{n}{cell}\PY{p}{]}\PY{p}{[}\PY{n}{index}\PY{p}{]}\PY{p}{[}\PY{l+m+mi}{1}\PY{p}{]}\PY{p}{)}\PY{p}{,}
                     \PY{n}{matrix}\PY{p}{[}\PY{n}{cell}\PY{p}{]}\PY{p}{[}\PY{n}{index}\PY{p}{]}\PY{p}{[}\PY{l+m+mi}{2}\PY{p}{]}\PY{p}{)}\PY{p}{)}
        \PY{k}{if} \PY{n}{pcky}\PY{p}{:}
            \PY{n}{child}\PY{o}{.}\PY{n}{append}\PY{p}{(}
                \PY{n}{tree}\PY{p}{(}\PY{n}{matrix}\PY{p}{,}
                     \PY{p}{(}\PY{n}{matrix}\PY{p}{[}\PY{n}{cell}\PY{p}{]}\PY{p}{[}\PY{n}{index}\PY{p}{]}\PY{p}{[}\PY{l+m+mi}{1}\PY{p}{]}\PY{p}{,}\PY{n}{cell}\PY{p}{[}\PY{l+m+mi}{1}\PY{p}{]}\PY{p}{)}\PY{p}{,}
                     \PY{n}{matrix}\PY{p}{[}\PY{n}{cell}\PY{p}{]}\PY{p}{[}\PY{n}{index}\PY{p}{]}\PY{p}{[}\PY{l+m+mi}{3}\PY{p}{]}\PY{p}{,}\PY{n}{pcky}\PY{o}{=}\PY{k+kc}{True}\PY{p}{)}\PY{p}{)}
        \PY{k}{else}\PY{p}{:}
            \PY{n}{child}\PY{o}{.}\PY{n}{append}\PY{p}{(}
                \PY{n}{tree}\PY{p}{(}\PY{n}{matrix}\PY{p}{,}
                     \PY{p}{(}\PY{n}{matrix}\PY{p}{[}\PY{n}{cell}\PY{p}{]}\PY{p}{[}\PY{n}{index}\PY{p}{]}\PY{p}{[}\PY{l+m+mi}{1}\PY{p}{]}\PY{p}{,}\PY{n}{cell}\PY{p}{[}\PY{l+m+mi}{1}\PY{p}{]}\PY{p}{)}\PY{p}{,}
                     \PY{n}{matrix}\PY{p}{[}\PY{n}{cell}\PY{p}{]}\PY{p}{[}\PY{n}{index}\PY{p}{]}\PY{p}{[}\PY{l+m+mi}{3}\PY{p}{]}\PY{p}{)}\PY{p}{)}
        \PY{n+nb}{list}\PY{o}{.}\PY{n}{append}\PY{p}{(}\PY{n}{child}\PY{p}{)}
        \PY{k}{return} \PY{n+nb}{list}
    \PY{k}{else}\PY{p}{:}
        \PY{k}{if} \PY{n}{index}\PY{o}{+}\PY{l+m+mi}{1} \PY{o}{\PYZlt{}} \PY{n+nb}{len}\PY{p}{(}\PY{n}{matrix}\PY{p}{[}\PY{n}{cell}\PY{p}{]}\PY{p}{)}\PY{p}{:}
            \PY{k}{if} \PY{n}{pcky}\PY{p}{:}
                \PY{k}{return} \PY{n}{tree}\PY{p}{(}\PY{n}{matrix}\PY{p}{,}\PY{n}{cell}\PY{p}{,}\PY{n}{parent}\PY{p}{,}\PY{n}{index}\PY{o}{+}\PY{l+m+mi}{1}\PY{p}{,}\PY{n}{pcky}\PY{o}{=}\PY{k+kc}{True}\PY{p}{)}
            \PY{k}{else}\PY{p}{:}
                \PY{k}{return} \PY{n}{tree}\PY{p}{(}\PY{n}{matrix}\PY{p}{,}\PY{n}{cell}\PY{p}{,}\PY{n}{parent}\PY{p}{,}\PY{n}{index}\PY{o}{+}\PY{l+m+mi}{1}\PY{p}{)}
        \PY{k}{else}\PY{p}{:}
            \PY{k}{return} \PY{n+nb}{list}
\end{Verbatim}
\end{tcolorbox}

    La función \texttt{find\_solutions()} servirá como \emph{frontend} para
todas las funciones anteriores, brindará la respuesta propuesta por el
algoritmo y la matriz correspondiente. Dependerá de los argumentos si usará CKY o PCKY (ver notebook adjunto).

%%     \begin{tcolorbox}[breakable, size=fbox, boxrule=1pt, pad at break*=1mm,colback=cellbackground, colframe=cellborder]
%% \prompt{In}{incolor}{ }{\boxspacing}
%% \begin{Verbatim}[commandchars=\\\{\}]
%% \PY{k}{def} \PY{n+nf}{find\PYZus{}solutions}\PY{p}{(}\PY{n}{sentence}\PY{p}{,} \PY{n}{grammar}\PY{p}{,} \PY{n}{lexicon}\PY{p}{,} \PY{n}{axiom}\PY{p}{,} \PY{n}{probabilities\PYZus{}table}\PY{o}{=}\PY{p}{\PYZob{}}\PY{p}{\PYZcb{}}\PY{p}{,} \PY{n}{verbose}\PY{o}{=}\PY{k+kc}{False}\PY{p}{)}\PY{p}{:}
%%     \PY{n}{N} \PY{o}{=} \PY{n+nb}{len}\PY{p}{(}\PY{n}{sentence}\PY{o}{.}\PY{n}{split}\PY{p}{(}\PY{p}{)}\PY{p}{)}
%%     \PY{n}{solutions} \PY{o}{=} \PY{n+nb}{list}\PY{p}{(}\PY{p}{)}
%%     \PY{k}{if} \PY{n}{probabilities\PYZus{}table}\PY{p}{:}
%%         \PY{n}{matrix} \PY{o}{=} \PY{n}{pcky\PYZus{}parser}\PY{p}{(}\PY{n}{sentence}\PY{p}{,}\PY{n}{grammar}\PY{p}{,}\PY{n}{lexicon}\PY{p}{,}\PY{n}{probabilities\PYZus{}table}\PY{p}{,}\PY{n}{verbose}\PY{p}{)}
%%         \PY{n}{candidates} \PY{o}{=} \PY{n+nb}{list}\PY{p}{(}\PY{p}{)}
%%         \PY{k}{for} \PY{n}{i} \PY{o+ow}{in} \PY{n+nb}{enumerate}\PY{p}{(}\PY{n}{matrix}\PY{p}{[}\PY{p}{(}\PY{l+m+mi}{0}\PY{p}{,}\PY{n}{N}\PY{p}{)}\PY{p}{]}\PY{p}{)}\PY{p}{:}
%%             \PY{k}{if} \PY{n}{i}\PY{p}{[}\PY{l+m+mi}{1}\PY{p}{]}\PY{p}{[}\PY{l+m+mi}{0}\PY{p}{]} \PY{o}{==} \PY{n}{axiom}\PY{p}{:}
%%                 \PY{n}{candidates}\PY{o}{.}\PY{n}{append}\PY{p}{(}\PY{p}{[}\PY{n}{i}\PY{p}{[}\PY{l+m+mi}{0}\PY{p}{]}\PY{p}{,}\PY{n}{i}\PY{p}{[}\PY{l+m+mi}{1}\PY{p}{]}\PY{p}{[}\PY{l+m+mi}{4}\PY{p}{]}\PY{p}{]}\PY{p}{)}
%%         \PY{n}{candidates} \PY{o}{=} \PY{n+nb}{max}\PY{p}{(}\PY{n}{candidates}\PY{p}{,}\PY{n}{key}\PY{o}{=}\PY{n}{itemgetter}\PY{p}{(}\PY{l+m+mi}{1}\PY{p}{)}\PY{p}{)}
%%         \PY{n}{solutions} \PY{o}{=} \PY{n}{tree}\PY{p}{(}\PY{n}{matrix}\PY{p}{,}\PY{p}{(}\PY{l+m+mi}{0}\PY{p}{,}\PY{n}{N}\PY{p}{)}\PY{p}{,}\PY{n}{index}\PY{o}{=}\PY{n}{candidates}\PY{p}{[}\PY{l+m+mi}{0}\PY{p}{]}\PY{p}{,}\PY{n}{pcky}\PY{o}{=}\PY{k+kc}{True}\PY{p}{)}
%%     \PY{k}{else}\PY{p}{:}
%%         \PY{n}{matrix} \PY{o}{=} \PY{n}{cky\PYZus{}parser}\PY{p}{(}\PY{n}{sentence}\PY{p}{,}\PY{n}{grammar}\PY{p}{,}\PY{n}{lexicon}\PY{p}{,}\PY{n}{verbose}\PY{p}{)}
%%         \PY{k}{for} \PY{n}{i} \PY{o+ow}{in} \PY{n+nb}{enumerate}\PY{p}{(}\PY{n}{matrix}\PY{p}{[}\PY{p}{(}\PY{l+m+mi}{0}\PY{p}{,}\PY{n}{N}\PY{p}{)}\PY{p}{]}\PY{p}{)}\PY{p}{:}
%%             \PY{k}{if} \PY{n}{i}\PY{p}{[}\PY{l+m+mi}{1}\PY{p}{]}\PY{p}{[}\PY{l+m+mi}{0}\PY{p}{]} \PY{o}{==} \PY{n}{axiom}\PY{p}{:}
%%                 \PY{n}{solutions}\PY{o}{.}\PY{n}{append}\PY{p}{(}\PY{n}{tree}\PY{p}{(}\PY{n}{matrix}\PY{p}{,}\PY{p}{(}\PY{l+m+mi}{0}\PY{p}{,}\PY{n}{N}\PY{p}{)}\PY{p}{,}\PY{n}{index}\PY{o}{=}\PY{n}{i}\PY{p}{[}\PY{l+m+mi}{0}\PY{p}{]}\PY{p}{)}\PY{p}{)}
%%     \PY{n}{df} \PY{o}{=} \PY{n}{pd}\PY{o}{.}\PY{n}{DataFrame}\PY{p}{(}\PY{n}{columns} \PY{o}{=} \PY{p}{[}\PY{n}{i} \PY{k}{for} \PY{n}{i} \PY{o+ow}{in} \PY{n+nb}{range}\PY{p}{(}\PY{l+m+mi}{0}\PY{p}{,}\PY{n}{N}\PY{p}{)}\PY{p}{]}\PY{p}{,}
%%                         \PY{n}{index} \PY{o}{=} \PY{p}{[}\PY{n}{i} \PY{k}{for} \PY{n}{i} \PY{o+ow}{in} \PY{n+nb}{range}\PY{p}{(}\PY{l+m+mi}{1}\PY{p}{,}\PY{n}{N}\PY{o}{+}\PY{l+m+mi}{1}\PY{p}{)}\PY{p}{]}\PY{p}{)}
%%     \PY{k}{for} \PY{n}{i} \PY{o+ow}{in} \PY{n+nb}{range}\PY{p}{(}\PY{l+m+mi}{0}\PY{p}{,}\PY{l+m+mi}{5}\PY{p}{)}\PY{p}{:}
%%         \PY{k}{for} \PY{n}{j} \PY{o+ow}{in} \PY{n+nb}{range}\PY{p}{(}\PY{l+m+mi}{0}\PY{p}{,}\PY{l+m+mi}{5}\PY{o}{\PYZhy{}}\PY{n}{i}\PY{p}{)}\PY{p}{:}
%%             \PY{n}{df}\PY{o}{.}\PY{n}{loc}\PY{p}{[}\PY{n}{j}\PY{o}{+}\PY{n}{i}\PY{o}{+}\PY{l+m+mi}{1}\PY{p}{,}\PY{n}{i}\PY{p}{]} \PY{o}{=} \PY{n}{matrix}\PY{p}{[}\PY{p}{(}\PY{n}{i}\PY{p}{,}\PY{n}{j}\PY{o}{+}\PY{n}{i}\PY{o}{+}\PY{l+m+mi}{1}\PY{p}{)}\PY{p}{]}
%%     \PY{n}{df} \PY{o}{=} \PY{n}{df}\PY{o}{.}\PY{n}{transpose}\PY{p}{(}\PY{p}{)}
%%     \PY{k}{return} \PY{n}{solutions}\PY{p}{,} \PY{n}{df}
%% \end{Verbatim}
%% \end{tcolorbox}

    \begin{tcolorbox}[breakable, size=fbox, boxrule=1pt, pad at break*=1mm,colback=cellbackground, colframe=cellborder]
\prompt{In}{incolor}{ }{\boxspacing}
\begin{Verbatim}[commandchars=\\\{\}]
\PY{n}{grammar\PYZus{}rules}\PY{p}{,} \PY{n}{lexicon\PYZus{}rules}\PY{p}{,} \PY{n}{probabilities} \PY{o}{=} \PY{n}{read\PYZus{}rules}\PY{p}{(}\PY{n}{grammar\PYZus{}file}\PY{p}{)}
\end{Verbatim}
\end{tcolorbox}

    \begin{tcolorbox}[breakable, size=fbox, boxrule=1pt, pad at break*=1mm,colback=cellbackground, colframe=cellborder]
\prompt{In}{incolor}{ }{\boxspacing}
\begin{Verbatim}[commandchars=\\\{\}]
\PY{n}{options}\PY{p}{,} \PY{n}{matrix} \PY{o}{=} \PY{n}{find\PYZus{}solutions}\PY{p}{(}\PY{l+s+s2}{\PYZdq{}}\PY{l+s+s2}{Time flies like an arrow}\PY{l+s+s2}{\PYZdq{}}\PY{p}{,}
                        \PY{n}{grammar\PYZus{}rules}\PY{p}{,}\PY{n}{lexicon\PYZus{}rules}\PY{p}{,}\PY{l+s+s2}{\PYZdq{}}\PY{l+s+s2}{S}\PY{l+s+s2}{\PYZdq{}}\PY{p}{)}
\PY{n}{pprint}\PY{p}{(}\PY{n}{options}\PY{p}{)}
\end{Verbatim}
\end{tcolorbox}

    \begin{Verbatim}[commandchars=\\\{\}]
[['S',
  [['NP', [['Nominal', 'time'], ['Nominal', 'flies']]],
   ['VP', [['Verb', 'like'], ['NP', [['Det', 'an'], ['Nominal', 'arrow']]]]]]],
 ['S',
  [['NP', 'time'],
   ['VP',
    [['Verb', 'flies'],
     ['PP',
      [['Preposition', 'like'],
       ['NP', [['Det', 'an'], ['Nominal', 'arrow']]]]]]]]]]
    \end{Verbatim}

%%             \begin{tcolorbox}[breakable, size=fbox, boxrule=.5pt, pad at break*=1mm, opacityfill=0]
%% \prompt{Out}{outcolor}{ }{\boxspacing}
%% \begin{Verbatim}[commandchars=\\\{\}]
%%                                                    1  \textbackslash{}
%% 0  [(NP, 0, time, time), (Nominal, 0, time, time){\ldots}
%% 1                                                NaN
%% 2                                                NaN
%% 3                                                NaN
%% 4                                                NaN

%%                                                    2  \textbackslash{}
%% 0  [(S, 1, NP, VP), (NP, 1, Nominal, Nominal), (N{\ldots}
%% 1  [(NP, 0, flies, flies), (Nominal, 0, flies, fl{\ldots}
%% 2                                                NaN
%% 3                                                NaN
%% 4                                                NaN

%%                                                    3                   4  \textbackslash{}
%% 0                                   [(S, 2, NP, VP)]                  []
%% 1                                   [(S, 2, NP, VP)]                  []
%% 2  [(VP, 0, like, like), (Verb, 0, like, like), ({\ldots}                  []
%% 3                                                NaN  [(Det, 0, an, an)]
%% 4                                                NaN                 NaN

%%                                                    5
%% 0  [(S, 2, NP, VP), (Nominal, 2, Nominal, PP), (S{\ldots}
%% 1  [(S, 2, NP, VP), (Nominal, 2, Nominal, PP), (V{\ldots}
%% 2      [(VP, 3, Verb, NP), (PP, 3, Preposition, NP)]
%% 3                            [(NP, 4, Det, Nominal)]
%% 4  [(NP, 0, arrow, arrow), (Nominal, 0, arrow, ar{\ldots}
%% \end{Verbatim}
%% \end{tcolorbox}
        
    \begin{tcolorbox}[breakable, size=fbox, boxrule=1pt, pad at break*=1mm,colback=cellbackground, colframe=cellborder]
\prompt{In}{incolor}{ }{\boxspacing}
\begin{Verbatim}[commandchars=\\\{\}]
\PY{n}{options}\PY{p}{,} \PY{n}{matrix} \PY{o}{=} \PY{n}{find\PYZus{}solutions}\PY{p}{(}\PY{l+s+s2}{\PYZdq{}}\PY{l+s+s2}{Time flies like an arrow}\PY{l+s+s2}{\PYZdq{}}\PY{p}{,}
                        \PY{n}{grammar\PYZus{}rules}\PY{p}{,}\PY{n}{lexicon\PYZus{}rules}\PY{p}{,}\PY{l+s+s2}{\PYZdq{}}\PY{l+s+s2}{S}\PY{l+s+s2}{\PYZdq{}}\PY{p}{,}\PY{n}{probabilities}\PY{p}{)}
\PY{n}{pprint}\PY{p}{(}\PY{n}{options}\PY{p}{)}
\end{Verbatim}
\end{tcolorbox}

    \begin{Verbatim}[commandchars=\\\{\}]
[('S', 9.600000000000002e-13),
 [[('NP', 0.002), 'time'],
  [('VP', 6.000000000000001e-10),
   [[('Verb', 0.02), 'flies'],
    [('PP', 1.5000000000000002e-07),
     [[('Preposition', 0.05), 'like'],
      [('NP', 3e-05),
       [[('Det', 0.05), 'an'], [('Nominal', 0.002), 'arrow']]]]]]]]]
    \end{Verbatim}

%%             \begin{tcolorbox}[breakable, size=fbox, boxrule=.5pt, pad at break*=1mm, opacityfill=0]
%% \prompt{Out}{outcolor}{ }{\boxspacing}
%% \begin{Verbatim}[commandchars=\\\{\}]
%%                                                    1  \textbackslash{}
%% 0  [(NP, 0, time, time, 0.002), (Nominal, 0, time{\ldots}
%% 1                                                NaN
%% 2                                                NaN
%% 3                                                NaN
%% 4                                                NaN

%%                                                    2  \textbackslash{}
%% 0  [(S, 1, NP, VP, 1.28e-05), (NP, 1, Nominal, No{\ldots}
%% 1  [(NP, 0, flies, flies, 0.002), (Nominal, 0, fl{\ldots}
%% 2                                                NaN
%% 3                                                NaN
%% 4                                                NaN

%%                                                    3  \textbackslash{}
%% 0           [(S, 2, NP, VP, 5.1200000000000005e-09)]
%% 1                         [(S, 2, NP, VP, 1.28e-05)]
%% 2  [(VP, 0, like, like, 0.008), (Verb, 0, like, l{\ldots}
%% 3                                                NaN
%% 4                                                NaN

%%                           4                                                  5
%% 0                        []  [(S, 2, NP, VP, 1.1520000000000003e-13), (Nomi{\ldots}
%% 1                        []  [(S, 2, NP, VP, 2.8800000000000004e-10), (Nomi{\ldots}
%% 2                        []  [(VP, 3, Verb, NP, 1.8000000000000002e-07), (P{\ldots}
%% 3  [(Det, 0, an, an, 0.05)]                     [(NP, 4, Det, Nominal, 3e-05)]
%% 4                       NaN  [(NP, 0, arrow, arrow, 0.002), (Nominal, 0, ar{\ldots}
%% \end{Verbatim}
%% \end{tcolorbox}
        
    \hypertarget{cuestionario}{%
\subsection*{Cuestionario}\label{cuestionario}}

\begin{enumerate}
\def\labelenumi{\arabic{enumi}.}
\tightlist
\item
  ¿Es correcto el análisis sintáctico que se ha obtenido? Justifica la
  respuesta.
\end{enumerate}

Sí. Basado en la gramática proporcionada el resultado es satisfactorio y
se muestra a continuación. Es el árbol más probable según el cálculo manual, los nodos respetan las reglas de la gramática proporcionada y también es adecuado según la gramática generativa tradicional.

\begin{figure}[!ht]
    \centering
        \Tree [.S~(9.600000000000002e-13)
                [.NP~(0.002) time ]
                [.VP 
                    [.Verb~(0.02) flies ]
                    [.PP~(1.5000000000000002e-07) 
                        [.Preposition~(0.05) like ]
                        [.NP~(3e-05) 
                            [.Det~(0.05) an ]
                            [.Nominal~(0.002) arrow ] ] ] ] ]
\caption{Árbol sintáctico proporcionado por el algoritmo PCKY}
\end{figure}

\begin{enumerate}
\def\labelenumi{\arabic{enumi}.}
\setcounter{enumi}{1}
\tightlist
\item
  ¿Cuáles son las limitaciones de aplicar el algoritmo CKY
  probabilístico para realizar el análisis sintáctico? Justifica la
  respuesta.
\end{enumerate}

Los problemas se presentan debido a la información guardada en la
gramática, si ésta no tiene un elemento particular necesario para el
análisis, proporcionará resultados truncos.

Por supuesto, las probabilidades de aparición de la gramática podrían
llevar a resultados dudosos, incluso cuando su objetivo es el contrario,
pero estas probabilidades siempre dependerán del corpus de origen, que
puede ser distinto al de la oración a analizar, ya sea en registro, en
dialecto, etc.

En el caso particular de este ejercicio, el hecho de tener algunos
símbolos intermedios creados para llegar a la forma normal de Chomsky
permite que el algoritmo genere posibles soluciones impensables para el
análisis linguístico. El algoritmo CKY propone dos soluciones y el PCKY
seleccionó el correcto, pero si se examina la opción desechada se puede
observar:

\begin{verbatim}
['NP', [['Nominal', 'time'], ['Nominal', 'flies']]]
\end{verbatim}

Lo cual es un elemento impensable pues sugiere que una frase nomimal
puede tener dos núcleos. Este análisis va en contra del principio de
endocentrismo que establece, como su nombre indica, que cada frase tiene
un núcleo que determina el tipo de la frase.

Finalmente, el algoritmo que usa este tipo de gramáticas, arrastra la
misma gran crítica que los lingǘistas hacen a la teoría de Chomsky: No
es capaz de analizar lenguas aglutinantes.

\begin{enumerate}
\def\labelenumi{\arabic{enumi}.}
\setcounter{enumi}{2}
\tightlist
\item
  ¿Qué posibles mejoras que se podrían aplicar para mejorar el
  rendimiento del análisis sintáctico? Justifica la respuesta.
\end{enumerate}

El algoritmo se basa en el uso de cualquier gramática en forma normal de
Chomsky, esto es porque computacionalmente es posible generar
automáticamente gramáticas mediante corpus de análisis, sin embargo, el
programa minimalista de Chomsky sugiere la teoría X-barra que tiene su
principio en los estudios de Ray Jackendoff. Esta teoría proporciona la
posibilidad de crear gramáticas consistentes siempre en la forma normal
de Chomsky. Dejar de ignorar el conocimiento linguístico proporcionaría
mejores gramáticas para la IA.

Por otra parte, analizar solo las probabilidades de los subárboles que
llevan a una propuesta de solución aumentaría el rendimiento a nivel de
cómputo.

    % Add a bibliography block to the postdoc
    
    
    
\end{document}
