\documentclass[a4paper,table,11pt]{article}

    \usepackage[breakable]{tcolorbox}
    \usepackage{parskip} % Stop auto-indenting (to mimic markdown behaviour)
    
    \usepackage{iftex}
    \ifPDFTeX
    	\usepackage[T1]{fontenc}
    	\usepackage{mathpazo}
    \else
    	\usepackage{fontspec}
    \fi

    % Basic figure setup, for now with no caption control since it's done
    % automatically by Pandoc (which extracts ![](path) syntax from Markdown).
    \usepackage{graphicx}
    % Maintain compatibility with old templates. Remove in nbconvert 6.0
    \let\Oldincludegraphics\includegraphics
    % Ensure that by default, figures have no caption (until we provide a
    % proper Figure object with a Caption API and a way to capture that
    % in the conversion process - todo).
    \usepackage{caption}
    \DeclareCaptionFormat{nocaption}{}
    \captionsetup{format=nocaption,aboveskip=0pt,belowskip=0pt}

    \usepackage{float}
    \floatplacement{figure}{H} % forces figures to be placed at the correct location
    \usepackage{xcolor} % Allow colors to be defined
    \usepackage{enumerate} % Needed for markdown enumerations to work
    \usepackage{geometry} % Used to adjust the document margins
    \usepackage{amsmath} % Equations
    \usepackage{amssymb} % Equations
    \usepackage{textcomp} % defines textquotesingle
    % Hack from http://tex.stackexchange.com/a/47451/13684:
    \AtBeginDocument{%
        \def\PYZsq{\textquotesingle}% Upright quotes in Pygmentized code
    }
    \usepackage{upquote} % Upright quotes for verbatim code
    \usepackage{eurosym} % defines \euro
    \usepackage[mathletters]{ucs} % Extended unicode (utf-8) support
    \usepackage{fancyvrb} % verbatim replacement that allows latex
    \usepackage{grffile} % extends the file name processing of package graphics 
                         % to support a larger range
    \makeatletter % fix for old versions of grffile with XeLaTeX
    \@ifpackagelater{grffile}{2019/11/01}
    {
      % Do nothing on new versions
    }
    {
      \def\Gread@@xetex#1{%
        \IfFileExists{"\Gin@base".bb}%
        {\Gread@eps{\Gin@base.bb}}%
        {\Gread@@xetex@aux#1}%
      }
    }
    \makeatother
    \usepackage[Export]{adjustbox} % Used to constrain images to a maximum size
    \adjustboxset{max size={0.9\linewidth}{0.9\paperheight}}

    % The hyperref package gives us a pdf with properly built
    % internal navigation ('pdf bookmarks' for the table of contents,
    % internal cross-reference links, web links for URLs, etc.)
    \usepackage{hyperref}
    % The default LaTeX title has an obnoxious amount of whitespace. By default,
    % titling removes some of it. It also provides customization options.
    \usepackage{titling}
    \usepackage{longtable} % longtable support required by pandoc >1.10
    \usepackage{booktabs}  % table support for pandoc > 1.12.2
    \usepackage[inline]{enumitem} % IRkernel/repr support (it uses the enumerate* environment)
    \usepackage[normalem]{ulem} % ulem is needed to support strikethroughs (\sout)
                                % normalem makes italics be italics, not underlines
    \usepackage{mathrsfs}
    

    
    % Colors for the hyperref package
    \definecolor{urlcolor}{rgb}{0,.145,.698}
    \definecolor{linkcolor}{rgb}{.71,0.21,0.01}
    \definecolor{citecolor}{rgb}{.12,.54,.11}

    % ANSI colors
    \definecolor{ansi-black}{HTML}{3E424D}
    \definecolor{ansi-black-intense}{HTML}{282C36}
    \definecolor{ansi-red}{HTML}{E75C58}
    \definecolor{ansi-red-intense}{HTML}{B22B31}
    \definecolor{ansi-green}{HTML}{00A250}
    \definecolor{ansi-green-intense}{HTML}{007427}
    \definecolor{ansi-yellow}{HTML}{DDB62B}
    \definecolor{ansi-yellow-intense}{HTML}{B27D12}
    \definecolor{ansi-blue}{HTML}{208FFB}
    \definecolor{ansi-blue-intense}{HTML}{0065CA}
    \definecolor{ansi-magenta}{HTML}{D160C4}
    \definecolor{ansi-magenta-intense}{HTML}{A03196}
    \definecolor{ansi-cyan}{HTML}{60C6C8}
    \definecolor{ansi-cyan-intense}{HTML}{258F8F}
    \definecolor{ansi-white}{HTML}{C5C1B4}
    \definecolor{ansi-white-intense}{HTML}{A1A6B2}
    \definecolor{ansi-default-inverse-fg}{HTML}{FFFFFF}
    \definecolor{ansi-default-inverse-bg}{HTML}{000000}

    % common color for the border for error outputs.
    \definecolor{outerrorbackground}{HTML}{FFDFDF}

    % commands and environments needed by pandoc snippets
    % extracted from the output of `pandoc -s`
    \providecommand{\tightlist}{%
      \setlength{\itemsep}{0pt}\setlength{\parskip}{0pt}}
    \DefineVerbatimEnvironment{Highlighting}{Verbatim}{commandchars=\\\{\}}
    % Add ',fontsize=\small' for more characters per line
    \newenvironment{Shaded}{}{}
    \newcommand{\KeywordTok}[1]{\textcolor[rgb]{0.00,0.44,0.13}{\textbf{{#1}}}}
    \newcommand{\DataTypeTok}[1]{\textcolor[rgb]{0.56,0.13,0.00}{{#1}}}
    \newcommand{\DecValTok}[1]{\textcolor[rgb]{0.25,0.63,0.44}{{#1}}}
    \newcommand{\BaseNTok}[1]{\textcolor[rgb]{0.25,0.63,0.44}{{#1}}}
    \newcommand{\FloatTok}[1]{\textcolor[rgb]{0.25,0.63,0.44}{{#1}}}
    \newcommand{\CharTok}[1]{\textcolor[rgb]{0.25,0.44,0.63}{{#1}}}
    \newcommand{\StringTok}[1]{\textcolor[rgb]{0.25,0.44,0.63}{{#1}}}
    \newcommand{\CommentTok}[1]{\textcolor[rgb]{0.38,0.63,0.69}{\textit{{#1}}}}
    \newcommand{\OtherTok}[1]{\textcolor[rgb]{0.00,0.44,0.13}{{#1}}}
    \newcommand{\AlertTok}[1]{\textcolor[rgb]{1.00,0.00,0.00}{\textbf{{#1}}}}
    \newcommand{\FunctionTok}[1]{\textcolor[rgb]{0.02,0.16,0.49}{{#1}}}
    \newcommand{\RegionMarkerTok}[1]{{#1}}
    \newcommand{\ErrorTok}[1]{\textcolor[rgb]{1.00,0.00,0.00}{\textbf{{#1}}}}
    \newcommand{\NormalTok}[1]{{#1}}
    
    % Additional commands for more recent versions of Pandoc
    \newcommand{\ConstantTok}[1]{\textcolor[rgb]{0.53,0.00,0.00}{{#1}}}
    \newcommand{\SpecialCharTok}[1]{\textcolor[rgb]{0.25,0.44,0.63}{{#1}}}
    \newcommand{\VerbatimStringTok}[1]{\textcolor[rgb]{0.25,0.44,0.63}{{#1}}}
    \newcommand{\SpecialStringTok}[1]{\textcolor[rgb]{0.73,0.40,0.53}{{#1}}}
    \newcommand{\ImportTok}[1]{{#1}}
    \newcommand{\DocumentationTok}[1]{\textcolor[rgb]{0.73,0.13,0.13}{\textit{{#1}}}}
    \newcommand{\AnnotationTok}[1]{\textcolor[rgb]{0.38,0.63,0.69}{\textbf{\textit{{#1}}}}}
    \newcommand{\CommentVarTok}[1]{\textcolor[rgb]{0.38,0.63,0.69}{\textbf{\textit{{#1}}}}}
    \newcommand{\VariableTok}[1]{\textcolor[rgb]{0.10,0.09,0.49}{{#1}}}
    \newcommand{\ControlFlowTok}[1]{\textcolor[rgb]{0.00,0.44,0.13}{\textbf{{#1}}}}
    \newcommand{\OperatorTok}[1]{\textcolor[rgb]{0.40,0.40,0.40}{{#1}}}
    \newcommand{\BuiltInTok}[1]{{#1}}
    \newcommand{\ExtensionTok}[1]{{#1}}
    \newcommand{\PreprocessorTok}[1]{\textcolor[rgb]{0.74,0.48,0.00}{{#1}}}
    \newcommand{\AttributeTok}[1]{\textcolor[rgb]{0.49,0.56,0.16}{{#1}}}
    \newcommand{\InformationTok}[1]{\textcolor[rgb]{0.38,0.63,0.69}{\textbf{\textit{{#1}}}}}
    \newcommand{\WarningTok}[1]{\textcolor[rgb]{0.38,0.63,0.69}{\textbf{\textit{{#1}}}}}
    
    
    % Define a nice break command that doesn't care if a line doesn't already
    % exist.
    \def\br{\hspace*{\fill} \\* }
    % Math Jax compatibility definitions
    \def\gt{>}
    \def\lt{<}
    \let\Oldtex\TeX
    \let\Oldlatex\LaTeX
    \renewcommand{\TeX}{\textrm{\Oldtex}}
    \renewcommand{\LaTeX}{\textrm{\Oldlatex}}
    % Document parameters
    % Document title
    \title{Tratamiento de anomalías con distintos filtros de imágenes}
    
    
    
    
    
% Pygments definitions
\makeatletter
\def\PY@reset{\let\PY@it=\relax \let\PY@bf=\relax%
    \let\PY@ul=\relax \let\PY@tc=\relax%
    \let\PY@bc=\relax \let\PY@ff=\relax}
\def\PY@tok#1{\csname PY@tok@#1\endcsname}
\def\PY@toks#1+{\ifx\relax#1\empty\else%
    \PY@tok{#1}\expandafter\PY@toks\fi}
\def\PY@do#1{\PY@bc{\PY@tc{\PY@ul{%
    \PY@it{\PY@bf{\PY@ff{#1}}}}}}}
\def\PY#1#2{\PY@reset\PY@toks#1+\relax+\PY@do{#2}}

\@namedef{PY@tok@w}{\def\PY@tc##1{\textcolor[rgb]{0.73,0.73,0.73}{##1}}}
\@namedef{PY@tok@c}{\let\PY@it=\textit\def\PY@tc##1{\textcolor[rgb]{0.25,0.50,0.50}{##1}}}
\@namedef{PY@tok@cp}{\def\PY@tc##1{\textcolor[rgb]{0.74,0.48,0.00}{##1}}}
\@namedef{PY@tok@k}{\let\PY@bf=\textbf\def\PY@tc##1{\textcolor[rgb]{0.00,0.50,0.00}{##1}}}
\@namedef{PY@tok@kp}{\def\PY@tc##1{\textcolor[rgb]{0.00,0.50,0.00}{##1}}}
\@namedef{PY@tok@kt}{\def\PY@tc##1{\textcolor[rgb]{0.69,0.00,0.25}{##1}}}
\@namedef{PY@tok@o}{\def\PY@tc##1{\textcolor[rgb]{0.40,0.40,0.40}{##1}}}
\@namedef{PY@tok@ow}{\let\PY@bf=\textbf\def\PY@tc##1{\textcolor[rgb]{0.67,0.13,1.00}{##1}}}
\@namedef{PY@tok@nb}{\def\PY@tc##1{\textcolor[rgb]{0.00,0.50,0.00}{##1}}}
\@namedef{PY@tok@nf}{\def\PY@tc##1{\textcolor[rgb]{0.00,0.00,1.00}{##1}}}
\@namedef{PY@tok@nc}{\let\PY@bf=\textbf\def\PY@tc##1{\textcolor[rgb]{0.00,0.00,1.00}{##1}}}
\@namedef{PY@tok@nn}{\let\PY@bf=\textbf\def\PY@tc##1{\textcolor[rgb]{0.00,0.00,1.00}{##1}}}
\@namedef{PY@tok@ne}{\let\PY@bf=\textbf\def\PY@tc##1{\textcolor[rgb]{0.82,0.25,0.23}{##1}}}
\@namedef{PY@tok@nv}{\def\PY@tc##1{\textcolor[rgb]{0.10,0.09,0.49}{##1}}}
\@namedef{PY@tok@no}{\def\PY@tc##1{\textcolor[rgb]{0.53,0.00,0.00}{##1}}}
\@namedef{PY@tok@nl}{\def\PY@tc##1{\textcolor[rgb]{0.63,0.63,0.00}{##1}}}
\@namedef{PY@tok@ni}{\let\PY@bf=\textbf\def\PY@tc##1{\textcolor[rgb]{0.60,0.60,0.60}{##1}}}
\@namedef{PY@tok@na}{\def\PY@tc##1{\textcolor[rgb]{0.49,0.56,0.16}{##1}}}
\@namedef{PY@tok@nt}{\let\PY@bf=\textbf\def\PY@tc##1{\textcolor[rgb]{0.00,0.50,0.00}{##1}}}
\@namedef{PY@tok@nd}{\def\PY@tc##1{\textcolor[rgb]{0.67,0.13,1.00}{##1}}}
\@namedef{PY@tok@s}{\def\PY@tc##1{\textcolor[rgb]{0.73,0.13,0.13}{##1}}}
\@namedef{PY@tok@sd}{\let\PY@it=\textit\def\PY@tc##1{\textcolor[rgb]{0.73,0.13,0.13}{##1}}}
\@namedef{PY@tok@si}{\let\PY@bf=\textbf\def\PY@tc##1{\textcolor[rgb]{0.73,0.40,0.53}{##1}}}
\@namedef{PY@tok@se}{\let\PY@bf=\textbf\def\PY@tc##1{\textcolor[rgb]{0.73,0.40,0.13}{##1}}}
\@namedef{PY@tok@sr}{\def\PY@tc##1{\textcolor[rgb]{0.73,0.40,0.53}{##1}}}
\@namedef{PY@tok@ss}{\def\PY@tc##1{\textcolor[rgb]{0.10,0.09,0.49}{##1}}}
\@namedef{PY@tok@sx}{\def\PY@tc##1{\textcolor[rgb]{0.00,0.50,0.00}{##1}}}
\@namedef{PY@tok@m}{\def\PY@tc##1{\textcolor[rgb]{0.40,0.40,0.40}{##1}}}
\@namedef{PY@tok@gh}{\let\PY@bf=\textbf\def\PY@tc##1{\textcolor[rgb]{0.00,0.00,0.50}{##1}}}
\@namedef{PY@tok@gu}{\let\PY@bf=\textbf\def\PY@tc##1{\textcolor[rgb]{0.50,0.00,0.50}{##1}}}
\@namedef{PY@tok@gd}{\def\PY@tc##1{\textcolor[rgb]{0.63,0.00,0.00}{##1}}}
\@namedef{PY@tok@gi}{\def\PY@tc##1{\textcolor[rgb]{0.00,0.63,0.00}{##1}}}
\@namedef{PY@tok@gr}{\def\PY@tc##1{\textcolor[rgb]{1.00,0.00,0.00}{##1}}}
\@namedef{PY@tok@ge}{\let\PY@it=\textit}
\@namedef{PY@tok@gs}{\let\PY@bf=\textbf}
\@namedef{PY@tok@gp}{\let\PY@bf=\textbf\def\PY@tc##1{\textcolor[rgb]{0.00,0.00,0.50}{##1}}}
\@namedef{PY@tok@go}{\def\PY@tc##1{\textcolor[rgb]{0.53,0.53,0.53}{##1}}}
\@namedef{PY@tok@gt}{\def\PY@tc##1{\textcolor[rgb]{0.00,0.27,0.87}{##1}}}
\@namedef{PY@tok@err}{\def\PY@bc##1{{\setlength{\fboxsep}{\string -\fboxrule}\fcolorbox[rgb]{1.00,0.00,0.00}{1,1,1}{\strut ##1}}}}
\@namedef{PY@tok@kc}{\let\PY@bf=\textbf\def\PY@tc##1{\textcolor[rgb]{0.00,0.50,0.00}{##1}}}
\@namedef{PY@tok@kd}{\let\PY@bf=\textbf\def\PY@tc##1{\textcolor[rgb]{0.00,0.50,0.00}{##1}}}
\@namedef{PY@tok@kn}{\let\PY@bf=\textbf\def\PY@tc##1{\textcolor[rgb]{0.00,0.50,0.00}{##1}}}
\@namedef{PY@tok@kr}{\let\PY@bf=\textbf\def\PY@tc##1{\textcolor[rgb]{0.00,0.50,0.00}{##1}}}
\@namedef{PY@tok@bp}{\def\PY@tc##1{\textcolor[rgb]{0.00,0.50,0.00}{##1}}}
\@namedef{PY@tok@fm}{\def\PY@tc##1{\textcolor[rgb]{0.00,0.00,1.00}{##1}}}
\@namedef{PY@tok@vc}{\def\PY@tc##1{\textcolor[rgb]{0.10,0.09,0.49}{##1}}}
\@namedef{PY@tok@vg}{\def\PY@tc##1{\textcolor[rgb]{0.10,0.09,0.49}{##1}}}
\@namedef{PY@tok@vi}{\def\PY@tc##1{\textcolor[rgb]{0.10,0.09,0.49}{##1}}}
\@namedef{PY@tok@vm}{\def\PY@tc##1{\textcolor[rgb]{0.10,0.09,0.49}{##1}}}
\@namedef{PY@tok@sa}{\def\PY@tc##1{\textcolor[rgb]{0.73,0.13,0.13}{##1}}}
\@namedef{PY@tok@sb}{\def\PY@tc##1{\textcolor[rgb]{0.73,0.13,0.13}{##1}}}
\@namedef{PY@tok@sc}{\def\PY@tc##1{\textcolor[rgb]{0.73,0.13,0.13}{##1}}}
\@namedef{PY@tok@dl}{\def\PY@tc##1{\textcolor[rgb]{0.73,0.13,0.13}{##1}}}
\@namedef{PY@tok@s2}{\def\PY@tc##1{\textcolor[rgb]{0.73,0.13,0.13}{##1}}}
\@namedef{PY@tok@sh}{\def\PY@tc##1{\textcolor[rgb]{0.73,0.13,0.13}{##1}}}
\@namedef{PY@tok@s1}{\def\PY@tc##1{\textcolor[rgb]{0.73,0.13,0.13}{##1}}}
\@namedef{PY@tok@mb}{\def\PY@tc##1{\textcolor[rgb]{0.40,0.40,0.40}{##1}}}
\@namedef{PY@tok@mf}{\def\PY@tc##1{\textcolor[rgb]{0.40,0.40,0.40}{##1}}}
\@namedef{PY@tok@mh}{\def\PY@tc##1{\textcolor[rgb]{0.40,0.40,0.40}{##1}}}
\@namedef{PY@tok@mi}{\def\PY@tc##1{\textcolor[rgb]{0.40,0.40,0.40}{##1}}}
\@namedef{PY@tok@il}{\def\PY@tc##1{\textcolor[rgb]{0.40,0.40,0.40}{##1}}}
\@namedef{PY@tok@mo}{\def\PY@tc##1{\textcolor[rgb]{0.40,0.40,0.40}{##1}}}
\@namedef{PY@tok@ch}{\let\PY@it=\textit\def\PY@tc##1{\textcolor[rgb]{0.25,0.50,0.50}{##1}}}
\@namedef{PY@tok@cm}{\let\PY@it=\textit\def\PY@tc##1{\textcolor[rgb]{0.25,0.50,0.50}{##1}}}
\@namedef{PY@tok@cpf}{\let\PY@it=\textit\def\PY@tc##1{\textcolor[rgb]{0.25,0.50,0.50}{##1}}}
\@namedef{PY@tok@c1}{\let\PY@it=\textit\def\PY@tc##1{\textcolor[rgb]{0.25,0.50,0.50}{##1}}}
\@namedef{PY@tok@cs}{\let\PY@it=\textit\def\PY@tc##1{\textcolor[rgb]{0.25,0.50,0.50}{##1}}}

\def\PYZbs{\char`\\}
\def\PYZus{\char`\_}
\def\PYZob{\char`\{}
\def\PYZcb{\char`\}}
\def\PYZca{\char`\^}
\def\PYZam{\char`\&}
\def\PYZlt{\char`\<}
\def\PYZgt{\char`\>}
\def\PYZsh{\char`\#}
\def\PYZpc{\char`\%}
\def\PYZdl{\char`\$}
\def\PYZhy{\char`\-}
\def\PYZsq{\char`\'}
\def\PYZdq{\char`\"}
\def\PYZti{\char`\~}
% for compatibility with earlier versions
\def\PYZat{@}
\def\PYZlb{[}
\def\PYZrb{]}
\makeatother


    % For linebreaks inside Verbatim environment from package fancyvrb. 
    \makeatletter
        \newbox\Wrappedcontinuationbox 
        \newbox\Wrappedvisiblespacebox 
        \newcommand*\Wrappedvisiblespace {\textcolor{red}{\textvisiblespace}} 
        \newcommand*\Wrappedcontinuationsymbol {\textcolor{red}{\llap{\tiny$\m@th\hookrightarrow$}}} 
        \newcommand*\Wrappedcontinuationindent {3ex } 
        \newcommand*\Wrappedafterbreak {\kern\Wrappedcontinuationindent\copy\Wrappedcontinuationbox} 
        % Take advantage of the already applied Pygments mark-up to insert 
        % potential linebreaks for TeX processing. 
        %        {, <, #, %, $, ' and ": go to next line. 
        %        _, }, ^, &, >, - and ~: stay at end of broken line. 
        % Use of \textquotesingle for straight quote. 
        \newcommand*\Wrappedbreaksatspecials {% 
            \def\PYGZus{\discretionary{\char`\_}{\Wrappedafterbreak}{\char`\_}}% 
            \def\PYGZob{\discretionary{}{\Wrappedafterbreak\char`\{}{\char`\{}}% 
            \def\PYGZcb{\discretionary{\char`\}}{\Wrappedafterbreak}{\char`\}}}% 
            \def\PYGZca{\discretionary{\char`\^}{\Wrappedafterbreak}{\char`\^}}% 
            \def\PYGZam{\discretionary{\char`\&}{\Wrappedafterbreak}{\char`\&}}% 
            \def\PYGZlt{\discretionary{}{\Wrappedafterbreak\char`\<}{\char`\<}}% 
            \def\PYGZgt{\discretionary{\char`\>}{\Wrappedafterbreak}{\char`\>}}% 
            \def\PYGZsh{\discretionary{}{\Wrappedafterbreak\char`\#}{\char`\#}}% 
            \def\PYGZpc{\discretionary{}{\Wrappedafterbreak\char`\%}{\char`\%}}% 
            \def\PYGZdl{\discretionary{}{\Wrappedafterbreak\char`\$}{\char`\$}}% 
            \def\PYGZhy{\discretionary{\char`\-}{\Wrappedafterbreak}{\char`\-}}% 
            \def\PYGZsq{\discretionary{}{\Wrappedafterbreak\textquotesingle}{\textquotesingle}}% 
            \def\PYGZdq{\discretionary{}{\Wrappedafterbreak\char`\"}{\char`\"}}% 
            \def\PYGZti{\discretionary{\char`\~}{\Wrappedafterbreak}{\char`\~}}% 
        } 
        % Some characters . , ; ? ! / are not pygmentized. 
        % This macro makes them "active" and they will insert potential linebreaks 
        \newcommand*\Wrappedbreaksatpunct {% 
            \lccode`\~`\.\lowercase{\def~}{\discretionary{\hbox{\char`\.}}{\Wrappedafterbreak}{\hbox{\char`\.}}}% 
            \lccode`\~`\,\lowercase{\def~}{\discretionary{\hbox{\char`\,}}{\Wrappedafterbreak}{\hbox{\char`\,}}}% 
            \lccode`\~`\;\lowercase{\def~}{\discretionary{\hbox{\char`\;}}{\Wrappedafterbreak}{\hbox{\char`\;}}}% 
            \lccode`\~`\:\lowercase{\def~}{\discretionary{\hbox{\char`\:}}{\Wrappedafterbreak}{\hbox{\char`\:}}}% 
            \lccode`\~`\?\lowercase{\def~}{\discretionary{\hbox{\char`\?}}{\Wrappedafterbreak}{\hbox{\char`\?}}}% 
            \lccode`\~`\!\lowercase{\def~}{\discretionary{\hbox{\char`\!}}{\Wrappedafterbreak}{\hbox{\char`\!}}}% 
            \lccode`\~`\/\lowercase{\def~}{\discretionary{\hbox{\char`\/}}{\Wrappedafterbreak}{\hbox{\char`\/}}}% 
            \catcode`\.\active
            \catcode`\,\active 
            \catcode`\;\active
            \catcode`\:\active
            \catcode`\?\active
            \catcode`\!\active
            \catcode`\/\active 
            \lccode`\~`\~ 	
        }
    \makeatother

    \let\OriginalVerbatim=\Verbatim
    \makeatletter
    \renewcommand{\Verbatim}[1][1]{%
        %\parskip\z@skip
        \sbox\Wrappedcontinuationbox {\Wrappedcontinuationsymbol}%
        \sbox\Wrappedvisiblespacebox {\FV@SetupFont\Wrappedvisiblespace}%
        \def\FancyVerbFormatLine ##1{\hsize\linewidth
            \vtop{\raggedright\hyphenpenalty\z@\exhyphenpenalty\z@
                \doublehyphendemerits\z@\finalhyphendemerits\z@
                \strut ##1\strut}%
        }%
        % If the linebreak is at a space, the latter will be displayed as visible
        % space at end of first line, and a continuation symbol starts next line.
        % Stretch/shrink are however usually zero for typewriter font.
        \def\FV@Space {%
            \nobreak\hskip\z@ plus\fontdimen3\font minus\fontdimen4\font
            \discretionary{\copy\Wrappedvisiblespacebox}{\Wrappedafterbreak}
            {\kern\fontdimen2\font}%
        }%
        
        % Allow breaks at special characters using \PYG... macros.
        \Wrappedbreaksatspecials
        % Breaks at punctuation characters . , ; ? ! and / need catcode=\active 	
        \OriginalVerbatim[#1,codes*=\Wrappedbreaksatpunct]%
    }
    \makeatother

    % Exact colors from NB
    \definecolor{incolor}{HTML}{303F9F}
    \definecolor{outcolor}{HTML}{D84315}
    \definecolor{cellborder}{HTML}{CFCFCF}
    \definecolor{cellbackground}{HTML}{F7F7F7}
    
    % prompt
    \makeatletter
    \newcommand{\boxspacing}{\kern\kvtcb@left@rule\kern\kvtcb@boxsep}
    \makeatother
    \newcommand{\prompt}[4]{
        {\ttfamily\llap{{\color{#2}[#3]:\hspace{3pt}#4}}\vspace{-\baselineskip}}
    }
    

    
    % Prevent overflowing lines due to hard-to-break entities
    \sloppy 
    % Setup hyperref package
    \hypersetup{
      breaklinks=true,  % so long urls are correctly broken across lines
      colorlinks=true,
      urlcolor=urlcolor,
      linkcolor=linkcolor,
      citecolor=citecolor,
      }
    % Slightly bigger margins than the latex defaults
    
    \geometry{verbose,tmargin=1in,bmargin=1in,lmargin=1in,rmargin=1in}
    
    % UNIR
    \usepackage[spanish,mexico]{babel}
    \makeatletter
    \let\newtitle\@title
    \makeatother
    \usepackage{amsmath}
    \usepackage{multirow}
    \definecolor{UnirLight}{HTML}{E6F4F9}
    \definecolor{UnirDark}{HTML}{0098CD}
    \arrayrulecolor{UnirDark}
    \usepackage{titlesec}
    \titleformat*{\section}{\color{UnirDark}\normalsize\bfseries}
    \titleformat*{\subsection}{\color{UnirDark}\normalsize\bfseries}
    \titleformat*{\subsubsection}{\color{UnirDark}\normalsize\bfseries}
    \usepackage{fancyhdr}
    \pagestyle{fancy}
    \renewcommand{\headrulewidth}{0pt}
    \headheight=45pt
    \setlength{\footskip}{64pt}
    \lhead{}
    \chead{
    \begin{tabular}{|c|l|c|}
     \hline
     \rowcolor{UnirLight}
     \textcolor{UnirDark}{Asignatura} & \textcolor{UnirDark}{Datos del alumno} & \textcolor{UnirDark}{Fecha} \\
     \hline
     \multirow{2}{12em}{\textbf{Percepción computacional}} & Apellidos: Domínguez Espinoza & \multirow{2}{6em}{\today} \\
     & Nombre: Edgar Uriel & \\
     \hline
    \end{tabular}}
    \rhead{}
    \lfoot{}
    \cfoot{}
    \rfoot{\makebox(70,56)[t]{\textcolor{UnirDark}{Actividades}}
        \colorbox{UnirDark}{
            \makebox(10,56)[t]{
                \textcolor{white}{\thepage}}}}
    \usepackage[color={[gray]{0.5}}, angle=90,fontsize=9pt,anchor=lb,pos={0.03\paperwidth,0.95\paperheight}]{draftwatermark}
    \SetWatermarkText{{\copyright} Universidad Internacional de La Rioja en México (UNIR)}
    \hypersetup{
      pdfauthor={Edgar Uriel Domínguez Espinoza},
      pdftitle={Tratamiento de anomalías con distintos filtros de imágenes},
      pdfkeywords={filtro, Percepción Computacional, imágenes, ruido},
      pdfsubject={Percepción computacional},
      pdfcreator={Emacs 27.2}, 
      pdflang={Spanish}}
    \usepackage[round]{natbib}

\begin{document}
    
    
    

    
    \hypertarget{tratamiento-de-anomaluxedas-con-distintos-filtros-de-imuxe1genes.}{%
\textcolor{UnirDark}{\Large\bfseries\newtitle}\label{tratamiento-de-anomaluxedas-con-distintos-filtros-de-imuxe1genes.}}

    \hypertarget{introducciuxf3n}{%
\section{Introducción}\label{introducciuxf3n}}

En este trabajo se repasan diferentes filtros útiles para la eliminación de anomalías en imágenes: Filtro por promedio, filtro gausiano y filtro de mediana. En el trabajo se usan imágenes con aomalías debido a ruido \emph{sal y pimienta} y ruido \emph{gausiano}, también se usa una imagen de control que se usará para ejemplificar que efecto tiene el filtro sobre cualquier imagen.

También se inspecciona la diferencia de tratar una determinada anomalía con un tipo de filtro en particular y finalmente se ejemplifica la diferencia entre distintas implementaciones de un filtro.

\hypertarget{libreruxedas-a-utilizar}{%
\section{Librerías a utilizar}\label{libreruxedas-a-utilizar}}

En la presente trabajo se usarán las librerías OpenCV para el procesamiento de imágenes y Numpy para manejar las imágenes como si fueran arreglos bidimensionales.

    \begin{tcolorbox}[breakable, size=fbox, boxrule=1pt, pad at break*=1mm,colback=cellbackground, colframe=cellborder]
\prompt{In}{incolor}{1}{\boxspacing}
\begin{Verbatim}[commandchars=\\\{\}]
\PY{k+kn}{import} \PY{n+nn}{cv2} \PY{k}{as} \PY{n+nn}{cv}
\PY{k+kn}{import} \PY{n+nn}{numpy} \PY{k}{as} \PY{n+nn}{np}
\end{Verbatim}
\end{tcolorbox}

    La librería matplotlib será usada para presentar las imágenes dentro del notebook de python. Esta librería espera imágenes RGB de otra manera los tonos y colores serán distintos. Para ver las imágenes de salida con detalle es necesario ir a la carpeta \texttt{out}. Las imágenes de entrada se encuentran en la carpeta \texttt{im}.

    \begin{tcolorbox}[breakable, size=fbox, boxrule=1pt, pad at break*=1mm,colback=cellbackground, colframe=cellborder]
\prompt{In}{incolor}{2}{\boxspacing}
\begin{Verbatim}[commandchars=\\\{\}]
\PY{k+kn}{from} \PY{n+nn}{matplotlib} \PY{k+kn}{import} \PY{n}{pyplot} \PY{k}{as} \PY{n}{plt}
\end{Verbatim}
\end{tcolorbox}

    \hypertarget{carga-de-imuxe1genes}{%
\section{Carga de imágenes}\label{carga-de-imuxe1genes}}

Aunque se deja en el código un mayor número de imágenes de prueba, en este trabajo usaremos tres imágenes base:

\begin{enumerate}
\def\labelenumi{\arabic{enumi}.}
\tightlist
\item
  \texttt{unir-1.jpg}: Es la imagen que servirá como control. No tiene
  anomalías aparentes.
\item
  \texttt{salt-pepper-1.png}: Esta imagen es obtenida de internet, tiene
  ruido conocido como \emph{sal y pimienta}.
\item
  \texttt{noisy-1.jpg}: Esta es una imagen antigua tomada en 2008 con
  una cámara Nikon, el ruido que presenta lo supondremos como
  \emph{gausiano}.
\end{enumerate}

Por medio de los comentarios en el código es posible cambiar las rutas fácilmente y hacer otras pruebas.

    \begin{tcolorbox}[breakable, size=fbox, boxrule=1pt, pad at break*=1mm,colback=cellbackground, colframe=cellborder]
\prompt{In}{incolor}{3}{\boxspacing}
\begin{Verbatim}[commandchars=\\\{\}]
\PY{n}{image\PYZus{}unir} \PY{o}{=} \PY{n}{cv}\PY{o}{.}\PY{n}{imread}\PY{p}{(}\PY{l+s+s1}{\PYZsq{}}\PY{l+s+s1}{im/unir\PYZhy{}1.jpg}\PY{l+s+s1}{\PYZsq{}}\PY{p}{)}
\PY{n}{image\PYZus{}snp} \PY{o}{=} \PY{n}{cv}\PY{o}{.}\PY{n}{imread}\PY{p}{(}\PY{l+s+s1}{\PYZsq{}}\PY{l+s+s1}{im/salt\PYZhy{}pepper\PYZhy{}1.png}\PY{l+s+s1}{\PYZsq{}}\PY{p}{)}
\PY{n}{image\PYZus{}noisy} \PY{o}{=} \PY{n}{cv}\PY{o}{.}\PY{n}{imread}\PY{p}{(}\PY{l+s+s1}{\PYZsq{}}\PY{l+s+s1}{im/noisy\PYZhy{}1.jpg}\PY{l+s+s1}{\PYZsq{}}\PY{p}{)}
\end{Verbatim}
\end{tcolorbox}

    \hypertarget{filtro-de-convoluciuxf3n-por-promedio}{%
\subsection{Filtro de convolución por
promedio}\label{filtro-de-convoluciuxf3n-por-promedio}}

En un filtro de convolución por promedio, el sistema pasa fácilmente los componentes de baja frecuencia de la señal y suprime los componentes de alta frecuencia. Cuando se aplica a una imagen, reemplaza cada píxel en la entrada por el promedio de los valores de un conjunto de sus píxeles vecinos {[}\ldots{]} Las anomalías se suavizan debido al promedio. La borrosidad aumentará proporcionalmente con filtros más grandes. Este filtro es separable. Multiplicando el filtro de columna con el filtro de fila, que es la transposición del filtro de columna, obtenemos el filtro promedio.\cite[cap. 2.4.2]{Sundararajan_2017}

Para usar este filtro primero es necesario crear una función que genere rápidamente una máscara de tamaño nxn. Está máscara es un arreglo con valores 1 y será multiplicado por un escalar que permita obtener el promedio.

    \begin{tcolorbox}[breakable, size=fbox, boxrule=1pt, pad at break*=1mm,colback=cellbackground, colframe=cellborder]
\prompt{In}{incolor}{4}{\boxspacing}
\begin{Verbatim}[commandchars=\\\{\}]
\PY{k}{def} \PY{n+nf}{mask}\PY{p}{(}\PY{n}{n}\PY{p}{)}\PY{p}{:}
    \PY{l+s+sd}{\PYZsq{}\PYZsq{}\PYZsq{}Define a nxn mask.}
\PY{l+s+sd}{    Useful for a quick filter}
\PY{l+s+sd}{    \PYZsq{}\PYZsq{}\PYZsq{}}
    \PY{n}{mask} \PY{o}{=} \PY{n}{np}\PY{o}{.}\PY{n}{ones}\PY{p}{(}\PY{p}{(}\PY{n}{n}\PY{p}{,}\PY{n}{n}\PY{p}{)}\PY{p}{,} \PY{n}{np}\PY{o}{.}\PY{n}{float32}\PY{p}{)}\PY{o}{*}\PY{p}{(}\PY{l+m+mi}{1}\PY{o}{/}\PY{p}{(}\PY{n}{n}\PY{o}{*}\PY{o}{*}\PY{l+m+mi}{2}\PY{p}{)}\PY{p}{)}
    \PY{k}{return} \PY{n}{mask}
\PY{c+c1}{\PYZsh{} Obtener una máscara particular de tamaño 5x5}
\PY{n}{kernel} \PY{o}{=} \PY{n}{mask}\PY{p}{(}\PY{l+m+mi}{5}\PY{p}{)}
\PY{n+nb}{print}\PY{p}{(}\PY{n}{kernel}\PY{p}{)}
\end{Verbatim}
\end{tcolorbox}

    \begin{Verbatim}[commandchars=\\\{\}]
[[0.04 0.04 0.04 0.04 0.04]
 [0.04 0.04 0.04 0.04 0.04]
 [0.04 0.04 0.04 0.04 0.04]
 [0.04 0.04 0.04 0.04 0.04]
 [0.04 0.04 0.04 0.04 0.04]]
    \end{Verbatim}

    Ahora se debe probar con la función \texttt{cv.filter2D()}. No es necesario hacer nuevas funciones. Usaremos la imagen de control para observar el efecto del filtro.

    \begin{tcolorbox}[breakable, size=fbox, boxrule=1pt, pad at break*=1mm,colback=cellbackground, colframe=cellborder]
\prompt{In}{incolor}{5}{\boxspacing}
\begin{Verbatim}[commandchars=\\\{\}]
\PY{n}{im} \PY{o}{=} \PY{n}{image\PYZus{}unir}
\PY{c+c1}{\PYZsh{} El filtro recibe como parámetros: 1. la imagen, 2. la profundidad, \PYZhy{}1 conserva el valor original, 3. el kernel a aplicar.}
\PY{n}{out\PYZus{}5} \PY{o}{=} \PY{n}{cv}\PY{o}{.}\PY{n}{filter2D}\PY{p}{(}\PY{n}{im}\PY{p}{,}\PY{o}{\PYZhy{}}\PY{l+m+mi}{1}\PY{p}{,}\PY{n}{kernel}\PY{p}{)}
\PY{n}{cv}\PY{o}{.}\PY{n}{imwrite}\PY{p}{(}\PY{l+s+s1}{\PYZsq{}}\PY{l+s+s1}{out/filter2D.jpg}\PY{l+s+s1}{\PYZsq{}}\PY{p}{,} \PY{n}{out\PYZus{}5}\PY{p}{)}
\PY{n}{kernel} \PY{o}{=} \PY{n}{mask}\PY{p}{(}\PY{l+m+mi}{50}\PY{p}{)}
\PY{n}{out\PYZus{}50} \PY{o}{=} \PY{n}{cv}\PY{o}{.}\PY{n}{filter2D}\PY{p}{(}\PY{n}{im}\PY{p}{,}\PY{o}{\PYZhy{}}\PY{l+m+mi}{1}\PY{p}{,}\PY{n}{kernel}\PY{p}{)}
\PY{n}{cv}\PY{o}{.}\PY{n}{imwrite}\PY{p}{(}\PY{l+s+s1}{\PYZsq{}}\PY{l+s+s1}{out/filter2D\PYZhy{}bigKernel.jpg}\PY{l+s+s1}{\PYZsq{}}\PY{p}{,} \PY{n}{out\PYZus{}50}\PY{p}{)}
\end{Verbatim}
\end{tcolorbox}

%%             \begin{tcolorbox}[breakable, size=fbox, boxrule=.5pt, pad at break*=1mm, opacityfill=0]
%% \prompt{Out}{outcolor}{5}{\boxspacing}
%% \begin{Verbatim}[commandchars=\\\{\}]
%% True
%% \end{Verbatim}
%% \end{tcolorbox}
        
    Puede observarse como resultado del filtro que la imagen se difumina ligeramente, sobre todo en el texto. Ahora si el valor de la máscara es muy alto, por ejemplo cincuenta, se notará un fuerte efecto de difuminado en la imagen.

%%     \begin{tcolorbox}[breakable, size=fbox, boxrule=1pt, pad at break*=1mm,colback=cellbackground, colframe=cellborder]
%% \prompt{In}{incolor}{6}{\boxspacing}
%% \begin{Verbatim}[commandchars=\\\{\}]
%% \PY{n}{im} \PY{o}{=} \PY{n}{cv}\PY{o}{.}\PY{n}{cvtColor}\PY{p}{(}\PY{n}{im}\PY{p}{,} \PY{n}{cv}\PY{o}{.}\PY{n}{COLOR\PYZus{}BGR2RGB}\PY{p}{)}
%% \PY{n}{out\PYZus{}5} \PY{o}{=} \PY{n}{cv}\PY{o}{.}\PY{n}{cvtColor}\PY{p}{(}\PY{n}{out\PYZus{}5}\PY{p}{,} \PY{n}{cv}\PY{o}{.}\PY{n}{COLOR\PYZus{}BGR2RGB}\PY{p}{)}
%% \PY{n}{out\PYZus{}50} \PY{o}{=} \PY{n}{cv}\PY{o}{.}\PY{n}{cvtColor}\PY{p}{(}\PY{n}{out\PYZus{}50}\PY{p}{,} \PY{n}{cv}\PY{o}{.}\PY{n}{COLOR\PYZus{}BGR2RGB}\PY{p}{)}
%% \PY{n}{plt}\PY{o}{.}\PY{n}{subplot}\PY{p}{(}\PY{l+m+mi}{131}\PY{p}{)}
%% \PY{n}{plt}\PY{o}{.}\PY{n}{imshow}\PY{p}{(}\PY{n}{im}\PY{p}{)}
%% \PY{n}{plt}\PY{o}{.}\PY{n}{title}\PY{p}{(}\PY{l+s+s1}{\PYZsq{}}\PY{l+s+s1}{Imagen Original}\PY{l+s+s1}{\PYZsq{}}\PY{p}{)}\PY{p}{,} \PY{n}{plt}\PY{o}{.}\PY{n}{xticks}\PY{p}{(}\PY{p}{[}\PY{p}{]}\PY{p}{)}\PY{p}{,} \PY{n}{plt}\PY{o}{.}\PY{n}{yticks}\PY{p}{(}\PY{p}{[}\PY{p}{]}\PY{p}{)}
%% \PY{n}{plt}\PY{o}{.}\PY{n}{subplot}\PY{p}{(}\PY{l+m+mi}{132}\PY{p}{)}
%% \PY{n}{plt}\PY{o}{.}\PY{n}{imshow}\PY{p}{(}\PY{n}{out\PYZus{}5}\PY{p}{)}
%% \PY{n}{plt}\PY{o}{.}\PY{n}{title}\PY{p}{(}\PY{l+s+s1}{\PYZsq{}}\PY{l+s+s1}{Filtro 5x5}\PY{l+s+s1}{\PYZsq{}}\PY{p}{)}\PY{p}{,} \PY{n}{plt}\PY{o}{.}\PY{n}{xticks}\PY{p}{(}\PY{p}{[}\PY{p}{]}\PY{p}{)}\PY{p}{,} \PY{n}{plt}\PY{o}{.}\PY{n}{yticks}\PY{p}{(}\PY{p}{[}\PY{p}{]}\PY{p}{)}
%% \PY{n}{plt}\PY{o}{.}\PY{n}{subplot}\PY{p}{(}\PY{l+m+mi}{133}\PY{p}{)}
%% \PY{n}{plt}\PY{o}{.}\PY{n}{imshow}\PY{p}{(}\PY{n}{out\PYZus{}50}\PY{p}{)}
%% \PY{n}{plt}\PY{o}{.}\PY{n}{title}\PY{p}{(}\PY{l+s+s1}{\PYZsq{}}\PY{l+s+s1}{Filtro 50x50}\PY{l+s+s1}{\PYZsq{}}\PY{p}{)}\PY{p}{,} \PY{n}{plt}\PY{o}{.}\PY{n}{xticks}\PY{p}{(}\PY{p}{[}\PY{p}{]}\PY{p}{)}\PY{p}{,} \PY{n}{plt}\PY{o}{.}\PY{n}{yticks}\PY{p}{(}\PY{p}{[}\PY{p}{]}\PY{p}{)}
%% \PY{n}{plt}\PY{o}{.}\PY{n}{show}\PY{p}{(}\PY{p}{)}
%% \end{Verbatim}
%% \end{tcolorbox}

    \begin{center}
    \adjustimage{max size={0.9\linewidth}{0.9\paperheight}}{output_12_0.png}
    \end{center}
    { \hspace*{\fill} \\}
    
    El efecto de  limpieza de ruido con este  filtro pasa baja (LPF) puede  observarse al usarlo
    con las imágenes guardadas un las variables \texttt{image\_noisy} e \texttt{image\_snp}.

    \begin{tcolorbox}[breakable, size=fbox, boxrule=1pt, pad at break*=1mm,colback=cellbackground, colframe=cellborder]
\prompt{In}{incolor}{7}{\boxspacing}
\begin{Verbatim}[commandchars=\\\{\}]
\PY{n}{im} \PY{o}{=} \PY{n}{image\PYZus{}noisy}
\PY{c+c1}{\PYZsh{} Cambie el tamaño de la máscara si así lo desea}
\PY{n}{kernel} \PY{o}{=} \PY{n}{mask}\PY{p}{(}\PY{l+m+mi}{10}\PY{p}{)}
\PY{n}{out} \PY{o}{=} \PY{n}{cv}\PY{o}{.}\PY{n}{filter2D}\PY{p}{(}\PY{n}{im}\PY{p}{,}\PY{o}{\PYZhy{}}\PY{l+m+mi}{1}\PY{p}{,}\PY{n}{kernel}\PY{p}{)}
\PY{n}{cv}\PY{o}{.}\PY{n}{imwrite}\PY{p}{(}\PY{l+s+s1}{\PYZsq{}}\PY{l+s+s1}{out/filter2D\PYZhy{}cleaning.jpg}\PY{l+s+s1}{\PYZsq{}}\PY{p}{,} \PY{n}{out}\PY{p}{)}
\end{Verbatim}
\end{tcolorbox}

%%             \begin{tcolorbox}[breakable, size=fbox, boxrule=.5pt, pad at break*=1mm, opacityfill=0]
%% \prompt{Out}{outcolor}{7}{\boxspacing}
%% \begin{Verbatim}[commandchars=\\\{\}]
%% True
%% \end{Verbatim}
%% \end{tcolorbox}
        
%%     \begin{tcolorbox}[breakable, size=fbox, boxrule=1pt, pad at break*=1mm,colback=cellbackground, colframe=cellborder]
%% \prompt{In}{incolor}{8}{\boxspacing}
%% \begin{Verbatim}[commandchars=\\\{\}]
%% \PY{n}{im} \PY{o}{=} \PY{n}{cv}\PY{o}{.}\PY{n}{cvtColor}\PY{p}{(}\PY{n}{im}\PY{p}{,} \PY{n}{cv}\PY{o}{.}\PY{n}{COLOR\PYZus{}BGR2RGB}\PY{p}{)}
%% \PY{n}{out} \PY{o}{=} \PY{n}{cv}\PY{o}{.}\PY{n}{cvtColor}\PY{p}{(}\PY{n}{out}\PY{p}{,} \PY{n}{cv}\PY{o}{.}\PY{n}{COLOR\PYZus{}BGR2RGB}\PY{p}{)}
%% \PY{n}{plt}\PY{o}{.}\PY{n}{subplot}\PY{p}{(}\PY{l+m+mi}{1}\PY{p}{,}\PY{l+m+mi}{2}\PY{p}{,}\PY{l+m+mi}{1}\PY{p}{)}
%% \PY{n}{plt}\PY{o}{.}\PY{n}{imshow}\PY{p}{(}\PY{n}{im}\PY{p}{)}
%% \PY{n}{plt}\PY{o}{.}\PY{n}{title}\PY{p}{(}\PY{l+s+s1}{\PYZsq{}}\PY{l+s+s1}{Imagen Original}\PY{l+s+s1}{\PYZsq{}}\PY{p}{)}\PY{p}{,} \PY{n}{plt}\PY{o}{.}\PY{n}{xticks}\PY{p}{(}\PY{p}{[}\PY{p}{]}\PY{p}{)}\PY{p}{,} \PY{n}{plt}\PY{o}{.}\PY{n}{yticks}\PY{p}{(}\PY{p}{[}\PY{p}{]}\PY{p}{)}
%% \PY{n}{plt}\PY{o}{.}\PY{n}{subplot}\PY{p}{(}\PY{l+m+mi}{1}\PY{p}{,}\PY{l+m+mi}{2}\PY{p}{,}\PY{l+m+mi}{2}\PY{p}{)} 
%% \PY{n}{plt}\PY{o}{.}\PY{n}{imshow}\PY{p}{(}\PY{n}{out}\PY{p}{)}
%% \PY{n}{plt}\PY{o}{.}\PY{n}{title}\PY{p}{(}\PY{l+s+s1}{\PYZsq{}}\PY{l+s+s1}{Filtro promedio}\PY{l+s+s1}{\PYZsq{}}\PY{p}{)}\PY{p}{,} \PY{n}{plt}\PY{o}{.}\PY{n}{xticks}\PY{p}{(}\PY{p}{[}\PY{p}{]}\PY{p}{)}\PY{p}{,} \PY{n}{plt}\PY{o}{.}\PY{n}{yticks}\PY{p}{(}\PY{p}{[}\PY{p}{]}\PY{p}{)}
%% \PY{n}{plt}\PY{o}{.}\PY{n}{show}\PY{p}{(}\PY{p}{)}
%% \end{Verbatim}
%% \end{tcolorbox}

    \begin{center}
    \adjustimage{max size={0.9\linewidth}{0.9\paperheight}}{output_15_0.png}
    \end{center}
    { \hspace*{\fill} \\}
    
    Es notorio que este filtro tiene un efecto relativo sobre el ruido: en el caso \emph{sal y pimienta} limpia muy poco la imagen debido a que se encarga de las anomalías de alta frecuencia (la sal) pero no soluciona las anomalías de baja frecuencia, mientras tanto, en sobre el ruido gausiano la mejora es más notoria. Ahora se procederá a comparar este filtro con otros más y comparar los resultados.

\hypertarget{filtro-gausiano}{%
\subsection{Filtro gausiano}\label{filtro-gausiano}}

Los LPF gaussianos se basan en la función de distribución de probabilidad gaussiana. El filtro gaussiano es ampliamente utilizado.  Las características de este filtro incluyen\cite[cap. 2.4.1, 4.3]{Sundararajan_2017}:

\begin{enumerate}
\def\labelenumi{\arabic{enumi}.}
\tightlist
\item
  Simetría.
\item
  Al variar el valor de la desviación estándar, se controla el requisito
  conflictivo de menos borrosidad y más eliminación de ruido.
\item
  Los coeficientes caen a niveles insignificantes en los bordes.
\item
  La transformada de Fourier de una función gaussiana es otra función
  gaussiana.
\item
  La convolución de dos funciones gaussianas es otra función gaussiana.
\end{enumerate}

Probarémos este filtro con la imagen almacenada en \texttt{image\_noisy}, debido a la naturaleza de dicho ruido.

    \begin{tcolorbox}[breakable, size=fbox, boxrule=1pt, pad at break*=1mm,colback=cellbackground, colframe=cellborder]
\prompt{In}{incolor}{9}{\boxspacing}
\begin{Verbatim}[commandchars=\\\{\}]
\PY{n}{im} \PY{o}{=} \PY{n}{image\PYZus{}noisy}
\PY{c+c1}{\PYZsh{} Los parámetros principales son: 1. imagen, 2. Kernel gaussiano (puede ser solo las dimensiones) 3. Desviación estándar.}
\PY{n}{out} \PY{o}{=} \PY{n}{cv}\PY{o}{.}\PY{n}{GaussianBlur}\PY{p}{(}\PY{n}{im}\PY{p}{,}\PY{p}{(}\PY{l+m+mi}{9}\PY{p}{,}\PY{l+m+mi}{9}\PY{p}{)}\PY{p}{,}\PY{l+m+mi}{0}\PY{p}{)}
\PY{n}{cv}\PY{o}{.}\PY{n}{imwrite}\PY{p}{(}\PY{l+s+s1}{\PYZsq{}}\PY{l+s+s1}{out/gaussian\PYZhy{}0.jpg}\PY{l+s+s1}{\PYZsq{}}\PY{p}{,} \PY{n}{out}\PY{p}{)}
\end{Verbatim}
\end{tcolorbox}

%%             \begin{tcolorbox}[breakable, size=fbox, boxrule=.5pt, pad at break*=1mm, opacityfill=0]
%% \prompt{Out}{outcolor}{9}{\boxspacing}
%% \begin{Verbatim}[commandchars=\\\{\}]
%% True
%% \end{Verbatim}
%% \end{tcolorbox}
        
%%     \begin{tcolorbox}[breakable, size=fbox, boxrule=1pt, pad at break*=1mm,colback=cellbackground, colframe=cellborder]
%% \prompt{In}{incolor}{10}{\boxspacing}
%% \begin{Verbatim}[commandchars=\\\{\}]
%% \PY{n}{im} \PY{o}{=} \PY{n}{cv}\PY{o}{.}\PY{n}{cvtColor}\PY{p}{(}\PY{n}{im}\PY{p}{,} \PY{n}{cv}\PY{o}{.}\PY{n}{COLOR\PYZus{}BGR2RGB}\PY{p}{)}
%% \PY{n}{out} \PY{o}{=} \PY{n}{cv}\PY{o}{.}\PY{n}{cvtColor}\PY{p}{(}\PY{n}{out}\PY{p}{,} \PY{n}{cv}\PY{o}{.}\PY{n}{COLOR\PYZus{}BGR2RGB}\PY{p}{)}
%% \PY{n}{plt}\PY{o}{.}\PY{n}{subplot}\PY{p}{(}\PY{l+m+mi}{1}\PY{p}{,}\PY{l+m+mi}{2}\PY{p}{,}\PY{l+m+mi}{1}\PY{p}{)}
%% \PY{n}{plt}\PY{o}{.}\PY{n}{imshow}\PY{p}{(}\PY{n}{im}\PY{p}{)}
%% \PY{n}{plt}\PY{o}{.}\PY{n}{title}\PY{p}{(}\PY{l+s+s1}{\PYZsq{}}\PY{l+s+s1}{Imagen Original}\PY{l+s+s1}{\PYZsq{}}\PY{p}{)}\PY{p}{,} \PY{n}{plt}\PY{o}{.}\PY{n}{xticks}\PY{p}{(}\PY{p}{[}\PY{p}{]}\PY{p}{)}\PY{p}{,} \PY{n}{plt}\PY{o}{.}\PY{n}{yticks}\PY{p}{(}\PY{p}{[}\PY{p}{]}\PY{p}{)}
%% \PY{n}{plt}\PY{o}{.}\PY{n}{subplot}\PY{p}{(}\PY{l+m+mi}{1}\PY{p}{,}\PY{l+m+mi}{2}\PY{p}{,}\PY{l+m+mi}{2}\PY{p}{)} 
%% \PY{n}{plt}\PY{o}{.}\PY{n}{imshow}\PY{p}{(}\PY{n}{out}\PY{p}{)}
%% \PY{n}{plt}\PY{o}{.}\PY{n}{title}\PY{p}{(}\PY{l+s+s1}{\PYZsq{}}\PY{l+s+s1}{Filtro gaussiano}\PY{l+s+s1}{\PYZsq{}}\PY{p}{)}\PY{p}{,} \PY{n}{plt}\PY{o}{.}\PY{n}{xticks}\PY{p}{(}\PY{p}{[}\PY{p}{]}\PY{p}{)}\PY{p}{,} \PY{n}{plt}\PY{o}{.}\PY{n}{yticks}\PY{p}{(}\PY{p}{[}\PY{p}{]}\PY{p}{)}
%% \PY{n}{plt}\PY{o}{.}\PY{n}{show}\PY{p}{(}\PY{p}{)}
%% \end{Verbatim}
%% \end{tcolorbox}

    \begin{center}
    \adjustimage{max size={0.9\linewidth}{0.9\paperheight}}{output_18_0.png}
    \end{center}
    { \hspace*{\fill} \\}
    
    La función \texttt{GaussianBlur()} debe recibir un kernel $n\times m$ donde ambos son números impares aunque distintos. En el código mostrado se usa un kernel 9x9 con desviación estándar igual a cero. Bajo estas condiciones el resultado es muy similar al del filtro por promedio, sin embargo, debido al mayor número de parámetros es posible mejorar el resultado si se conserva el kernel y aumenta la desviación estándar. Si este cambio se aplicara a la imagen de control sería claro que el aumento de la desviación estándar incrementa el desenfoque. Por estos motivos, el filtro gausiano es capaz de obtener un resultado optimizado respecto al filtro por promedio.

\hypertarget{filtro-de-mediana}{%
\subsection{Filtro de mediana}\label{filtro-de-mediana}}

El filtrado de mediana permite eliminar valores bajos y altos anómalos, por eso es útil para eliminar el ruido \emph{sal y pimienta} de la imágen guardada en la variable \texttt{image\_snp}. Este filtro reemplaza un píxel por la mediana de una máscara de píxeles en su vecindad. Implica clasificar los píxeles de la máscara en orden ascendente o descendente y seleccionar el valor medio, si el número de píxeles es impar.En el caso de una imagen, todas las muestras en la máscara se enumeran como datos 1-D para el cálculo de la mediana, por lo tanto los valores superlativos siempre serán descartados, justo lo que provoca el ruido \emph{sal y pimienta}. Las máscaras de este filtro son comúnmente de tamaño $3\times 3, 5\times 5$ o $7\times 7$.\cite[cap. 2.4.2]{Sundararajan_2017}

    \begin{tcolorbox}[breakable, size=fbox, boxrule=1pt, pad at break*=1mm,colback=cellbackground, colframe=cellborder]
\prompt{In}{incolor}{11}{\boxspacing}
\begin{Verbatim}[commandchars=\\\{\}]
\PY{n}{im} \PY{o}{=} \PY{n}{image\PYZus{}snp}
\PY{c+c1}{\PYZsh{} La función tiene dos parámetros: 1. la imagen, 2. El tamaño de la máscara}
\PY{n}{out} \PY{o}{=} \PY{n}{cv}\PY{o}{.}\PY{n}{medianBlur}\PY{p}{(}\PY{n}{im}\PY{p}{,}\PY{l+m+mi}{5}\PY{p}{)}
\PY{n}{cv}\PY{o}{.}\PY{n}{imwrite}\PY{p}{(}\PY{l+s+s1}{\PYZsq{}}\PY{l+s+s1}{out/median\PYZhy{}cv.jpg}\PY{l+s+s1}{\PYZsq{}}\PY{p}{,} \PY{n}{out}\PY{p}{)}
\end{Verbatim}
\end{tcolorbox}

%%             \begin{tcolorbox}[breakable, size=fbox, boxrule=.5pt, pad at break*=1mm, opacityfill=0]
%% \prompt{Out}{outcolor}{11}{\boxspacing}
%% \begin{Verbatim}[commandchars=\\\{\}]
%% True
%% \end{Verbatim}
%% \end{tcolorbox}
        
%%     \begin{tcolorbox}[breakable, size=fbox, boxrule=1pt, pad at break*=1mm,colback=cellbackground, colframe=cellborder]
%% \prompt{In}{incolor}{12}{\boxspacing}
%% \begin{Verbatim}[commandchars=\\\{\}]
%% \PY{n}{im} \PY{o}{=} \PY{n}{cv}\PY{o}{.}\PY{n}{cvtColor}\PY{p}{(}\PY{n}{im}\PY{p}{,} \PY{n}{cv}\PY{o}{.}\PY{n}{COLOR\PYZus{}BGR2RGB}\PY{p}{)}
%% \PY{n}{out} \PY{o}{=} \PY{n}{cv}\PY{o}{.}\PY{n}{cvtColor}\PY{p}{(}\PY{n}{out}\PY{p}{,} \PY{n}{cv}\PY{o}{.}\PY{n}{COLOR\PYZus{}BGR2RGB}\PY{p}{)}
%% \PY{n}{plt}\PY{o}{.}\PY{n}{subplot}\PY{p}{(}\PY{l+m+mi}{1}\PY{p}{,}\PY{l+m+mi}{2}\PY{p}{,}\PY{l+m+mi}{1}\PY{p}{)}
%% \PY{n}{plt}\PY{o}{.}\PY{n}{imshow}\PY{p}{(}\PY{n}{im}\PY{p}{)}
%% \PY{n}{plt}\PY{o}{.}\PY{n}{title}\PY{p}{(}\PY{l+s+s1}{\PYZsq{}}\PY{l+s+s1}{Imagen Original}\PY{l+s+s1}{\PYZsq{}}\PY{p}{)}\PY{p}{,} \PY{n}{plt}\PY{o}{.}\PY{n}{xticks}\PY{p}{(}\PY{p}{[}\PY{p}{]}\PY{p}{)}\PY{p}{,} \PY{n}{plt}\PY{o}{.}\PY{n}{yticks}\PY{p}{(}\PY{p}{[}\PY{p}{]}\PY{p}{)}
%% \PY{n}{plt}\PY{o}{.}\PY{n}{subplot}\PY{p}{(}\PY{l+m+mi}{1}\PY{p}{,}\PY{l+m+mi}{2}\PY{p}{,}\PY{l+m+mi}{2}\PY{p}{)}
%% \PY{n}{plt}\PY{o}{.}\PY{n}{imshow}\PY{p}{(}\PY{n}{out}\PY{p}{)}
%% \PY{n}{plt}\PY{o}{.}\PY{n}{title}\PY{p}{(}\PY{l+s+s1}{\PYZsq{}}\PY{l+s+s1}{Filtro de mediana}\PY{l+s+s1}{\PYZsq{}}\PY{p}{)}\PY{p}{,} \PY{n}{plt}\PY{o}{.}\PY{n}{xticks}\PY{p}{(}\PY{p}{[}\PY{p}{]}\PY{p}{)}\PY{p}{,} \PY{n}{plt}\PY{o}{.}\PY{n}{yticks}\PY{p}{(}\PY{p}{[}\PY{p}{]}\PY{p}{)}
%% \PY{n}{plt}\PY{o}{.}\PY{n}{show}\PY{p}{(}\PY{p}{)}
%% \end{Verbatim}
%% \end{tcolorbox}

    \begin{center}
    \adjustimage{max size={0.9\linewidth}{0.9\paperheight}}{output_21_0.png}
    \end{center}
    { \hspace*{\fill} \\}
    
    Como puede observarse, este filtro es muy efectivo para eliminar las anomalías, si el tamaño de la máscara comienza a crecer entonces la imagen comienza a sobrecorregir valores.

\hypertarget{otra-implementaciuxf3n}{%
\subsubsection{Otra implementación}\label{otra-implementaciuxf3n}}

Tanto el filtro de mediana como los vistos anteriormente tienen distintas implementaciones en librerías distintas a OpenCV, por ejemplo Scikit-image. Para este trabajo tomaremos en cuenta una discusión generada a partir de un \href{https://github.com/MeteHanC/Python-Median-Filter/blob/master/MedianFilter.py}{código de Github} y simplificaremos el código con el uso de \href{https://www.python.org/search/?q=comprehensions\&submit=}{Comprehensions}, de esta forma el código será sencillo y se podrán incorporar comentarios.

    \begin{tcolorbox}[breakable, size=fbox, boxrule=1pt, pad at break*=1mm,colback=cellbackground, colframe=cellborder]
\prompt{In}{incolor}{13}{\boxspacing}
\begin{Verbatim}[commandchars=\\\{\}]
\PY{k}{def} \PY{n+nf}{median\PYZus{}filter}\PY{p}{(}\PY{n}{data}\PY{p}{,} \PY{n}{filter\PYZus{}size}\PY{p}{)}\PY{p}{:}
    \PY{l+s+sd}{\PYZdq{}\PYZdq{}\PYZdq{}Apply the median filter.}
\PY{l+s+sd}{    }
\PY{l+s+sd}{    Arguments:}
\PY{l+s+sd}{    data \PYZhy{}\PYZhy{} An array that represents an image.}
\PY{l+s+sd}{    filter\PYZus{}size \PYZhy{}\PYZhy{} An impair integer that represents mask/window size.}
\PY{l+s+sd}{    }
\PY{l+s+sd}{    Return a filtered array that represents an image.}
\PY{l+s+sd}{    \PYZdq{}\PYZdq{}\PYZdq{}}
    \PY{k}{if} \PY{p}{(}\PY{n}{filter\PYZus{}size} \PY{o}{\PYZpc{}} \PY{l+m+mi}{2}\PY{p}{)} \PY{o}{==} \PY{l+m+mi}{0}\PY{p}{:}
        \PY{k}{raise} \PY{n+ne}{TypeError}\PY{p}{(}\PY{l+s+s2}{\PYZdq{}}\PY{l+s+s2}{filter\PYZus{}size must be an impair integer}\PY{l+s+s2}{\PYZdq{}}\PY{p}{)}
    \PY{k}{if} \PY{o+ow}{not} \PY{n+nb}{isinstance}\PY{p}{(}\PY{n}{data}\PY{p}{,} \PY{n}{np}\PY{o}{.}\PY{n}{ndarray}\PY{p}{)}\PY{p}{:}
        \PY{k}{raise} \PY{n+ne}{TypeError}\PY{p}{(}\PY{l+s+s2}{\PYZdq{}}\PY{l+s+s2}{data must be an array}\PY{l+s+s2}{\PYZdq{}}\PY{p}{)}
    \PY{n}{indexer} \PY{o}{=} \PY{n}{filter\PYZus{}size} \PY{o}{/}\PY{o}{/} \PY{l+m+mi}{2}
    \PY{n}{mask} \PY{o}{=} \PY{p}{[} \PY{c+c1}{\PYZsh{} Crea un arreglo del tamaño adecuado}
        \PY{p}{(}\PY{n}{i}\PY{p}{,} \PY{n}{j}\PY{p}{)}
        \PY{k}{for} \PY{n}{i} \PY{o+ow}{in} \PY{n+nb}{range}\PY{p}{(}\PY{o}{\PYZhy{}}\PY{n}{indexer}\PY{p}{,} \PY{n}{filter\PYZus{}size}\PY{o}{\PYZhy{}}\PY{n}{indexer}\PY{p}{)}
        \PY{k}{for} \PY{n}{j} \PY{o+ow}{in} \PY{n+nb}{range}\PY{p}{(}\PY{o}{\PYZhy{}}\PY{n}{indexer}\PY{p}{,} \PY{n}{filter\PYZus{}size}\PY{o}{\PYZhy{}}\PY{n}{indexer}\PY{p}{)}
    \PY{p}{]}
    \PY{n}{index} \PY{o}{=} \PY{n+nb}{len}\PY{p}{(}\PY{n}{mask}\PY{p}{)} \PY{o}{/}\PY{o}{/} \PY{l+m+mi}{2}
    \PY{c+c1}{\PYZsh{} Recorre data}
    \PY{k}{for} \PY{n}{i} \PY{o+ow}{in} \PY{n+nb}{range}\PY{p}{(}\PY{n+nb}{len}\PY{p}{(}\PY{n}{data}\PY{p}{)}\PY{p}{)}\PY{p}{:}
        \PY{k}{for} \PY{n}{j} \PY{o+ow}{in} \PY{n+nb}{range}\PY{p}{(}\PY{n+nb}{len}\PY{p}{(}\PY{n}{data}\PY{p}{[}\PY{l+m+mi}{0}\PY{p}{]}\PY{p}{)}\PY{p}{)}\PY{p}{:}
            \PY{c+c1}{\PYZsh{} Ordena data}
            \PY{n}{data}\PY{p}{[}\PY{n}{i}\PY{p}{]}\PY{p}{[}\PY{n}{j}\PY{p}{]} \PY{o}{=} \PY{n+nb}{sorted}\PY{p}{(}
                \PY{c+c1}{\PYZsh{} Será 0 si es un límite de la imagen}
                \PY{l+m+mi}{0} \PY{k}{if} \PY{p}{(}
                    \PY{n+nb}{min}\PY{p}{(}\PY{n}{i}\PY{o}{+}\PY{n}{a}\PY{p}{,} \PY{n}{j}\PY{o}{+}\PY{n}{b}\PY{p}{)} \PY{o}{\PYZlt{}} \PY{l+m+mi}{0}
                    \PY{o+ow}{or} \PY{n+nb}{len}\PY{p}{(}\PY{n}{data}\PY{p}{)} \PY{o}{\PYZlt{}}\PY{o}{=} \PY{n}{i}\PY{o}{+}\PY{n}{a}
                    \PY{o+ow}{or} \PY{n+nb}{len}\PY{p}{(}\PY{n}{data}\PY{p}{[}\PY{l+m+mi}{0}\PY{p}{]}\PY{p}{)} \PY{o}{\PYZlt{}}\PY{o}{=} \PY{n}{j}\PY{o}{+}\PY{n}{b}
                \PY{p}{)} \PY{k}{else} \PY{n}{data}\PY{p}{[}\PY{n}{i}\PY{o}{+}\PY{n}{a}\PY{p}{]}\PY{p}{[}\PY{n}{j}\PY{o}{+}\PY{n}{b}\PY{p}{]} \PY{c+c1}{\PYZsh{} De otro modo será el valor de en la imagen}
                \PY{c+c1}{\PYZsh{} Eso lo recorre en cada punto de la máscara}
                \PY{k}{for} \PY{n}{a}\PY{p}{,} \PY{n}{b} \PY{o+ow}{in} \PY{n}{mask}
            \PY{p}{)}\PY{p}{[}\PY{n}{index}\PY{p}{]} \PY{c+c1}{\PYZsh{} Toma el valor de la mediana una vez ordenada data}
    \PY{k}{return} \PY{n}{data}
\end{Verbatim}
\end{tcolorbox}

    \begin{tcolorbox}[breakable, size=fbox, boxrule=1pt, pad at break*=1mm,colback=cellbackground, colframe=cellborder]
\prompt{In}{incolor}{14}{\boxspacing}
\begin{Verbatim}[commandchars=\\\{\}]
\PY{k+kn}{from} \PY{n+nn}{PIL} \PY{k+kn}{import} \PY{n}{Image}
\PY{c+c1}{\PYZsh{} Lectura de la imagen}
\PY{n}{im} \PY{o}{=} \PY{n}{Image}\PY{o}{.}\PY{n}{open}\PY{p}{(}\PY{l+s+s1}{\PYZsq{}}\PY{l+s+s1}{im/salt\PYZhy{}pepper\PYZhy{}1.png}\PY{l+s+s1}{\PYZsq{}}\PY{p}{)}
\PY{c+c1}{\PYZsh{} Convierte la imagen a un arreglo}
\PY{n+nb}{input} \PY{o}{=} \PY{n}{np}\PY{o}{.}\PY{n}{asarray}\PY{p}{(}\PY{n}{im}\PY{p}{)}
\PY{n}{out} \PY{o}{=} \PY{n}{median\PYZus{}filter}\PY{p}{(}\PY{n+nb}{input}\PY{p}{,} \PY{l+m+mi}{3}\PY{p}{)}
\PY{n}{Image}\PY{o}{.}\PY{n}{fromarray}\PY{p}{(}\PY{n}{out}\PY{p}{)}\PY{o}{.}\PY{n}{save}\PY{p}{(}\PY{l+s+s2}{\PYZdq{}}\PY{l+s+s2}{out/median\PYZhy{}alg.jpg}\PY{l+s+s2}{\PYZdq{}}\PY{p}{)}
\end{Verbatim}
\end{tcolorbox}

%%     \begin{tcolorbox}[breakable, size=fbox, boxrule=1pt, pad at break*=1mm,colback=cellbackground, colframe=cellborder]
%% \prompt{In}{incolor}{15}{\boxspacing}
%% \begin{Verbatim}[commandchars=\\\{\}]
%% \PY{n}{plt}\PY{o}{.}\PY{n}{subplot}\PY{p}{(}\PY{l+m+mi}{1}\PY{p}{,}\PY{l+m+mi}{2}\PY{p}{,}\PY{l+m+mi}{1}\PY{p}{)}
%% \PY{n}{plt}\PY{o}{.}\PY{n}{imshow}\PY{p}{(}\PY{n}{im}\PY{p}{)}
%% \PY{n}{plt}\PY{o}{.}\PY{n}{title}\PY{p}{(}\PY{l+s+s1}{\PYZsq{}}\PY{l+s+s1}{Imagen Original}\PY{l+s+s1}{\PYZsq{}}\PY{p}{)}\PY{p}{,} \PY{n}{plt}\PY{o}{.}\PY{n}{xticks}\PY{p}{(}\PY{p}{[}\PY{p}{]}\PY{p}{)}\PY{p}{,} \PY{n}{plt}\PY{o}{.}\PY{n}{yticks}\PY{p}{(}\PY{p}{[}\PY{p}{]}\PY{p}{)}
%% \PY{n}{plt}\PY{o}{.}\PY{n}{subplot}\PY{p}{(}\PY{l+m+mi}{1}\PY{p}{,}\PY{l+m+mi}{2}\PY{p}{,}\PY{l+m+mi}{2}\PY{p}{)} 
%% \PY{n}{plt}\PY{o}{.}\PY{n}{imshow}\PY{p}{(}\PY{n}{out}\PY{p}{)}
%% \PY{n}{plt}\PY{o}{.}\PY{n}{title}\PY{p}{(}\PY{l+s+s1}{\PYZsq{}}\PY{l+s+s1}{Filtro de mediana}\PY{l+s+s1}{\PYZsq{}}\PY{p}{)}\PY{p}{,} \PY{n}{plt}\PY{o}{.}\PY{n}{xticks}\PY{p}{(}\PY{p}{[}\PY{p}{]}\PY{p}{)}\PY{p}{,} \PY{n}{plt}\PY{o}{.}\PY{n}{yticks}\PY{p}{(}\PY{p}{[}\PY{p}{]}\PY{p}{)}
%% \PY{n}{plt}\PY{o}{.}\PY{n}{show}\PY{p}{(}\PY{p}{)}
%% \end{Verbatim}
%% \end{tcolorbox}

    \begin{center}
    \adjustimage{max size={0.9\linewidth}{0.9\paperheight}}{output_25_0.png}
    \end{center}
    { \hspace*{\fill} \\}
    
    Es posible observar que el algoritmo implementado manualmente es más agresivo, con una máscara más pequeña ($3\times 3$) se obtiene un resultado similar al de OpenCV con una máscara más grande ($5\times 5$). Sin embargo, el algoritmo de OpenCV no deja tantas anomalías evidentes, a diferencia de la implementación manual, donde aún se observan algunas anomalías sal y pimienta.


    % Add a bibliography block to the postdoc
    \bibliographystyle{apalike}
    \bibliography{main}
    
    
\end{document}
