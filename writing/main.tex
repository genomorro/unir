\documentclass[12pt,a4paper,table]{article}

    \usepackage[breakable]{tcolorbox}
    \usepackage{parskip} % Stop auto-indenting (to mimic markdown behaviour)
    
    \usepackage{iftex}
    \ifPDFTeX
    	\usepackage[T1]{fontenc}
    	\usepackage{mathpazo}
    \else
    	\usepackage{fontspec}
    \fi

    % Basic figure setup, for now with no caption control since it's done
    % automatically by Pandoc (which extracts ![](path) syntax from Markdown).
    \usepackage{graphicx}
    % Maintain compatibility with old templates. Remove in nbconvert 6.0
    \let\Oldincludegraphics\includegraphics
    % Ensure that by default, figures have no caption (until we provide a
    % proper Figure object with a Caption API and a way to capture that
    % in the conversion process - todo).
    \usepackage{caption}
    \DeclareCaptionFormat{nocaption}{}
    \captionsetup{format=nocaption,aboveskip=0pt,belowskip=0pt}

    \usepackage{float}
    \floatplacement{figure}{H} % forces figures to be placed at the correct location
    \usepackage{xcolor} % Allow colors to be defined
    \usepackage{enumerate} % Needed for markdown enumerations to work
    \usepackage{geometry} % Used to adjust the document margins
    \usepackage{amsmath} % Equations
    \usepackage{amssymb} % Equations
    \usepackage{textcomp} % defines textquotesingle
    % Hack from http://tex.stackexchange.com/a/47451/13684:
    \AtBeginDocument{%
        \def\PYZsq{\textquotesingle}% Upright quotes in Pygmentized code
    }
    \usepackage{upquote} % Upright quotes for verbatim code
    \usepackage{eurosym} % defines \euro
    \usepackage[mathletters]{ucs} % Extended unicode (utf-8) support
    \usepackage{fancyvrb} % verbatim replacement that allows latex
    \usepackage{grffile} % extends the file name processing of package graphics 
                         % to support a larger range
    \makeatletter % fix for old versions of grffile with XeLaTeX
    \@ifpackagelater{grffile}{2019/11/01}
    {
      % Do nothing on new versions
    }
    {
      \def\Gread@@xetex#1{%
        \IfFileExists{"\Gin@base".bb}%
        {\Gread@eps{\Gin@base.bb}}%
        {\Gread@@xetex@aux#1}%
      }
    }
    \makeatother
    \usepackage[Export]{adjustbox} % Used to constrain images to a maximum size
    \adjustboxset{max size={0.9\linewidth}{0.9\paperheight}}

    % The hyperref package gives us a pdf with properly built
    % internal navigation ('pdf bookmarks' for the table of contents,
    % internal cross-reference links, web links for URLs, etc.)
    \usepackage{hyperref}
    % The default LaTeX title has an obnoxious amount of whitespace. By default,
    % titling removes some of it. It also provides customization options.
    \usepackage{titling}
    \usepackage{longtable} % longtable support required by pandoc >1.10
    \usepackage{booktabs}  % table support for pandoc > 1.12.2
    \usepackage[inline]{enumitem} % IRkernel/repr support (it uses the enumerate* environment)
    \usepackage[normalem]{ulem} % ulem is needed to support strikethroughs (\sout)
                                % normalem makes italics be italics, not underlines
    \usepackage{mathrsfs}
    

    
    % Colors for the hyperref package
    \definecolor{urlcolor}{rgb}{0,.145,.698}
    \definecolor{linkcolor}{rgb}{.71,0.21,0.01}
    \definecolor{citecolor}{rgb}{.12,.54,.11}

    % ANSI colors
    \definecolor{ansi-black}{HTML}{3E424D}
    \definecolor{ansi-black-intense}{HTML}{282C36}
    \definecolor{ansi-red}{HTML}{E75C58}
    \definecolor{ansi-red-intense}{HTML}{B22B31}
    \definecolor{ansi-green}{HTML}{00A250}
    \definecolor{ansi-green-intense}{HTML}{007427}
    \definecolor{ansi-yellow}{HTML}{DDB62B}
    \definecolor{ansi-yellow-intense}{HTML}{B27D12}
    \definecolor{ansi-blue}{HTML}{208FFB}
    \definecolor{ansi-blue-intense}{HTML}{0065CA}
    \definecolor{ansi-magenta}{HTML}{D160C4}
    \definecolor{ansi-magenta-intense}{HTML}{A03196}
    \definecolor{ansi-cyan}{HTML}{60C6C8}
    \definecolor{ansi-cyan-intense}{HTML}{258F8F}
    \definecolor{ansi-white}{HTML}{C5C1B4}
    \definecolor{ansi-white-intense}{HTML}{A1A6B2}
    \definecolor{ansi-default-inverse-fg}{HTML}{FFFFFF}
    \definecolor{ansi-default-inverse-bg}{HTML}{000000}

    % common color for the border for error outputs.
    \definecolor{outerrorbackground}{HTML}{FFDFDF}

    % commands and environments needed by pandoc snippets
    % extracted from the output of `pandoc -s`
    \providecommand{\tightlist}{%
      \setlength{\itemsep}{0pt}\setlength{\parskip}{0pt}}
    \DefineVerbatimEnvironment{Highlighting}{Verbatim}{commandchars=\\\{\}}
    % Add ',fontsize=\small' for more characters per line
    \newenvironment{Shaded}{}{}
    \newcommand{\KeywordTok}[1]{\textcolor[rgb]{0.00,0.44,0.13}{\textbf{{#1}}}}
    \newcommand{\DataTypeTok}[1]{\textcolor[rgb]{0.56,0.13,0.00}{{#1}}}
    \newcommand{\DecValTok}[1]{\textcolor[rgb]{0.25,0.63,0.44}{{#1}}}
    \newcommand{\BaseNTok}[1]{\textcolor[rgb]{0.25,0.63,0.44}{{#1}}}
    \newcommand{\FloatTok}[1]{\textcolor[rgb]{0.25,0.63,0.44}{{#1}}}
    \newcommand{\CharTok}[1]{\textcolor[rgb]{0.25,0.44,0.63}{{#1}}}
    \newcommand{\StringTok}[1]{\textcolor[rgb]{0.25,0.44,0.63}{{#1}}}
    \newcommand{\CommentTok}[1]{\textcolor[rgb]{0.38,0.63,0.69}{\textit{{#1}}}}
    \newcommand{\OtherTok}[1]{\textcolor[rgb]{0.00,0.44,0.13}{{#1}}}
    \newcommand{\AlertTok}[1]{\textcolor[rgb]{1.00,0.00,0.00}{\textbf{{#1}}}}
    \newcommand{\FunctionTok}[1]{\textcolor[rgb]{0.02,0.16,0.49}{{#1}}}
    \newcommand{\RegionMarkerTok}[1]{{#1}}
    \newcommand{\ErrorTok}[1]{\textcolor[rgb]{1.00,0.00,0.00}{\textbf{{#1}}}}
    \newcommand{\NormalTok}[1]{{#1}}
    
    % Additional commands for more recent versions of Pandoc
    \newcommand{\ConstantTok}[1]{\textcolor[rgb]{0.53,0.00,0.00}{{#1}}}
    \newcommand{\SpecialCharTok}[1]{\textcolor[rgb]{0.25,0.44,0.63}{{#1}}}
    \newcommand{\VerbatimStringTok}[1]{\textcolor[rgb]{0.25,0.44,0.63}{{#1}}}
    \newcommand{\SpecialStringTok}[1]{\textcolor[rgb]{0.73,0.40,0.53}{{#1}}}
    \newcommand{\ImportTok}[1]{{#1}}
    \newcommand{\DocumentationTok}[1]{\textcolor[rgb]{0.73,0.13,0.13}{\textit{{#1}}}}
    \newcommand{\AnnotationTok}[1]{\textcolor[rgb]{0.38,0.63,0.69}{\textbf{\textit{{#1}}}}}
    \newcommand{\CommentVarTok}[1]{\textcolor[rgb]{0.38,0.63,0.69}{\textbf{\textit{{#1}}}}}
    \newcommand{\VariableTok}[1]{\textcolor[rgb]{0.10,0.09,0.49}{{#1}}}
    \newcommand{\ControlFlowTok}[1]{\textcolor[rgb]{0.00,0.44,0.13}{\textbf{{#1}}}}
    \newcommand{\OperatorTok}[1]{\textcolor[rgb]{0.40,0.40,0.40}{{#1}}}
    \newcommand{\BuiltInTok}[1]{{#1}}
    \newcommand{\ExtensionTok}[1]{{#1}}
    \newcommand{\PreprocessorTok}[1]{\textcolor[rgb]{0.74,0.48,0.00}{{#1}}}
    \newcommand{\AttributeTok}[1]{\textcolor[rgb]{0.49,0.56,0.16}{{#1}}}
    \newcommand{\InformationTok}[1]{\textcolor[rgb]{0.38,0.63,0.69}{\textbf{\textit{{#1}}}}}
    \newcommand{\WarningTok}[1]{\textcolor[rgb]{0.38,0.63,0.69}{\textbf{\textit{{#1}}}}}
    
    
    % Define a nice break command that doesn't care if a line doesn't already
    % exist.
    \def\br{\hspace*{\fill} \\* }
    % Math Jax compatibility definitions
    \def\gt{>}
    \def\lt{<}
    \let\Oldtex\TeX
    \let\Oldlatex\LaTeX
    \renewcommand{\TeX}{\textrm{\Oldtex}}
    \renewcommand{\LaTeX}{\textrm{\Oldlatex}}
    % Document parameters
    % Document title
    \title{Análisis: Mobile price classification con SVM y Redes neuronales}
    
    
    
    
    
% Pygments definitions
\makeatletter
\def\PY@reset{\let\PY@it=\relax \let\PY@bf=\relax%
    \let\PY@ul=\relax \let\PY@tc=\relax%
    \let\PY@bc=\relax \let\PY@ff=\relax}
\def\PY@tok#1{\csname PY@tok@#1\endcsname}
\def\PY@toks#1+{\ifx\relax#1\empty\else%
    \PY@tok{#1}\expandafter\PY@toks\fi}
\def\PY@do#1{\PY@bc{\PY@tc{\PY@ul{%
    \PY@it{\PY@bf{\PY@ff{#1}}}}}}}
\def\PY#1#2{\PY@reset\PY@toks#1+\relax+\PY@do{#2}}

\@namedef{PY@tok@w}{\def\PY@tc##1{\textcolor[rgb]{0.73,0.73,0.73}{##1}}}
\@namedef{PY@tok@c}{\let\PY@it=\textit\def\PY@tc##1{\textcolor[rgb]{0.24,0.48,0.48}{##1}}}
\@namedef{PY@tok@cp}{\def\PY@tc##1{\textcolor[rgb]{0.61,0.40,0.00}{##1}}}
\@namedef{PY@tok@k}{\let\PY@bf=\textbf\def\PY@tc##1{\textcolor[rgb]{0.00,0.50,0.00}{##1}}}
\@namedef{PY@tok@kp}{\def\PY@tc##1{\textcolor[rgb]{0.00,0.50,0.00}{##1}}}
\@namedef{PY@tok@kt}{\def\PY@tc##1{\textcolor[rgb]{0.69,0.00,0.25}{##1}}}
\@namedef{PY@tok@o}{\def\PY@tc##1{\textcolor[rgb]{0.40,0.40,0.40}{##1}}}
\@namedef{PY@tok@ow}{\let\PY@bf=\textbf\def\PY@tc##1{\textcolor[rgb]{0.67,0.13,1.00}{##1}}}
\@namedef{PY@tok@nb}{\def\PY@tc##1{\textcolor[rgb]{0.00,0.50,0.00}{##1}}}
\@namedef{PY@tok@nf}{\def\PY@tc##1{\textcolor[rgb]{0.00,0.00,1.00}{##1}}}
\@namedef{PY@tok@nc}{\let\PY@bf=\textbf\def\PY@tc##1{\textcolor[rgb]{0.00,0.00,1.00}{##1}}}
\@namedef{PY@tok@nn}{\let\PY@bf=\textbf\def\PY@tc##1{\textcolor[rgb]{0.00,0.00,1.00}{##1}}}
\@namedef{PY@tok@ne}{\let\PY@bf=\textbf\def\PY@tc##1{\textcolor[rgb]{0.80,0.25,0.22}{##1}}}
\@namedef{PY@tok@nv}{\def\PY@tc##1{\textcolor[rgb]{0.10,0.09,0.49}{##1}}}
\@namedef{PY@tok@no}{\def\PY@tc##1{\textcolor[rgb]{0.53,0.00,0.00}{##1}}}
\@namedef{PY@tok@nl}{\def\PY@tc##1{\textcolor[rgb]{0.46,0.46,0.00}{##1}}}
\@namedef{PY@tok@ni}{\let\PY@bf=\textbf\def\PY@tc##1{\textcolor[rgb]{0.44,0.44,0.44}{##1}}}
\@namedef{PY@tok@na}{\def\PY@tc##1{\textcolor[rgb]{0.41,0.47,0.13}{##1}}}
\@namedef{PY@tok@nt}{\let\PY@bf=\textbf\def\PY@tc##1{\textcolor[rgb]{0.00,0.50,0.00}{##1}}}
\@namedef{PY@tok@nd}{\def\PY@tc##1{\textcolor[rgb]{0.67,0.13,1.00}{##1}}}
\@namedef{PY@tok@s}{\def\PY@tc##1{\textcolor[rgb]{0.73,0.13,0.13}{##1}}}
\@namedef{PY@tok@sd}{\let\PY@it=\textit\def\PY@tc##1{\textcolor[rgb]{0.73,0.13,0.13}{##1}}}
\@namedef{PY@tok@si}{\let\PY@bf=\textbf\def\PY@tc##1{\textcolor[rgb]{0.64,0.35,0.47}{##1}}}
\@namedef{PY@tok@se}{\let\PY@bf=\textbf\def\PY@tc##1{\textcolor[rgb]{0.67,0.36,0.12}{##1}}}
\@namedef{PY@tok@sr}{\def\PY@tc##1{\textcolor[rgb]{0.64,0.35,0.47}{##1}}}
\@namedef{PY@tok@ss}{\def\PY@tc##1{\textcolor[rgb]{0.10,0.09,0.49}{##1}}}
\@namedef{PY@tok@sx}{\def\PY@tc##1{\textcolor[rgb]{0.00,0.50,0.00}{##1}}}
\@namedef{PY@tok@m}{\def\PY@tc##1{\textcolor[rgb]{0.40,0.40,0.40}{##1}}}
\@namedef{PY@tok@gh}{\let\PY@bf=\textbf\def\PY@tc##1{\textcolor[rgb]{0.00,0.00,0.50}{##1}}}
\@namedef{PY@tok@gu}{\let\PY@bf=\textbf\def\PY@tc##1{\textcolor[rgb]{0.50,0.00,0.50}{##1}}}
\@namedef{PY@tok@gd}{\def\PY@tc##1{\textcolor[rgb]{0.63,0.00,0.00}{##1}}}
\@namedef{PY@tok@gi}{\def\PY@tc##1{\textcolor[rgb]{0.00,0.52,0.00}{##1}}}
\@namedef{PY@tok@gr}{\def\PY@tc##1{\textcolor[rgb]{0.89,0.00,0.00}{##1}}}
\@namedef{PY@tok@ge}{\let\PY@it=\textit}
\@namedef{PY@tok@gs}{\let\PY@bf=\textbf}
\@namedef{PY@tok@gp}{\let\PY@bf=\textbf\def\PY@tc##1{\textcolor[rgb]{0.00,0.00,0.50}{##1}}}
\@namedef{PY@tok@go}{\def\PY@tc##1{\textcolor[rgb]{0.44,0.44,0.44}{##1}}}
\@namedef{PY@tok@gt}{\def\PY@tc##1{\textcolor[rgb]{0.00,0.27,0.87}{##1}}}
\@namedef{PY@tok@err}{\def\PY@bc##1{{\setlength{\fboxsep}{\string -\fboxrule}\fcolorbox[rgb]{1.00,0.00,0.00}{1,1,1}{\strut ##1}}}}
\@namedef{PY@tok@kc}{\let\PY@bf=\textbf\def\PY@tc##1{\textcolor[rgb]{0.00,0.50,0.00}{##1}}}
\@namedef{PY@tok@kd}{\let\PY@bf=\textbf\def\PY@tc##1{\textcolor[rgb]{0.00,0.50,0.00}{##1}}}
\@namedef{PY@tok@kn}{\let\PY@bf=\textbf\def\PY@tc##1{\textcolor[rgb]{0.00,0.50,0.00}{##1}}}
\@namedef{PY@tok@kr}{\let\PY@bf=\textbf\def\PY@tc##1{\textcolor[rgb]{0.00,0.50,0.00}{##1}}}
\@namedef{PY@tok@bp}{\def\PY@tc##1{\textcolor[rgb]{0.00,0.50,0.00}{##1}}}
\@namedef{PY@tok@fm}{\def\PY@tc##1{\textcolor[rgb]{0.00,0.00,1.00}{##1}}}
\@namedef{PY@tok@vc}{\def\PY@tc##1{\textcolor[rgb]{0.10,0.09,0.49}{##1}}}
\@namedef{PY@tok@vg}{\def\PY@tc##1{\textcolor[rgb]{0.10,0.09,0.49}{##1}}}
\@namedef{PY@tok@vi}{\def\PY@tc##1{\textcolor[rgb]{0.10,0.09,0.49}{##1}}}
\@namedef{PY@tok@vm}{\def\PY@tc##1{\textcolor[rgb]{0.10,0.09,0.49}{##1}}}
\@namedef{PY@tok@sa}{\def\PY@tc##1{\textcolor[rgb]{0.73,0.13,0.13}{##1}}}
\@namedef{PY@tok@sb}{\def\PY@tc##1{\textcolor[rgb]{0.73,0.13,0.13}{##1}}}
\@namedef{PY@tok@sc}{\def\PY@tc##1{\textcolor[rgb]{0.73,0.13,0.13}{##1}}}
\@namedef{PY@tok@dl}{\def\PY@tc##1{\textcolor[rgb]{0.73,0.13,0.13}{##1}}}
\@namedef{PY@tok@s2}{\def\PY@tc##1{\textcolor[rgb]{0.73,0.13,0.13}{##1}}}
\@namedef{PY@tok@sh}{\def\PY@tc##1{\textcolor[rgb]{0.73,0.13,0.13}{##1}}}
\@namedef{PY@tok@s1}{\def\PY@tc##1{\textcolor[rgb]{0.73,0.13,0.13}{##1}}}
\@namedef{PY@tok@mb}{\def\PY@tc##1{\textcolor[rgb]{0.40,0.40,0.40}{##1}}}
\@namedef{PY@tok@mf}{\def\PY@tc##1{\textcolor[rgb]{0.40,0.40,0.40}{##1}}}
\@namedef{PY@tok@mh}{\def\PY@tc##1{\textcolor[rgb]{0.40,0.40,0.40}{##1}}}
\@namedef{PY@tok@mi}{\def\PY@tc##1{\textcolor[rgb]{0.40,0.40,0.40}{##1}}}
\@namedef{PY@tok@il}{\def\PY@tc##1{\textcolor[rgb]{0.40,0.40,0.40}{##1}}}
\@namedef{PY@tok@mo}{\def\PY@tc##1{\textcolor[rgb]{0.40,0.40,0.40}{##1}}}
\@namedef{PY@tok@ch}{\let\PY@it=\textit\def\PY@tc##1{\textcolor[rgb]{0.24,0.48,0.48}{##1}}}
\@namedef{PY@tok@cm}{\let\PY@it=\textit\def\PY@tc##1{\textcolor[rgb]{0.24,0.48,0.48}{##1}}}
\@namedef{PY@tok@cpf}{\let\PY@it=\textit\def\PY@tc##1{\textcolor[rgb]{0.24,0.48,0.48}{##1}}}
\@namedef{PY@tok@c1}{\let\PY@it=\textit\def\PY@tc##1{\textcolor[rgb]{0.24,0.48,0.48}{##1}}}
\@namedef{PY@tok@cs}{\let\PY@it=\textit\def\PY@tc##1{\textcolor[rgb]{0.24,0.48,0.48}{##1}}}

\def\PYZbs{\char`\\}
\def\PYZus{\char`\_}
\def\PYZob{\char`\{}
\def\PYZcb{\char`\}}
\def\PYZca{\char`\^}
\def\PYZam{\char`\&}
\def\PYZlt{\char`\<}
\def\PYZgt{\char`\>}
\def\PYZsh{\char`\#}
\def\PYZpc{\char`\%}
\def\PYZdl{\char`\$}
\def\PYZhy{\char`\-}
\def\PYZsq{\char`\'}
\def\PYZdq{\char`\"}
\def\PYZti{\char`\~}
% for compatibility with earlier versions
\def\PYZat{@}
\def\PYZlb{[}
\def\PYZrb{]}
\makeatother


    % For linebreaks inside Verbatim environment from package fancyvrb. 
    \makeatletter
        \newbox\Wrappedcontinuationbox 
        \newbox\Wrappedvisiblespacebox 
        \newcommand*\Wrappedvisiblespace {\textcolor{red}{\textvisiblespace}} 
        \newcommand*\Wrappedcontinuationsymbol {\textcolor{red}{\llap{\tiny$\m@th\hookrightarrow$}}} 
        \newcommand*\Wrappedcontinuationindent {3ex } 
        \newcommand*\Wrappedafterbreak {\kern\Wrappedcontinuationindent\copy\Wrappedcontinuationbox} 
        % Take advantage of the already applied Pygments mark-up to insert 
        % potential linebreaks for TeX processing. 
        %        {, <, #, %, $, ' and ": go to next line. 
        %        _, }, ^, &, >, - and ~: stay at end of broken line. 
        % Use of \textquotesingle for straight quote. 
        \newcommand*\Wrappedbreaksatspecials {% 
            \def\PYGZus{\discretionary{\char`\_}{\Wrappedafterbreak}{\char`\_}}% 
            \def\PYGZob{\discretionary{}{\Wrappedafterbreak\char`\{}{\char`\{}}% 
            \def\PYGZcb{\discretionary{\char`\}}{\Wrappedafterbreak}{\char`\}}}% 
            \def\PYGZca{\discretionary{\char`\^}{\Wrappedafterbreak}{\char`\^}}% 
            \def\PYGZam{\discretionary{\char`\&}{\Wrappedafterbreak}{\char`\&}}% 
            \def\PYGZlt{\discretionary{}{\Wrappedafterbreak\char`\<}{\char`\<}}% 
            \def\PYGZgt{\discretionary{\char`\>}{\Wrappedafterbreak}{\char`\>}}% 
            \def\PYGZsh{\discretionary{}{\Wrappedafterbreak\char`\#}{\char`\#}}% 
            \def\PYGZpc{\discretionary{}{\Wrappedafterbreak\char`\%}{\char`\%}}% 
            \def\PYGZdl{\discretionary{}{\Wrappedafterbreak\char`\$}{\char`\$}}% 
            \def\PYGZhy{\discretionary{\char`\-}{\Wrappedafterbreak}{\char`\-}}% 
            \def\PYGZsq{\discretionary{}{\Wrappedafterbreak\textquotesingle}{\textquotesingle}}% 
            \def\PYGZdq{\discretionary{}{\Wrappedafterbreak\char`\"}{\char`\"}}% 
            \def\PYGZti{\discretionary{\char`\~}{\Wrappedafterbreak}{\char`\~}}% 
        } 
        % Some characters . , ; ? ! / are not pygmentized. 
        % This macro makes them "active" and they will insert potential linebreaks 
        \newcommand*\Wrappedbreaksatpunct {% 
            \lccode`\~`\.\lowercase{\def~}{\discretionary{\hbox{\char`\.}}{\Wrappedafterbreak}{\hbox{\char`\.}}}% 
            \lccode`\~`\,\lowercase{\def~}{\discretionary{\hbox{\char`\,}}{\Wrappedafterbreak}{\hbox{\char`\,}}}% 
            \lccode`\~`\;\lowercase{\def~}{\discretionary{\hbox{\char`\;}}{\Wrappedafterbreak}{\hbox{\char`\;}}}% 
            \lccode`\~`\:\lowercase{\def~}{\discretionary{\hbox{\char`\:}}{\Wrappedafterbreak}{\hbox{\char`\:}}}% 
            \lccode`\~`\?\lowercase{\def~}{\discretionary{\hbox{\char`\?}}{\Wrappedafterbreak}{\hbox{\char`\?}}}% 
            \lccode`\~`\!\lowercase{\def~}{\discretionary{\hbox{\char`\!}}{\Wrappedafterbreak}{\hbox{\char`\!}}}% 
            \lccode`\~`\/\lowercase{\def~}{\discretionary{\hbox{\char`\/}}{\Wrappedafterbreak}{\hbox{\char`\/}}}% 
            \catcode`\.\active
            \catcode`\,\active 
            \catcode`\;\active
            \catcode`\:\active
            \catcode`\?\active
            \catcode`\!\active
            \catcode`\/\active 
            \lccode`\~`\~ 	
        }
    \makeatother

    \let\OriginalVerbatim=\Verbatim
    \makeatletter
    \renewcommand{\Verbatim}[1][1]{%
        %\parskip\z@skip
        \sbox\Wrappedcontinuationbox {\Wrappedcontinuationsymbol}%
        \sbox\Wrappedvisiblespacebox {\FV@SetupFont\Wrappedvisiblespace}%
        \def\FancyVerbFormatLine ##1{\hsize\linewidth
            \vtop{\raggedright\hyphenpenalty\z@\exhyphenpenalty\z@
                \doublehyphendemerits\z@\finalhyphendemerits\z@
                \strut ##1\strut}%
        }%
        % If the linebreak is at a space, the latter will be displayed as visible
        % space at end of first line, and a continuation symbol starts next line.
        % Stretch/shrink are however usually zero for typewriter font.
        \def\FV@Space {%
            \nobreak\hskip\z@ plus\fontdimen3\font minus\fontdimen4\font
            \discretionary{\copy\Wrappedvisiblespacebox}{\Wrappedafterbreak}
            {\kern\fontdimen2\font}%
        }%
        
        % Allow breaks at special characters using \PYG... macros.
        \Wrappedbreaksatspecials
        % Breaks at punctuation characters . , ; ? ! and / need catcode=\active 	
        \OriginalVerbatim[#1,codes*=\Wrappedbreaksatpunct]%
    }
    \makeatother

    % Exact colors from NB
    \definecolor{incolor}{HTML}{303F9F}
    \definecolor{outcolor}{HTML}{D84315}
    \definecolor{cellborder}{HTML}{CFCFCF}
    \definecolor{cellbackground}{HTML}{F7F7F7}
    
    % prompt
    \makeatletter
    \newcommand{\boxspacing}{\kern\kvtcb@left@rule\kern\kvtcb@boxsep}
    \makeatother
    \newcommand{\prompt}[4]{
        {\ttfamily\llap{{\color{#2}[#3]:\hspace{3pt}#4}}\vspace{-\baselineskip}}
    }
    

    
    % Prevent overflowing lines due to hard-to-break entities
    \sloppy 
    % Setup hyperref package
    \hypersetup{
      breaklinks=true,  % so long urls are correctly broken across lines
      colorlinks=true,
      urlcolor=urlcolor,
      linkcolor=linkcolor,
      citecolor=citecolor,
      }
    % Slightly bigger margins than the latex defaults
    
    \geometry{verbose,tmargin=1in,bmargin=1in,lmargin=1in,rmargin=1in}
    
    %% BEGIN: UNIR
    \usepackage[spanish,mexico]{babel}
    \usepackage[sfdefault,lf]{carlito}
    \makeatletter
    \let\newtitle\@title
    \makeatother
    \usepackage{amsmath}
    \usepackage{multirow}
    \definecolor{UnirLight}{HTML}{E6F4F9}
    \definecolor{UnirDark}{HTML}{0098CD}
    \arrayrulecolor{UnirDark}
    \usepackage{titlesec}
    \titleformat*{\section}{\color{UnirDark}\normalsize\bfseries}
    \titleformat*{\subsection}{\color{UnirDark}\normalsize\bfseries}
    \titleformat*{\subsubsection}{\color{UnirDark}\normalsize\bfseries}
    \usepackage{fancyhdr}
    \pagestyle{fancy}
    \renewcommand{\headrulewidth}{0pt}
    \headheight=45pt
    \setlength{\footskip}{64pt}
    \lhead{}
    \chead{
      \begin{tabular}{|c|l|c|}
        \hline
        \rowcolor{UnirLight}
        \textcolor{UnirDark}{Asignatura} & \textcolor{UnirDark}{Datos del alumno} & \textcolor{UnirDark}{Fecha} \\
        \hline
        \multirow{2}{12em}{\textbf{Aprendizaje automático}} & Apellidos: Domínguez Espinoza & \multirow{2}{6em}{6 de mayo de 2022} \\
        & Nombre: Edgar Uriel & \\
        \hline
    \end{tabular}}
    \rhead{}
    \lfoot{}
    \cfoot{}
    \rfoot{\makebox(70,56)[t]{\textcolor{UnirDark}{Actividades}}
      \colorbox{UnirDark}{
        \makebox(10,56)[t]{
          \textcolor{white}{\thepage}}}}
    \usepackage[color={[gray]{0.5}}, angle=90,fontsize=9pt,anchor=lb,pos={0.03\paperwidth,0.95\paperheight}]{draftwatermark}
    \SetWatermarkText{{\copyright} Universidad Internacional de La Rioja en México (UNIR)}
    \hypersetup{
      pdfauthor={Edgar Uriel Domínguez Espinoza},
      pdftitle={Análisis: Mobile price classification con SVM y Redes neuronales},
      pdfkeywords={aprendizaje automático, svm, redes neuronales, clasificación, tensorflow},
      pdfsubject={Aprendizaje automático},
      pdfcreator={Emacs 27.2}, 
      pdflang={Spanish}}
    \usepackage[round]{natbib}
    %% END: UNIR
 

\begin{document}
    
    
    

    
    \hypertarget{actividad-clasificaciuxf3n-con-muxe1quina-de-vectores-de-soporte-y-redes-de-neuronas-mobile-price-classification}{%
\textcolor{UnirDark}{\Large\bfseries\newtitle}\label{actividad-clasificaciuxf3n-con-muxe1quina-de-vectores-de-soporte-y-redes-de-neuronas-mobile-price-classification}}

\hypertarget{introducciuxf3n}{%
\section{Introducción}\label{introducciuxf3n}}

El problema original tiene una breve descripción contextual (.sic):

\begin{quote}
Bob has started his own mobile company. He wants to give tough fight to
big companies like Apple,Samsung etc.

He does not know how to estimate price of mobiles his company creates.
In this competitive mobile phone market you cannot simply assume things.
To solve this problem he collects sales data of mobile phones of various
companies.

Bob wants to find out some relation between features of a mobile
phone(eg:- RAM,Internal Memory etc) and its selling price. But he is not
so good at Machine Learning. So he needs your help to solve this
problem.

In this problem you do not have to predict actual price but a price
range indicating how high the price is. \citep{Sharma2018}
\end{quote}

La variable objetivo es la variable ``price\_range''. En este análisis
no se usarán los dos datasets, solo
\href{https://www.kaggle.com/iabhishekofficial/mobile-price-classification\#train.csv}{train.csv}
que corresponde a los datos de entrenamiento.

\hypertarget{bibliotecas-a-utilizar}{%
\subsection{Bibliotecas a utilizar}\label{bibliotecas-a-utilizar}}

En el presente análisis se comparará el funcionamiento de una SVM y una
red neuronal básica por lo tanto se requieren las siguientes
bibliotecas.

    \begin{tcolorbox}[breakable, size=fbox, boxrule=1pt, pad at break*=1mm,colback=cellbackground, colframe=cellborder]
\prompt{In}{incolor}{1}{\boxspacing}
\begin{Verbatim}[commandchars=\\\{\}]
\PY{k+kn}{from} \PY{n+nn}{sklearn} \PY{k+kn}{import} \PY{n}{metrics}
\PY{k+kn}{from} \PY{n+nn}{sklearn} \PY{k+kn}{import} \PY{n}{svm}
\PY{k+kn}{from} \PY{n+nn}{sklearn}\PY{n+nn}{.}\PY{n+nn}{model\PYZus{}selection} \PY{k+kn}{import} \PY{n}{train\PYZus{}test\PYZus{}split}\PY{p}{,} \PY{n}{RandomizedSearchCV}
\PY{k+kn}{from} \PY{n+nn}{tensorflow}\PY{n+nn}{.}\PY{n+nn}{keras}\PY{n+nn}{.}\PY{n+nn}{layers} \PY{k+kn}{import} \PY{n}{Dense}
\PY{k+kn}{from} \PY{n+nn}{tensorflow}\PY{n+nn}{.}\PY{n+nn}{keras}\PY{n+nn}{.}\PY{n+nn}{models} \PY{k+kn}{import} \PY{n}{Sequential}
\PY{k+kn}{import} \PY{n+nn}{matplotlib}\PY{n+nn}{.}\PY{n+nn}{pyplot} \PY{k}{as} \PY{n+nn}{plt}
\PY{k+kn}{import} \PY{n+nn}{numpy} \PY{k}{as} \PY{n+nn}{np}
\PY{k+kn}{import} \PY{n+nn}{pandas} \PY{k}{as} \PY{n+nn}{pd}
\PY{k+kn}{import} \PY{n+nn}{seaborn} \PY{k}{as} \PY{n+nn}{sns}
\end{Verbatim}
\end{tcolorbox}

    \hypertarget{carga-de-dataset-y-anuxe1lisis-descriptivo-de-datos}{%
\section{Carga de dataset y análisis descriptivo de
datos}\label{carga-de-dataset-y-anuxe1lisis-descriptivo-de-datos}}

    \begin{tcolorbox}[breakable, size=fbox, boxrule=1pt, pad at break*=1mm,colback=cellbackground, colframe=cellborder]
\prompt{In}{incolor}{2}{\boxspacing}
\begin{Verbatim}[commandchars=\\\{\}]
\PY{n}{df\PYZus{}train} \PY{o}{=} \PY{n}{pd}\PY{o}{.}\PY{n}{read\PYZus{}csv}\PY{p}{(}\PY{l+s+s2}{\PYZdq{}}\PY{l+s+s2}{ds/train.csv}\PY{l+s+s2}{\PYZdq{}}\PY{p}{)}
\end{Verbatim}
\end{tcolorbox}

    Según los datos proporcionados no existen valores perdidos. Además todas
las columnas tienen valores numéricos.

    \begin{tcolorbox}[breakable, size=fbox, boxrule=1pt, pad at break*=1mm,colback=cellbackground, colframe=cellborder]
\prompt{In}{incolor}{3}{\boxspacing}
\begin{Verbatim}[commandchars=\\\{\}]
\PY{n}{df\PYZus{}train}\PY{o}{.}\PY{n}{info}\PY{p}{(}\PY{p}{)}
\end{Verbatim}
\end{tcolorbox}

    \begin{Verbatim}[commandchars=\\\{\}]
<class 'pandas.core.frame.DataFrame'>
RangeIndex: 2000 entries, 0 to 1999
Data columns (total 21 columns):
 \#   Column         Non-Null Count  Dtype
---  ------         --------------  -----
 0   battery\_power  2000 non-null   int64
 1   blue           2000 non-null   int64
 2   clock\_speed    2000 non-null   float64
 3   dual\_sim       2000 non-null   int64
 4   fc             2000 non-null   int64
 5   four\_g         2000 non-null   int64
 6   int\_memory     2000 non-null   int64
 7   m\_dep          2000 non-null   float64
 8   mobile\_wt      2000 non-null   int64
 9   n\_cores        2000 non-null   int64
 10  pc             2000 non-null   int64
 11  px\_height      2000 non-null   int64
 12  px\_width       2000 non-null   int64
 13  ram            2000 non-null   int64
 14  sc\_h           2000 non-null   int64
 15  sc\_w           2000 non-null   int64
 16  talk\_time      2000 non-null   int64
 17  three\_g        2000 non-null   int64
 18  touch\_screen   2000 non-null   int64
 19  wifi           2000 non-null   int64
 20  price\_range    2000 non-null   int64
dtypes: float64(2), int64(19)
memory usage: 328.2 KB
    \end{Verbatim}

    \begin{tcolorbox}[breakable, size=fbox, boxrule=1pt, pad at break*=1mm,colback=cellbackground, colframe=cellborder]
\prompt{In}{incolor}{4}{\boxspacing}
\begin{Verbatim}[commandchars=\\\{\}]
\PY{n}{df\PYZus{}train}\PY{o}{.}\PY{n}{describe}\PY{p}{(}\PY{p}{)}\PY{o}{.}\PY{n}{transpose}\PY{p}{(}\PY{p}{)} 
\end{Verbatim}
\end{tcolorbox}

            \begin{tcolorbox}[breakable, size=fbox, boxrule=.5pt, pad at break*=1mm, opacityfill=0]
\prompt{Out}{outcolor}{4}{\boxspacing}
\begin{Verbatim}[commandchars=\\\{\}]
                count        mean          std    min      25\%     50\%  \textbackslash{}
battery\_power  2000.0  1238.51850   439.418206  501.0   851.75  1226.0
blue           2000.0     0.49500     0.500100    0.0     0.00     0.0
clock\_speed    2000.0     1.52225     0.816004    0.5     0.70     1.5
dual\_sim       2000.0     0.50950     0.500035    0.0     0.00     1.0
fc             2000.0     4.30950     4.341444    0.0     1.00     3.0
four\_g         2000.0     0.52150     0.499662    0.0     0.00     1.0
int\_memory     2000.0    32.04650    18.145715    2.0    16.00    32.0
m\_dep          2000.0     0.50175     0.288416    0.1     0.20     0.5
mobile\_wt      2000.0   140.24900    35.399655   80.0   109.00   141.0
n\_cores        2000.0     4.52050     2.287837    1.0     3.00     4.0
pc             2000.0     9.91650     6.064315    0.0     5.00    10.0
px\_height      2000.0   645.10800   443.780811    0.0   282.75   564.0
px\_width       2000.0  1251.51550   432.199447  500.0   874.75  1247.0
ram            2000.0  2124.21300  1084.732044  256.0  1207.50  2146.5
sc\_h           2000.0    12.30650     4.213245    5.0     9.00    12.0
sc\_w           2000.0     5.76700     4.356398    0.0     2.00     5.0
talk\_time      2000.0    11.01100     5.463955    2.0     6.00    11.0
three\_g        2000.0     0.76150     0.426273    0.0     1.00     1.0
touch\_screen   2000.0     0.50300     0.500116    0.0     0.00     1.0
wifi           2000.0     0.50700     0.500076    0.0     0.00     1.0
price\_range    2000.0     1.50000     1.118314    0.0     0.75     1.5

                   75\%     max
battery\_power  1615.25  1998.0
blue              1.00     1.0
clock\_speed       2.20     3.0
dual\_sim          1.00     1.0
fc                7.00    19.0
four\_g            1.00     1.0
int\_memory       48.00    64.0
m\_dep             0.80     1.0
mobile\_wt       170.00   200.0
n\_cores           7.00     8.0
pc               15.00    20.0
px\_height       947.25  1960.0
px\_width       1633.00  1998.0
ram            3064.50  3998.0
sc\_h             16.00    19.0
sc\_w              9.00    18.0
talk\_time        16.00    20.0
three\_g           1.00     1.0
touch\_screen      1.00     1.0
wifi              1.00     1.0
price\_range       2.25     3.0
\end{Verbatim}
\end{tcolorbox}
        
    Aún así parece que no todas las variables son realmente numéricas, pues
no caen en valores continuos. Como ejemplo es posible observar la
columna objetivo, la cual por medio de números distingue cuatro
categorías. Debido a la falta de metadatos en el dataset no es posible
saber con certeza a que corresponde cada una, sin embargo, una
clasificación tradicional es: el segmento de entra (0), la gama media
(1), la gama alta (2) y la gama premium (3).

    \begin{tcolorbox}[breakable, size=fbox, boxrule=1pt, pad at break*=1mm,colback=cellbackground, colframe=cellborder]
\prompt{In}{incolor}{5}{\boxspacing}
\begin{Verbatim}[commandchars=\\\{\}]
\PY{n}{sns}\PY{o}{.}\PY{n}{countplot}\PY{p}{(}\PY{n}{x}\PY{o}{=}\PY{l+s+s1}{\PYZsq{}}\PY{l+s+s1}{price\PYZus{}range}\PY{l+s+s1}{\PYZsq{}}\PY{p}{,} \PY{n}{data}\PY{o}{=}\PY{n}{df\PYZus{}train}\PY{p}{)}
\PY{n}{plt}\PY{o}{.}\PY{n}{title}\PY{p}{(}\PY{l+s+s1}{\PYZsq{}}\PY{l+s+s1}{Frecuencias de price\PYZus{}range}\PY{l+s+s1}{\PYZsq{}}\PY{p}{,} \PY{n}{fontsize}\PY{o}{=}\PY{l+m+mi}{12}\PY{p}{)}
\end{Verbatim}
\end{tcolorbox}

            \begin{tcolorbox}[breakable, size=fbox, boxrule=.5pt, pad at break*=1mm, opacityfill=0]
\prompt{Out}{outcolor}{5}{\boxspacing}
\begin{Verbatim}[commandchars=\\\{\}]
Text(0.5, 1.0, 'Frecuencias de price\_range')
\end{Verbatim}
\end{tcolorbox}
        
    \begin{center}
    \adjustimage{max size={0.9\linewidth}{0.9\paperheight}}{output_8_1.png}
    \end{center}
    { \hspace*{\fill} \\}
    
    \hypertarget{variables-categuxf3ricas}{%
\subsection{Variables categóricas}\label{variables-categuxf3ricas}}

Es importante observar entonces que datos son categóricos y examinarlos
como tal.

    \begin{tcolorbox}[breakable, size=fbox, boxrule=1pt, pad at break*=1mm,colback=cellbackground, colframe=cellborder]
\prompt{In}{incolor}{6}{\boxspacing}
\begin{Verbatim}[commandchars=\\\{\}]
\PY{k}{for} \PY{n}{i} \PY{o+ow}{in} \PY{n}{df\PYZus{}train}\PY{o}{.}\PY{n}{columns}\PY{p}{:}
    \PY{n+nb}{print}\PY{p}{(}\PY{l+s+sa}{f}\PY{l+s+s1}{\PYZsq{}}\PY{l+s+s1}{Valores posibles de }\PY{l+s+si}{\PYZob{}}\PY{n}{i}\PY{o}{.}\PY{n}{title}\PY{p}{(}\PY{p}{)}\PY{l+s+si}{\PYZcb{}}\PY{l+s+s1}{: }\PY{l+s+si}{\PYZob{}}\PY{n}{df\PYZus{}train}\PY{p}{[}\PY{n}{i}\PY{p}{]}\PY{o}{.}\PY{n}{unique}\PY{p}{(}\PY{p}{)}\PY{l+s+si}{\PYZcb{}}\PY{l+s+s1}{\PYZsq{}}\PY{p}{)}
\end{Verbatim}
\end{tcolorbox}

    \begin{Verbatim}[commandchars=\\\{\}]
Valores posibles de Battery\_Power: [ 842 1021  563 {\ldots} 1139 1467  858]
Valores posibles de Blue: [0 1]
Valores posibles de Clock\_Speed: [2.2 0.5 2.5 1.2 1.7 0.6 2.9 2.8 2.1 1.  0.9
1.1 2.6 1.4 1.6 2.7 1.3 2.3
 2.  1.8 3.  1.5 1.9 2.4 0.8 0.7]
Valores posibles de Dual\_Sim: [0 1]
Valores posibles de Fc: [ 1  0  2 13  3  4  5  7 11 12 16  6 15  8  9 10 18 17
14 19]
Valores posibles de Four\_G: [0 1]
Valores posibles de Int\_Memory: [ 7 53 41 10 44 22 24  9 33 17 52 46 13 23 49 19
39 47 38  8 57 51 21  5
 60 61  6 11 50 34 20 27 42 40 64 14 63 43 16 48 12 55 36 30 45 29 58 25
  3 54 15 37 31 32  4 18  2 56 26 35 59 28 62]
Valores posibles de M\_Dep: [0.6 0.7 0.9 0.8 0.1 0.5 1.  0.3 0.4 0.2]
Valores posibles de Mobile\_Wt: [188 136 145 131 141 164 139 187 174  93 182 177
159 198 185 196 121 101
  81 156 199 114 111 132 143  96 200  88 150 107 100 157 160 119  87 152
 166 110 118 162 127 109 102 104 148 180 128 134 144 168 155 165  80 138
 142  90 197 172 116  85 163 178 171 103  83 140 194 146 192 106 135 153
  89  82 130 189 181  99 184 195 108 133 179 147 137 190 176  84  97 124
 183 113  92  95 151 117  94 173 105 115  91 112 123 129 154 191 175  86
  98 125 126 158 170 161 193 169 120 149 186 122 167]
Valores posibles de N\_Cores: [2 3 5 6 1 8 4 7]
Valores posibles de Pc: [ 2  6  9 14  7 10  0 15  1 18 17 11 16  4 20 13  3 19
8  5 12]
Valores posibles de Px\_Height: [  20  905 1263 {\ldots}  528  915  483]
Valores posibles de Px\_Width: [ 756 1988 1716 {\ldots}  743 1890 1632]
Valores posibles de Ram: [2549 2631 2603 {\ldots} 2032 3057 3919]
Valores posibles de Sc\_H: [ 9 17 11 16  8 13 19  5 14 18  7 10 12  6 15]
Valores posibles de Sc\_W: [ 7  3  2  8  1 10  9  0 15 13  5 11  4 12  6 17 14 16
18]
Valores posibles de Talk\_Time: [19  7  9 11 15 10 18  5 20 12 13  2  4  3 16  6
14 17  8]
Valores posibles de Three\_G: [0 1]
Valores posibles de Touch\_Screen: [0 1]
Valores posibles de Wifi: [1 0]
Valores posibles de Price\_Range: [1 2 3 0]
    \end{Verbatim}

    Las variables categóricas parecen ser: \texttt{blue},
\texttt{dual\_sim}, \texttt{four\_g}, \texttt{three\_g},
\texttt{touch\_screen}, \texttt{wifi} y \texttt{price\_range}. Debido a
que \texttt{price\_range} es nuestra variable objetivo, en seguida se
grafican las frecuencias de las otras variables categóricas tomando en
cuenta el rango de precios.

    \begin{tcolorbox}[breakable, size=fbox, boxrule=1pt, pad at break*=1mm,colback=cellbackground, colframe=cellborder]
\prompt{In}{incolor}{7}{\boxspacing}
\begin{Verbatim}[commandchars=\\\{\}]
\PY{n}{cv}\PY{o}{=}\PY{p}{[}\PY{l+s+s1}{\PYZsq{}}\PY{l+s+s1}{blue}\PY{l+s+s1}{\PYZsq{}}\PY{p}{,} \PY{l+s+s1}{\PYZsq{}}\PY{l+s+s1}{dual\PYZus{}sim}\PY{l+s+s1}{\PYZsq{}}\PY{p}{,} \PY{l+s+s1}{\PYZsq{}}\PY{l+s+s1}{four\PYZus{}g}\PY{l+s+s1}{\PYZsq{}}\PY{p}{,} \PY{l+s+s1}{\PYZsq{}}\PY{l+s+s1}{three\PYZus{}g}\PY{l+s+s1}{\PYZsq{}}\PY{p}{,} \PY{l+s+s1}{\PYZsq{}}\PY{l+s+s1}{touch\PYZus{}screen}\PY{l+s+s1}{\PYZsq{}}\PY{p}{,} \PY{l+s+s1}{\PYZsq{}}\PY{l+s+s1}{wifi}\PY{l+s+s1}{\PYZsq{}}\PY{p}{]}
\PY{n}{fig}\PY{o}{=}\PY{n}{plt}\PY{o}{.}\PY{n}{figure}\PY{p}{(}\PY{n}{figsize}\PY{o}{=}\PY{p}{(}\PY{l+m+mi}{24}\PY{p}{,}\PY{l+m+mi}{12}\PY{p}{)}\PY{p}{)}
\PY{n}{plt}\PY{o}{.}\PY{n}{title}\PY{p}{(}\PY{l+s+s1}{\PYZsq{}}\PY{l+s+s1}{Frecuencias de variables categóricas tomando en cuenta el rango de precios}\PY{l+s+s1}{\PYZsq{}}\PY{p}{,}\PY{n}{fontdict}\PY{o}{=}\PY{p}{\PYZob{}}\PY{l+s+s1}{\PYZsq{}}\PY{l+s+s1}{fontsize}\PY{l+s+s1}{\PYZsq{}}\PY{p}{:}\PY{l+m+mi}{20}\PY{p}{\PYZcb{}}\PY{p}{)}
\PY{n}{plt}\PY{o}{.}\PY{n}{axis}\PY{p}{(}\PY{l+s+s1}{\PYZsq{}}\PY{l+s+s1}{off}\PY{l+s+s1}{\PYZsq{}}\PY{p}{)}

\PY{k}{for} \PY{n}{i} \PY{o+ow}{in} \PY{n+nb}{range}\PY{p}{(}\PY{n+nb}{len}\PY{p}{(}\PY{n}{cv}\PY{p}{)}\PY{p}{)}\PY{p}{:}
    \PY{n}{fig}\PY{o}{.}\PY{n}{add\PYZus{}subplot}\PY{p}{(}\PY{l+m+mi}{2}\PY{p}{,}\PY{l+m+mi}{3}\PY{p}{,}\PY{n}{i}\PY{o}{+}\PY{l+m+mi}{1}\PY{p}{)}
    \PY{n}{sns}\PY{o}{.}\PY{n}{countplot}\PY{p}{(}\PY{n}{data}\PY{o}{=}\PY{n}{df\PYZus{}train}\PY{p}{,}\PY{n}{x}\PY{o}{=}\PY{n}{cv}\PY{p}{[}\PY{n}{i}\PY{p}{]}\PY{p}{,} \PY{n}{hue}\PY{o}{=}\PY{l+s+s1}{\PYZsq{}}\PY{l+s+s1}{price\PYZus{}range}\PY{l+s+s1}{\PYZsq{}}\PY{p}{)}
    \PY{n}{plt}\PY{o}{.}\PY{n}{legend}\PY{p}{(}\PY{n}{bbox\PYZus{}to\PYZus{}anchor}\PY{o}{=}\PY{p}{(}\PY{l+m+mf}{1.02}\PY{p}{,} \PY{l+m+mi}{1}\PY{p}{)}\PY{p}{,} \PY{n}{borderaxespad}\PY{o}{=}\PY{l+m+mi}{0}\PY{p}{)}
\end{Verbatim}
\end{tcolorbox}

    \begin{center}
    \adjustimage{max size={0.9\linewidth}{0.9\paperheight}}{output_12_0.png}
    \end{center}
    { \hspace*{\fill} \\}
    
    Es posible observar que los teléfonos más costosos cuentan con más de
estas características, salvo en el caso de \texttt{touch\_screen}, donde
no hay diferencia frecuencial visible. La variable \texttt{three\_g}
parece ser más significativa ya que la división es un poco más clara.
También hay que matizar que esta diferencia no es suficiente para
distinguir invariablemente las categorías de rango de precio.

\hypertarget{variables-numuxe9ricas}{%
\subsection{Variables numéricas}\label{variables-numuxe9ricas}}

Se repetirá el procedimiento anterior procedimiento para las variables
que son efectivamente numéricas pues servirá para observar el contraste
entre los tipos de variables.

    \begin{tcolorbox}[breakable, size=fbox, boxrule=1pt, pad at break*=1mm,colback=cellbackground, colframe=cellborder]
\prompt{In}{incolor}{8}{\boxspacing}
\begin{Verbatim}[commandchars=\\\{\}]
\PY{n}{nv}\PY{o}{=}\PY{p}{[}\PY{l+s+s1}{\PYZsq{}}\PY{l+s+s1}{battery\PYZus{}power}\PY{l+s+s1}{\PYZsq{}}\PY{p}{,} \PY{l+s+s1}{\PYZsq{}}\PY{l+s+s1}{clock\PYZus{}speed}\PY{l+s+s1}{\PYZsq{}}\PY{p}{,} \PY{l+s+s1}{\PYZsq{}}\PY{l+s+s1}{fc}\PY{l+s+s1}{\PYZsq{}}\PY{p}{,} \PY{l+s+s1}{\PYZsq{}}\PY{l+s+s1}{int\PYZus{}memory}\PY{l+s+s1}{\PYZsq{}}\PY{p}{,} \PY{l+s+s1}{\PYZsq{}}\PY{l+s+s1}{m\PYZus{}dep}\PY{l+s+s1}{\PYZsq{}}\PY{p}{,} \PY{l+s+s1}{\PYZsq{}}\PY{l+s+s1}{mobile\PYZus{}wt}\PY{l+s+s1}{\PYZsq{}}\PY{p}{,} \PY{l+s+s1}{\PYZsq{}}\PY{l+s+s1}{n\PYZus{}cores}\PY{l+s+s1}{\PYZsq{}}\PY{p}{,} \PY{l+s+s1}{\PYZsq{}}\PY{l+s+s1}{pc}\PY{l+s+s1}{\PYZsq{}}\PY{p}{,} \PY{l+s+s1}{\PYZsq{}}\PY{l+s+s1}{px\PYZus{}height}\PY{l+s+s1}{\PYZsq{}}\PY{p}{,} \PY{l+s+s1}{\PYZsq{}}\PY{l+s+s1}{px\PYZus{}width}\PY{l+s+s1}{\PYZsq{}}\PY{p}{,} \PY{l+s+s1}{\PYZsq{}}\PY{l+s+s1}{ram}\PY{l+s+s1}{\PYZsq{}}\PY{p}{,} \PY{l+s+s1}{\PYZsq{}}\PY{l+s+s1}{sc\PYZus{}h}\PY{l+s+s1}{\PYZsq{}}\PY{p}{,} \PY{l+s+s1}{\PYZsq{}}\PY{l+s+s1}{sc\PYZus{}w}\PY{l+s+s1}{\PYZsq{}}\PY{p}{,} \PY{l+s+s1}{\PYZsq{}}\PY{l+s+s1}{talk\PYZus{}time}\PY{l+s+s1}{\PYZsq{}}\PY{p}{]}
\PY{n}{fig}\PY{o}{=}\PY{n}{plt}\PY{o}{.}\PY{n}{figure}\PY{p}{(}\PY{n}{figsize}\PY{o}{=}\PY{p}{(}\PY{l+m+mi}{24}\PY{p}{,}\PY{l+m+mi}{30}\PY{p}{)}\PY{p}{)}
\PY{n}{plt}\PY{o}{.}\PY{n}{title}\PY{p}{(}\PY{l+s+s1}{\PYZsq{}}\PY{l+s+s1}{Frecuencia de variables numéricas tomando en cuenta el rango de precios}\PY{l+s+s1}{\PYZsq{}}\PY{p}{)}
\PY{n}{plt}\PY{o}{.}\PY{n}{axis}\PY{p}{(}\PY{l+s+s1}{\PYZsq{}}\PY{l+s+s1}{off}\PY{l+s+s1}{\PYZsq{}}\PY{p}{)}

\PY{k}{for} \PY{n}{i} \PY{o+ow}{in} \PY{n+nb}{range}\PY{p}{(}\PY{n+nb}{len}\PY{p}{(}\PY{n}{nv}\PY{p}{)}\PY{p}{)}\PY{p}{:}
    \PY{n}{fig}\PY{o}{.}\PY{n}{add\PYZus{}subplot}\PY{p}{(}\PY{l+m+mi}{7}\PY{p}{,}\PY{l+m+mi}{2}\PY{p}{,}\PY{n}{i}\PY{o}{+}\PY{l+m+mi}{1}\PY{p}{)}
    \PY{n}{sns}\PY{o}{.}\PY{n}{kdeplot}\PY{p}{(}\PY{n}{data}\PY{o}{=}\PY{n}{df\PYZus{}train}\PY{p}{,} \PY{n}{x}\PY{o}{=}\PY{n}{nv}\PY{p}{[}\PY{n}{i}\PY{p}{]}\PY{p}{,} \PY{n}{hue}\PY{o}{=}\PY{l+s+s1}{\PYZsq{}}\PY{l+s+s1}{price\PYZus{}range}\PY{l+s+s1}{\PYZsq{}}\PY{p}{)}
\end{Verbatim}
\end{tcolorbox}

    \begin{center}
    \adjustimage{max size={0.9\linewidth}{0.9\paperheight}}{output_14_0.png}
    \end{center}
    { \hspace*{\fill} \\}
    
    La dentro de las gráficas anteriores lo más destacado es aquella que
representa la variable \texttt{ram}. En dicha representación se
distinguen muy claramente los segmentos de precio, es posible pensar que
que la variable \texttt{ram}, tenga una correlación alta respecto a
\texttt{price\_range}, mientras otras como \texttt{clock\_speed} sean de
muy poca relevancia.

\hypertarget{matriz-de-correlaciuxf3n}{%
\section{Matriz de correlación}\label{matriz-de-correlaciuxf3n}}

La matriz es indispensable para distinguir la importancia de las
variables y así pensar en un mejor modelo.

    \begin{tcolorbox}[breakable, size=fbox, boxrule=1pt, pad at break*=1mm,colback=cellbackground, colframe=cellborder]
\prompt{In}{incolor}{9}{\boxspacing}
\begin{Verbatim}[commandchars=\\\{\}]
\PY{n}{plt}\PY{o}{.}\PY{n}{figure}\PY{p}{(}\PY{n}{figsize}\PY{o}{=}\PY{p}{(}\PY{l+m+mi}{20}\PY{p}{,}\PY{l+m+mi}{8}\PY{p}{)}\PY{p}{,}\PY{n}{dpi}\PY{o}{=}\PY{l+m+mi}{80}\PY{p}{)}
\PY{n}{corrmat} \PY{o}{=} \PY{n}{df\PYZus{}train}\PY{o}{.}\PY{n}{corr}\PY{p}{(}\PY{p}{)}
\PY{n}{sns}\PY{o}{.}\PY{n}{heatmap}\PY{p}{(}\PY{n}{corrmat}\PY{p}{,} \PY{n}{cmap}\PY{o}{=}\PY{l+s+s1}{\PYZsq{}}\PY{l+s+s1}{coolwarm}\PY{l+s+s1}{\PYZsq{}}\PY{p}{,} \PY{n}{vmax}\PY{o}{=}\PY{l+m+mf}{.8}\PY{p}{,} \PY{n}{fmt}\PY{o}{=}\PY{l+s+s1}{\PYZsq{}}\PY{l+s+s1}{.1f}\PY{l+s+s1}{\PYZsq{}}\PY{p}{,} \PY{n}{annot}\PY{o}{=}\PY{k+kc}{True}\PY{p}{)}
\end{Verbatim}
\end{tcolorbox}

            \begin{tcolorbox}[breakable, size=fbox, boxrule=.5pt, pad at break*=1mm, opacityfill=0]
\prompt{Out}{outcolor}{9}{\boxspacing}
\begin{Verbatim}[commandchars=\\\{\}]
<AxesSubplot:>
\end{Verbatim}
\end{tcolorbox}
        
    \begin{center}
    \adjustimage{max size={0.9\linewidth}{0.9\paperheight}}{output_16_1.png}
    \end{center}
    { \hspace*{\fill} \\}
    
    \begin{tcolorbox}[breakable, size=fbox, boxrule=1pt, pad at break*=1mm,colback=cellbackground, colframe=cellborder]
\prompt{In}{incolor}{10}{\boxspacing}
\begin{Verbatim}[commandchars=\\\{\}]
\PY{n}{df\PYZus{}train}\PY{o}{.}\PY{n}{corr}\PY{p}{(}\PY{p}{)}\PY{p}{[}\PY{l+s+s1}{\PYZsq{}}\PY{l+s+s1}{price\PYZus{}range}\PY{l+s+s1}{\PYZsq{}}\PY{p}{]}\PY{o}{.}\PY{n}{sort\PYZus{}values}\PY{p}{(}\PY{n}{ascending}\PY{o}{=}\PY{k+kc}{False}\PY{p}{)}\PY{p}{[}\PY{l+m+mi}{1}\PY{p}{:}\PY{l+m+mi}{21}\PY{p}{]}
\end{Verbatim}
\end{tcolorbox}

            \begin{tcolorbox}[breakable, size=fbox, boxrule=.5pt, pad at break*=1mm, opacityfill=0]
\prompt{Out}{outcolor}{10}{\boxspacing}
\begin{Verbatim}[commandchars=\\\{\}]
ram              0.917046
battery\_power    0.200723
px\_width         0.165818
px\_height        0.148858
int\_memory       0.044435
sc\_w             0.038711
pc               0.033599
three\_g          0.023611
sc\_h             0.022986
fc               0.021998
talk\_time        0.021859
blue             0.020573
wifi             0.018785
dual\_sim         0.017444
four\_g           0.014772
n\_cores          0.004399
m\_dep            0.000853
clock\_speed     -0.006606
mobile\_wt       -0.030302
touch\_screen    -0.030411
Name: price\_range, dtype: float64
\end{Verbatim}
\end{tcolorbox}
        
    Tal y como se dijo anteriormente, la variable \texttt{ram} tiene una
alta correlación con la variable objetivo.

    \hypertarget{svm}{%
\section{SVM}\label{svm}}

Se implementará una SVM. Al trabajar con métodos supervisados es
necesario dividir el dataframe en dos partes.

    \begin{tcolorbox}[breakable, size=fbox, boxrule=1pt, pad at break*=1mm,colback=cellbackground, colframe=cellborder]
\prompt{In}{incolor}{11}{\boxspacing}
\begin{Verbatim}[commandchars=\\\{\}]
\PY{n}{train}\PY{p}{,} \PY{n}{test} \PY{o}{=} \PY{n}{train\PYZus{}test\PYZus{}split}\PY{p}{(}\PY{n}{df\PYZus{}train}\PY{p}{,} \PY{n}{test\PYZus{}size}\PY{o}{=}\PY{l+m+mf}{0.2}\PY{p}{)}
\PY{n}{predictors} \PY{o}{=} \PY{n}{cv} \PY{o}{+} \PY{n}{nv}
\PY{c+c1}{\PYZsh{}predictors = [\PYZsq{}ram\PYZsq{}, \PYZsq{}battery\PYZus{}power\PYZsq{}, \PYZsq{}px\PYZus{}width\PYZsq{}, \PYZsq{}px\PYZus{}height\PYZsq{}, \PYZsq{}int\PYZus{}memory\PYZsq{}]}
\PY{n}{target} \PY{o}{=} \PY{p}{[}\PY{l+s+s1}{\PYZsq{}}\PY{l+s+s1}{price\PYZus{}range}\PY{l+s+s1}{\PYZsq{}}\PY{p}{]}
\end{Verbatim}
\end{tcolorbox}

    Ahora, ante la diversidad de hiperparámetros se crea la SVM al mismo
tiempo que se busca el mejor modelo posible. Este proceso es muy lento,
a continuación se muestran solo los hiperparámetros más relevantes,
otros fueron descartados pues en una ejecución individual dieron
resultados deficientes.

    \begin{tcolorbox}[breakable, size=fbox, boxrule=1pt, pad at break*=1mm,colback=cellbackground, colframe=cellborder]
\prompt{In}{incolor}{12}{\boxspacing}
\begin{Verbatim}[commandchars=\\\{\}]
\PY{n}{parameters} \PY{o}{=} \PY{p}{[}\PY{p}{\PYZob{}}\PY{l+s+s1}{\PYZsq{}}\PY{l+s+s1}{kernel}\PY{l+s+s1}{\PYZsq{}}\PY{p}{:} \PY{p}{[}\PY{l+s+s1}{\PYZsq{}}\PY{l+s+s1}{rbf}\PY{l+s+s1}{\PYZsq{}}\PY{p}{,} \PY{l+s+s1}{\PYZsq{}}\PY{l+s+s1}{linear}\PY{l+s+s1}{\PYZsq{}}\PY{p}{]}\PY{p}{,} \PY{l+s+s1}{\PYZsq{}}\PY{l+s+s1}{gamma}\PY{l+s+s1}{\PYZsq{}}\PY{p}{:} \PY{p}{[}\PY{l+s+s1}{\PYZsq{}}\PY{l+s+s1}{scale}\PY{l+s+s1}{\PYZsq{}}\PY{p}{,} \PY{l+s+s1}{\PYZsq{}}\PY{l+s+s1}{auto}\PY{l+s+s1}{\PYZsq{}}\PY{p}{]}\PY{p}{,} \PY{l+s+s1}{\PYZsq{}}\PY{l+s+s1}{C}\PY{l+s+s1}{\PYZsq{}}\PY{p}{:} \PY{p}{[}\PY{l+m+mf}{0.5}\PY{p}{,} \PY{l+m+mf}{1.0}\PY{p}{,} \PY{l+m+mf}{1.5}\PY{p}{,} \PY{l+m+mf}{2.0}\PY{p}{,} \PY{l+m+mf}{2.5}\PY{p}{]}\PY{p}{\PYZcb{}}\PY{p}{]}
\PY{n}{svmc} \PY{o}{=} \PY{n}{RandomizedSearchCV}\PY{p}{(}\PY{n}{svm}\PY{o}{.}\PY{n}{SVC}\PY{p}{(}\PY{n}{decision\PYZus{}function\PYZus{}shape}\PY{o}{=}\PY{l+s+s1}{\PYZsq{}}\PY{l+s+s1}{ovr}\PY{l+s+s1}{\PYZsq{}}\PY{p}{)}\PY{p}{,} \PY{n}{param\PYZus{}distributions}\PY{o}{=}\PY{n}{parameters}\PY{p}{,} \PY{n}{cv}\PY{o}{=}\PY{l+m+mi}{5}\PY{p}{,} \PY{n}{scoring}\PY{o}{=}\PY{l+s+s1}{\PYZsq{}}\PY{l+s+s1}{accuracy}\PY{l+s+s1}{\PYZsq{}}\PY{p}{)}
\PY{n}{svmc}\PY{o}{.}\PY{n}{fit}\PY{p}{(}\PY{n}{train}\PY{p}{[}\PY{n}{predictors}\PY{p}{]}\PY{p}{,} \PY{n}{np}\PY{o}{.}\PY{n}{ravel}\PY{p}{(}\PY{n}{train}\PY{p}{[}\PY{n}{target}\PY{p}{]}\PY{p}{)}\PY{p}{)}
\PY{n}{svmc}\PY{o}{.}\PY{n}{best\PYZus{}params\PYZus{}}
\end{Verbatim}
\end{tcolorbox}

            \begin{tcolorbox}[breakable, size=fbox, boxrule=.5pt, pad at break*=1mm, opacityfill=0]
\prompt{Out}{outcolor}{12}{\boxspacing}
\begin{Verbatim}[commandchars=\\\{\}]
\{'kernel': 'linear', 'gamma': 'auto', 'C': 0.5\}
\end{Verbatim}
\end{tcolorbox}
        
    \begin{tcolorbox}[breakable, size=fbox, boxrule=1pt, pad at break*=1mm,colback=cellbackground, colframe=cellborder]
\prompt{In}{incolor}{13}{\boxspacing}
\begin{Verbatim}[commandchars=\\\{\}]
\PY{n}{svm\PYZus{}predicted} \PY{o}{=} \PY{n}{svmc}\PY{o}{.}\PY{n}{predict}\PY{p}{(}\PY{n}{test}\PY{p}{[}\PY{n}{predictors}\PY{p}{]}\PY{p}{)}
\PY{n}{crosstab} \PY{o}{=} \PY{n}{sns}\PY{o}{.}\PY{n}{heatmap}\PY{p}{(}\PY{n}{pd}\PY{o}{.}\PY{n}{crosstab}\PY{p}{(}\PY{n}{test}\PY{p}{[}\PY{n}{target}\PY{p}{]}\PY{o}{.}\PY{n}{values}\PY{o}{.}\PY{n}{ravel}\PY{p}{(}\PY{p}{)}\PY{p}{,} \PY{n}{svm\PYZus{}predicted}\PY{p}{)}\PY{p}{,}
                \PY{n}{cmap}\PY{o}{=}\PY{l+s+s1}{\PYZsq{}}\PY{l+s+s1}{gist\PYZus{}yarg}\PY{l+s+s1}{\PYZsq{}}\PY{p}{,} \PY{n}{vmax}\PY{o}{=}\PY{l+m+mf}{.8}\PY{p}{,} \PY{n}{fmt}\PY{o}{=}\PY{l+s+s1}{\PYZsq{}}\PY{l+s+s1}{g}\PY{l+s+s1}{\PYZsq{}}\PY{p}{,} \PY{n}{annot}\PY{o}{=}\PY{k+kc}{True}\PY{p}{)}
\PY{n}{crosstab}\PY{o}{.}\PY{n}{set\PYZus{}xlabel}\PY{p}{(}\PY{l+s+s1}{\PYZsq{}}\PY{l+s+s1}{Predicciones}\PY{l+s+s1}{\PYZsq{}}\PY{p}{)}
\PY{n}{crosstab}\PY{o}{.}\PY{n}{set\PYZus{}ylabel}\PY{p}{(}\PY{l+s+s1}{\PYZsq{}}\PY{l+s+s1}{Reales}\PY{l+s+s1}{\PYZsq{}}\PY{p}{)}
\PY{n+nb}{print}\PY{p}{(}\PY{l+s+s2}{\PYZdq{}}\PY{l+s+s2}{Accuracy: }\PY{l+s+s2}{\PYZdq{}}\PY{p}{,} \PY{n}{metrics}\PY{o}{.}\PY{n}{accuracy\PYZus{}score}\PY{p}{(}\PY{n}{svm\PYZus{}predicted}\PY{p}{,} \PY{n}{np}\PY{o}{.}\PY{n}{ravel}\PY{p}{(}\PY{n}{test}\PY{p}{[}\PY{n}{target}\PY{p}{]}\PY{p}{)}\PY{p}{)}\PY{p}{)}
\end{Verbatim}
\end{tcolorbox}

    \begin{Verbatim}[commandchars=\\\{\}]
Accuracy:  0.975
    \end{Verbatim}

    \begin{center}
    \adjustimage{max size={0.9\linewidth}{0.9\paperheight}}{output_23_1.png}
    \end{center}
    { \hspace*{\fill} \\}
    
    Los resultados de la SVM resultan bastante prometedores, la precisión
alcanzada es muy alta, en alguna ejecución llegó a ser del 99\%, sin
embargo, se consigue en pocos casos.

    \hypertarget{red-neuronal}{%
\section{Red Neuronal}\label{red-neuronal}}

La red neuronal ha resultado ser más complicada de implementar debido a
que requiere el diseño de la las capaz de la red. Aquí se presenta una
estructura básica. \citep{rickytb2021, Keras2022}

    \begin{tcolorbox}[breakable, size=fbox, boxrule=1pt, pad at break*=1mm,colback=cellbackground, colframe=cellborder]
\prompt{In}{incolor}{14}{\boxspacing}
\begin{Verbatim}[commandchars=\\\{\}]
\PY{n}{nn} \PY{o}{=} \PY{n}{Sequential}\PY{p}{(}\PY{p}{)}
\PY{n}{nn}\PY{o}{.}\PY{n}{add}\PY{p}{(}\PY{n}{Dense}\PY{p}{(}\PY{n}{units}\PY{o}{=}\PY{l+m+mi}{16}\PY{p}{,} \PY{n}{input\PYZus{}dim}\PY{o}{=}\PY{l+m+mi}{20}\PY{p}{,} \PY{n}{activation}\PY{o}{=}\PY{l+s+s1}{\PYZsq{}}\PY{l+s+s1}{relu}\PY{l+s+s1}{\PYZsq{}}\PY{p}{)}\PY{p}{)}
\PY{n}{nn}\PY{o}{.}\PY{n}{add}\PY{p}{(}\PY{n}{Dense}\PY{p}{(}\PY{n}{units}\PY{o}{=}\PY{l+m+mi}{12}\PY{p}{,} \PY{n}{activation}\PY{o}{=}\PY{l+s+s1}{\PYZsq{}}\PY{l+s+s1}{relu}\PY{l+s+s1}{\PYZsq{}}\PY{p}{)}\PY{p}{)}
\PY{n}{nn}\PY{o}{.}\PY{n}{add}\PY{p}{(}\PY{n}{Dense}\PY{p}{(}\PY{n}{units}\PY{o}{=}\PY{l+m+mi}{4}\PY{p}{,} \PY{n}{activation}\PY{o}{=}\PY{l+s+s1}{\PYZsq{}}\PY{l+s+s1}{softmax}\PY{l+s+s1}{\PYZsq{}}\PY{p}{)}\PY{p}{)}
\end{Verbatim}
\end{tcolorbox}

    \begin{Verbatim}[commandchars=\\\{\}]
WARNING:tensorflow:From /srv/conda/envs/notebook/lib/python3.7/site-
packages/tensorflow\_core/python/ops/resource\_variable\_ops.py:1630: calling
BaseResourceVariable.\_\_init\_\_ (from tensorflow.python.ops.resource\_variable\_ops)
with constraint is deprecated and will be removed in a future version.
Instructions for updating:
If using Keras pass *\_constraint arguments to layers.
    \end{Verbatim}

    \begin{tcolorbox}[breakable, size=fbox, boxrule=1pt, pad at break*=1mm,colback=cellbackground, colframe=cellborder]
\prompt{In}{incolor}{15}{\boxspacing}
\begin{Verbatim}[commandchars=\\\{\}]
\PY{n}{nn}\PY{o}{.}\PY{n}{compile}\PY{p}{(}\PY{n}{loss}\PY{o}{=}\PY{l+s+s1}{\PYZsq{}}\PY{l+s+s1}{sparse\PYZus{}categorical\PYZus{}crossentropy}\PY{l+s+s1}{\PYZsq{}}\PY{p}{,} \PY{n}{optimizer}\PY{o}{=}\PY{l+s+s1}{\PYZsq{}}\PY{l+s+s1}{adam}\PY{l+s+s1}{\PYZsq{}}\PY{p}{,} \PY{n}{metrics}\PY{o}{=}\PY{p}{[}\PY{l+s+s1}{\PYZsq{}}\PY{l+s+s1}{accuracy}\PY{l+s+s1}{\PYZsq{}}\PY{p}{]}\PY{p}{)}
\end{Verbatim}
\end{tcolorbox}

    \begin{tcolorbox}[breakable, size=fbox, boxrule=1pt, pad at break*=1mm,colback=cellbackground, colframe=cellborder]
\prompt{In}{incolor}{16}{\boxspacing}
\begin{Verbatim}[commandchars=\\\{\}]
\PY{n}{nn}\PY{o}{.}\PY{n}{fit}\PY{p}{(}\PY{n}{train}\PY{p}{[}\PY{n}{predictors}\PY{p}{]}\PY{p}{,} \PY{n}{np}\PY{o}{.}\PY{n}{ravel}\PY{p}{(}\PY{n}{train}\PY{p}{[}\PY{n}{target}\PY{p}{]}\PY{p}{)}\PY{p}{,} \PY{n}{epochs}\PY{o}{=}\PY{l+m+mi}{550}\PY{p}{,} \PY{n}{batch\PYZus{}size}\PY{o}{=}\PY{l+m+mi}{4}\PY{p}{,} \PY{n}{verbose}\PY{o}{=}\PY{l+m+mi}{0}\PY{p}{)}
\end{Verbatim}
\end{tcolorbox}

    \begin{Verbatim}[commandchars=\\\{\}]
2022-04-15 06:14:12.733464: W
tensorflow/stream\_executor/platform/default/dso\_loader.cc:55] Could not load
dynamic library 'libcuda.so.1'; dlerror: libcuda.so.1: cannot open shared object
file: No such file or directory
2022-04-15 06:14:12.733513: E
tensorflow/stream\_executor/cuda/cuda\_driver.cc:318] failed call to cuInit:
UNKNOWN ERROR (303)
2022-04-15 06:14:12.733549: I
tensorflow/stream\_executor/cuda/cuda\_diagnostics.cc:156] kernel driver does not
appear to be running on this host (jupyter-genomorro-2dunir-2duy5jhwqu):
/proc/driver/nvidia/version does not exist
2022-04-15 06:14:12.734649: I tensorflow/core/platform/cpu\_feature\_guard.cc:142]
Your CPU supports instructions that this TensorFlow binary was not compiled to
use: AVX2 AVX512F FMA
2022-04-15 06:14:12.765753: I
tensorflow/core/platform/profile\_utils/cpu\_utils.cc:94] CPU Frequency:
2300000000 Hz
2022-04-15 06:14:12.771347: I tensorflow/compiler/xla/service/service.cc:168]
XLA service 0x55b1568017f0 initialized for platform Host (this does not
guarantee that XLA will be used). Devices:
2022-04-15 06:14:12.771403: I tensorflow/compiler/xla/service/service.cc:176]
StreamExecutor device (0): Host, Default Version
    \end{Verbatim}

            \begin{tcolorbox}[breakable, size=fbox, boxrule=.5pt, pad at break*=1mm, opacityfill=0]
\prompt{Out}{outcolor}{16}{\boxspacing}
\begin{Verbatim}[commandchars=\\\{\}]
<tensorflow.python.keras.callbacks.History at 0x7f07f6a1c210>
\end{Verbatim}
\end{tcolorbox}
        
    \begin{tcolorbox}[breakable, size=fbox, boxrule=1pt, pad at break*=1mm,colback=cellbackground, colframe=cellborder]
\prompt{In}{incolor}{17}{\boxspacing}
\begin{Verbatim}[commandchars=\\\{\}]
\PY{n}{nn\PYZus{}pred} \PY{o}{=} \PY{n}{nn}\PY{o}{.}\PY{n}{predict}\PY{p}{(}\PY{n}{test}\PY{p}{[}\PY{n}{predictors}\PY{p}{]}\PY{p}{)}
\end{Verbatim}
\end{tcolorbox}

    El resultado de la red neuronal no es legible inmediatamente, se
requiere convertir las predicciones a sus respectivas etiquetas.
\citep{rickytb2021}

    \begin{tcolorbox}[breakable, size=fbox, boxrule=1pt, pad at break*=1mm,colback=cellbackground, colframe=cellborder]
\prompt{In}{incolor}{18}{\boxspacing}
\begin{Verbatim}[commandchars=\\\{\}]
\PY{c+c1}{\PYZsh{} Convertir las predicciones a sus respectivas etiquetas}
\PY{k}{def} \PY{n+nf}{pred\PYZus{}to\PYZus{}label}\PY{p}{(}\PY{n}{predictions}\PY{p}{)}\PY{p}{:}
    \PY{n}{pred} \PY{o}{=} \PY{n+nb}{list}\PY{p}{(}\PY{p}{)}
    \PY{k}{for} \PY{n}{i} \PY{o+ow}{in} \PY{n+nb}{range}\PY{p}{(}\PY{n+nb}{len}\PY{p}{(}\PY{n}{predictions}\PY{p}{)}\PY{p}{)}\PY{p}{:}
        \PY{n}{pred}\PY{o}{.}\PY{n}{append}\PY{p}{(}\PY{n}{np}\PY{o}{.}\PY{n}{argmax}\PY{p}{(}\PY{n}{predictions}\PY{p}{[}\PY{n}{i}\PY{p}{]}\PY{p}{)}\PY{p}{)}
    \PY{k}{return} \PY{n}{pred}
\end{Verbatim}
\end{tcolorbox}

    \begin{tcolorbox}[breakable, size=fbox, boxrule=1pt, pad at break*=1mm,colback=cellbackground, colframe=cellborder]
\prompt{In}{incolor}{19}{\boxspacing}
\begin{Verbatim}[commandchars=\\\{\}]
\PY{n}{nn\PYZus{}predicted} \PY{o}{=} \PY{n}{pred\PYZus{}to\PYZus{}label}\PY{p}{(}\PY{n}{nn\PYZus{}pred}\PY{p}{)}
\PY{n}{crosstab} \PY{o}{=} \PY{n}{sns}\PY{o}{.}\PY{n}{heatmap}\PY{p}{(}\PY{n}{pd}\PY{o}{.}\PY{n}{crosstab}\PY{p}{(}\PY{n}{test}\PY{p}{[}\PY{n}{target}\PY{p}{]}\PY{o}{.}\PY{n}{values}\PY{o}{.}\PY{n}{ravel}\PY{p}{(}\PY{p}{)}\PY{p}{,} \PY{n}{np}\PY{o}{.}\PY{n}{array}\PY{p}{(}\PY{n}{nn\PYZus{}predicted}\PY{p}{)}\PY{p}{)}\PY{p}{,}
                \PY{n}{cmap}\PY{o}{=}\PY{l+s+s1}{\PYZsq{}}\PY{l+s+s1}{gist\PYZus{}yarg}\PY{l+s+s1}{\PYZsq{}}\PY{p}{,} \PY{n}{vmax}\PY{o}{=}\PY{l+m+mf}{.8}\PY{p}{,} \PY{n}{fmt}\PY{o}{=}\PY{l+s+s1}{\PYZsq{}}\PY{l+s+s1}{g}\PY{l+s+s1}{\PYZsq{}}\PY{p}{,} \PY{n}{annot}\PY{o}{=}\PY{k+kc}{True}\PY{p}{)}
\PY{n}{crosstab}\PY{o}{.}\PY{n}{set\PYZus{}xlabel}\PY{p}{(}\PY{l+s+s1}{\PYZsq{}}\PY{l+s+s1}{Predicciones}\PY{l+s+s1}{\PYZsq{}}\PY{p}{)}
\PY{n}{crosstab}\PY{o}{.}\PY{n}{set\PYZus{}ylabel}\PY{p}{(}\PY{l+s+s1}{\PYZsq{}}\PY{l+s+s1}{Reales}\PY{l+s+s1}{\PYZsq{}}\PY{p}{)}
\PY{n+nb}{print}\PY{p}{(}\PY{l+s+s2}{\PYZdq{}}\PY{l+s+s2}{Accuracy: }\PY{l+s+s2}{\PYZdq{}}\PY{p}{,} \PY{n}{metrics}\PY{o}{.}\PY{n}{accuracy\PYZus{}score}\PY{p}{(}\PY{n}{nn\PYZus{}predicted}\PY{p}{,} \PY{n}{np}\PY{o}{.}\PY{n}{ravel}\PY{p}{(}\PY{n}{test}\PY{p}{[}\PY{n}{target}\PY{p}{]}\PY{p}{)}\PY{p}{)}\PY{p}{)}
\end{Verbatim}
\end{tcolorbox}

    \begin{Verbatim}[commandchars=\\\{\}]
Accuracy:  0.9525
    \end{Verbatim}

    \begin{center}
    \adjustimage{max size={0.9\linewidth}{0.9\paperheight}}{output_32_1.png}
    \end{center}
    { \hspace*{\fill} \\}
    
    La red neuronal no ha alcanzado la gran precisión que logró la SVM pero
el resultado no se ha alejado realmente. Si se considera que este modelo
fue creado a partir de la documentación básica de la biblioteca que lo
implementa y no ha sido totalmente personalizado, es posible pensar que
el resultado es positivo y podría alcanzar mejores predicciones con
mayor tiempo.

    \hypertarget{comparaciuxf3n-entre-muxe9todos}{%
\section{Comparación entre
métodos}\label{comparaciuxf3n-entre-muxe9todos}}

Como ya se ha visto, la precisión conseguida por la SVM es superior a la
obtenida en la red neuronal. Habrá que considerar que la potencia de
computo ha limitado la red neuronal, por lo tanto, para comparar mejor
los algoritmos se usarán las métricas proporcionadas por
\texttt{sklearn}.

    \begin{tcolorbox}[breakable, size=fbox, boxrule=1pt, pad at break*=1mm,colback=cellbackground, colframe=cellborder]
\prompt{In}{incolor}{20}{\boxspacing}
\begin{Verbatim}[commandchars=\\\{\}]
\PY{n+nb}{print}\PY{p}{(}\PY{n}{metrics}\PY{o}{.}\PY{n}{classification\PYZus{}report}\PY{p}{(}\PY{n}{test}\PY{p}{[}\PY{n}{target}\PY{p}{]}\PY{o}{.}\PY{n}{values}\PY{o}{.}\PY{n}{ravel}\PY{p}{(}\PY{p}{)}\PY{p}{,} \PY{n}{svm\PYZus{}predicted}\PY{p}{)}\PY{p}{)}
\end{Verbatim}
\end{tcolorbox}

    \begin{Verbatim}[commandchars=\\\{\}]
              precision    recall  f1-score   support

           0       0.99      0.98      0.99       105
           1       0.96      0.99      0.97        98
           2       0.96      0.97      0.96        97
           3       0.99      0.96      0.97       100

    accuracy                           0.97       400
   macro avg       0.97      0.97      0.97       400
weighted avg       0.98      0.97      0.98       400

    \end{Verbatim}

    \begin{tcolorbox}[breakable, size=fbox, boxrule=1pt, pad at break*=1mm,colback=cellbackground, colframe=cellborder]
\prompt{In}{incolor}{21}{\boxspacing}
\begin{Verbatim}[commandchars=\\\{\}]
\PY{n+nb}{print}\PY{p}{(}\PY{n}{metrics}\PY{o}{.}\PY{n}{classification\PYZus{}report}\PY{p}{(}\PY{n}{test}\PY{p}{[}\PY{n}{target}\PY{p}{]}\PY{o}{.}\PY{n}{values}\PY{o}{.}\PY{n}{ravel}\PY{p}{(}\PY{p}{)}\PY{p}{,} \PY{n}{nn\PYZus{}predicted}\PY{p}{)}\PY{p}{)}
\end{Verbatim}
\end{tcolorbox}

    \begin{Verbatim}[commandchars=\\\{\}]
              precision    recall  f1-score   support

           0       0.99      0.96      0.98       105
           1       0.95      0.96      0.95        98
           2       0.97      0.89      0.92        97
           3       0.91      1.00      0.95       100

    accuracy                           0.95       400
   macro avg       0.95      0.95      0.95       400
weighted avg       0.95      0.95      0.95       400

    \end{Verbatim}

    Los resultados son realmente parejos. Los mejores números presentados
por la SVM no hacen gran diferencia. El \emph{recall}, la capacidad del
clasificador para encontrar todas las muestras positivas pueden tener
números iguales según sea la ejecución del \emph{notebook} de Python. El
\emph{f1-score} que se interpreta como una media armónica de precisión y
recall, llega en alguna categoría, de cualquiera de los modelos, a su
mejor valor (1), según sea la ejecución del notebook.

Parece que con la implementación aquí elaborada de ambos modelos,
dependerá más de ejecutar el notebook hasta encontrar una ejecución que
logre los mejores resultados que el usuario pueda esperar. Ahora bien,
es importante señalar que la red neuronal parece menos consistente entre
categorías que la SVM, en otras palabras, la SVM clasifica cada una de
las categorías con presiciones similares, mientras que en la red
neuronal se observa que tiene notablemente menor presición en alguna de
las categorías.

    \hypertarget{conclusiuxf3n}{%
\section{Conclusión}\label{conclusiuxf3n}}

En el presente análisis se utilizó un dataset llamado \emph{Mobile price
classification}. Se realizó un estudio estadístico general del dataset y
se encontraron dos mil muestras en el mismo.

Se decidió aplicar dos algoritmos avanzados: SVM y redes neuronales.
Estos algoritmos requieren trabajo al afinar los hiperparámetros de cada
modelo, si bien son herramientas muy poderosas, también hay que decir
que ajustarlas para solucionar el problema puede ser más costoso que
usar otro algoritmo más básico.

En ejemplos sencillos como el resuelto en este documento, dichos
hiperparámetros son ajustables con cierta sencillez. Aún así, la fase de
entrenamiento es más lenta si se considera que el dataset es de solo
2000 muestras. Si a lo anterior se sumaran muestras con overlapping, la
SVM particularmente tenderá a cometer un mayor número de errores. La red
neuronal incluso sugería el uso de una tarjeta gráfica y otras
características para el CPU. Este mensaje puede observarse en la
correspondiente ejecución de la red neuronal del presente análisis.

Si el dataset tiene muchas dimensiones, la SVM puede ser una excelente
alternativa. Aunque no podrán ser visualizadas gráficamente, este
algoritmo lidiará bien con el dataset incluso si el número de
dimensiones supera el número de muestras, como en el procesamiento de
imágenes. Por su parte, las redes neuronales pueden ser más útiles con
información más desestructurada o en datos más complejos sin querer
abusar del método prueba/error.


    % Add a bibliography block to the postdoc
    \bibliographystyle{apalike}
    \bibliography{main}    
    
    
    
\end{document}
