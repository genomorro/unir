\documentclass[11pt,a4paper,table]{article}

    \usepackage[breakable]{tcolorbox}
    \usepackage{parskip} % Stop auto-indenting (to mimic markdown behaviour)
    
    \usepackage{iftex}
    \ifPDFTeX
    	\usepackage[T1]{fontenc}
    	\usepackage{mathpazo}
    \else
    	\usepackage{fontspec}
    \fi

    % Basic figure setup, for now with no caption control since it's done
    % automatically by Pandoc (which extracts ![](path) syntax from Markdown).
    \usepackage{graphicx}
    % Maintain compatibility with old templates. Remove in nbconvert 6.0
    \let\Oldincludegraphics\includegraphics
    % Ensure that by default, figures have no caption (until we provide a
    % proper Figure object with a Caption API and a way to capture that
    % in the conversion process - todo).
    \usepackage{caption}
    \DeclareCaptionFormat{nocaption}{}
    \captionsetup{format=nocaption,aboveskip=0pt,belowskip=0pt}

    \usepackage{float}
    \floatplacement{figure}{H} % forces figures to be placed at the correct location
    \usepackage{xcolor} % Allow colors to be defined
    \usepackage{enumerate} % Needed for markdown enumerations to work
    \usepackage{geometry} % Used to adjust the document margins
    \usepackage{amsmath} % Equations
    \usepackage{amssymb} % Equations
    \usepackage{textcomp} % defines textquotesingle
    % Hack from http://tex.stackexchange.com/a/47451/13684:
    \AtBeginDocument{%
        \def\PYZsq{\textquotesingle}% Upright quotes in Pygmentized code
    }
    \usepackage{upquote} % Upright quotes for verbatim code
    \usepackage{eurosym} % defines \euro
    \usepackage[mathletters]{ucs} % Extended unicode (utf-8) support
    \usepackage{fancyvrb} % verbatim replacement that allows latex
    \usepackage{grffile} % extends the file name processing of package graphics 
                         % to support a larger range
    \makeatletter % fix for old versions of grffile with XeLaTeX
    \@ifpackagelater{grffile}{2019/11/01}
    {
      % Do nothing on new versions
    }
    {
      \def\Gread@@xetex#1{%
        \IfFileExists{"\Gin@base".bb}%
        {\Gread@eps{\Gin@base.bb}}%
        {\Gread@@xetex@aux#1}%
      }
    }
    \makeatother
    \usepackage[Export]{adjustbox} % Used to constrain images to a maximum size
    \adjustboxset{max size={0.9\linewidth}{0.9\paperheight}}

    % The hyperref package gives us a pdf with properly built
    % internal navigation ('pdf bookmarks' for the table of contents,
    % internal cross-reference links, web links for URLs, etc.)
    \usepackage{hyperref}
    % The default LaTeX title has an obnoxious amount of whitespace. By default,
    % titling removes some of it. It also provides customization options.
    \usepackage{titling}
    \usepackage{longtable} % longtable support required by pandoc >1.10
    \usepackage{booktabs}  % table support for pandoc > 1.12.2
    \usepackage[inline]{enumitem} % IRkernel/repr support (it uses the enumerate* environment)
    \usepackage[normalem]{ulem} % ulem is needed to support strikethroughs (\sout)
                                % normalem makes italics be italics, not underlines
    \usepackage{mathrsfs}
    

    
    % Colors for the hyperref package
    \definecolor{urlcolor}{rgb}{0,.145,.698}
    \definecolor{linkcolor}{rgb}{.71,0.21,0.01}
    \definecolor{citecolor}{rgb}{.12,.54,.11}

    % ANSI colors
    \definecolor{ansi-black}{HTML}{3E424D}
    \definecolor{ansi-black-intense}{HTML}{282C36}
    \definecolor{ansi-red}{HTML}{E75C58}
    \definecolor{ansi-red-intense}{HTML}{B22B31}
    \definecolor{ansi-green}{HTML}{00A250}
    \definecolor{ansi-green-intense}{HTML}{007427}
    \definecolor{ansi-yellow}{HTML}{DDB62B}
    \definecolor{ansi-yellow-intense}{HTML}{B27D12}
    \definecolor{ansi-blue}{HTML}{208FFB}
    \definecolor{ansi-blue-intense}{HTML}{0065CA}
    \definecolor{ansi-magenta}{HTML}{D160C4}
    \definecolor{ansi-magenta-intense}{HTML}{A03196}
    \definecolor{ansi-cyan}{HTML}{60C6C8}
    \definecolor{ansi-cyan-intense}{HTML}{258F8F}
    \definecolor{ansi-white}{HTML}{C5C1B4}
    \definecolor{ansi-white-intense}{HTML}{A1A6B2}
    \definecolor{ansi-default-inverse-fg}{HTML}{FFFFFF}
    \definecolor{ansi-default-inverse-bg}{HTML}{000000}

    % common color for the border for error outputs.
    \definecolor{outerrorbackground}{HTML}{FFDFDF}

    % commands and environments needed by pandoc snippets
    % extracted from the output of `pandoc -s`
    \providecommand{\tightlist}{%
      \setlength{\itemsep}{0pt}\setlength{\parskip}{0pt}}
    \DefineVerbatimEnvironment{Highlighting}{Verbatim}{commandchars=\\\{\}}
    % Add ',fontsize=\small' for more characters per line
    \newenvironment{Shaded}{}{}
    \newcommand{\KeywordTok}[1]{\textcolor[rgb]{0.00,0.44,0.13}{\textbf{{#1}}}}
    \newcommand{\DataTypeTok}[1]{\textcolor[rgb]{0.56,0.13,0.00}{{#1}}}
    \newcommand{\DecValTok}[1]{\textcolor[rgb]{0.25,0.63,0.44}{{#1}}}
    \newcommand{\BaseNTok}[1]{\textcolor[rgb]{0.25,0.63,0.44}{{#1}}}
    \newcommand{\FloatTok}[1]{\textcolor[rgb]{0.25,0.63,0.44}{{#1}}}
    \newcommand{\CharTok}[1]{\textcolor[rgb]{0.25,0.44,0.63}{{#1}}}
    \newcommand{\StringTok}[1]{\textcolor[rgb]{0.25,0.44,0.63}{{#1}}}
    \newcommand{\CommentTok}[1]{\textcolor[rgb]{0.38,0.63,0.69}{\textit{{#1}}}}
    \newcommand{\OtherTok}[1]{\textcolor[rgb]{0.00,0.44,0.13}{{#1}}}
    \newcommand{\AlertTok}[1]{\textcolor[rgb]{1.00,0.00,0.00}{\textbf{{#1}}}}
    \newcommand{\FunctionTok}[1]{\textcolor[rgb]{0.02,0.16,0.49}{{#1}}}
    \newcommand{\RegionMarkerTok}[1]{{#1}}
    \newcommand{\ErrorTok}[1]{\textcolor[rgb]{1.00,0.00,0.00}{\textbf{{#1}}}}
    \newcommand{\NormalTok}[1]{{#1}}
    
    % Additional commands for more recent versions of Pandoc
    \newcommand{\ConstantTok}[1]{\textcolor[rgb]{0.53,0.00,0.00}{{#1}}}
    \newcommand{\SpecialCharTok}[1]{\textcolor[rgb]{0.25,0.44,0.63}{{#1}}}
    \newcommand{\VerbatimStringTok}[1]{\textcolor[rgb]{0.25,0.44,0.63}{{#1}}}
    \newcommand{\SpecialStringTok}[1]{\textcolor[rgb]{0.73,0.40,0.53}{{#1}}}
    \newcommand{\ImportTok}[1]{{#1}}
    \newcommand{\DocumentationTok}[1]{\textcolor[rgb]{0.73,0.13,0.13}{\textit{{#1}}}}
    \newcommand{\AnnotationTok}[1]{\textcolor[rgb]{0.38,0.63,0.69}{\textbf{\textit{{#1}}}}}
    \newcommand{\CommentVarTok}[1]{\textcolor[rgb]{0.38,0.63,0.69}{\textbf{\textit{{#1}}}}}
    \newcommand{\VariableTok}[1]{\textcolor[rgb]{0.10,0.09,0.49}{{#1}}}
    \newcommand{\ControlFlowTok}[1]{\textcolor[rgb]{0.00,0.44,0.13}{\textbf{{#1}}}}
    \newcommand{\OperatorTok}[1]{\textcolor[rgb]{0.40,0.40,0.40}{{#1}}}
    \newcommand{\BuiltInTok}[1]{{#1}}
    \newcommand{\ExtensionTok}[1]{{#1}}
    \newcommand{\PreprocessorTok}[1]{\textcolor[rgb]{0.74,0.48,0.00}{{#1}}}
    \newcommand{\AttributeTok}[1]{\textcolor[rgb]{0.49,0.56,0.16}{{#1}}}
    \newcommand{\InformationTok}[1]{\textcolor[rgb]{0.38,0.63,0.69}{\textbf{\textit{{#1}}}}}
    \newcommand{\WarningTok}[1]{\textcolor[rgb]{0.38,0.63,0.69}{\textbf{\textit{{#1}}}}}
    
    
    % Define a nice break command that doesn't care if a line doesn't already
    % exist.
    \def\br{\hspace*{\fill} \\* }
    % Math Jax compatibility definitions
    \def\gt{>}
    \def\lt{<}
    \let\Oldtex\TeX
    \let\Oldlatex\LaTeX
    \renewcommand{\TeX}{\textrm{\Oldtex}}
    \renewcommand{\LaTeX}{\textrm{\Oldlatex}}
    % Document parameters
    % Document title
    \title{Actividad grupal: Usos reales de filtros espaciales y morfológicos}
    
    
    
    
    
% Pygments definitions
\makeatletter
\def\PY@reset{\let\PY@it=\relax \let\PY@bf=\relax%
    \let\PY@ul=\relax \let\PY@tc=\relax%
    \let\PY@bc=\relax \let\PY@ff=\relax}
\def\PY@tok#1{\csname PY@tok@#1\endcsname}
\def\PY@toks#1+{\ifx\relax#1\empty\else%
    \PY@tok{#1}\expandafter\PY@toks\fi}
\def\PY@do#1{\PY@bc{\PY@tc{\PY@ul{%
    \PY@it{\PY@bf{\PY@ff{#1}}}}}}}
\def\PY#1#2{\PY@reset\PY@toks#1+\relax+\PY@do{#2}}

\@namedef{PY@tok@w}{\def\PY@tc##1{\textcolor[rgb]{0.73,0.73,0.73}{##1}}}
\@namedef{PY@tok@c}{\let\PY@it=\textit\def\PY@tc##1{\textcolor[rgb]{0.24,0.48,0.48}{##1}}}
\@namedef{PY@tok@cp}{\def\PY@tc##1{\textcolor[rgb]{0.61,0.40,0.00}{##1}}}
\@namedef{PY@tok@k}{\let\PY@bf=\textbf\def\PY@tc##1{\textcolor[rgb]{0.00,0.50,0.00}{##1}}}
\@namedef{PY@tok@kp}{\def\PY@tc##1{\textcolor[rgb]{0.00,0.50,0.00}{##1}}}
\@namedef{PY@tok@kt}{\def\PY@tc##1{\textcolor[rgb]{0.69,0.00,0.25}{##1}}}
\@namedef{PY@tok@o}{\def\PY@tc##1{\textcolor[rgb]{0.40,0.40,0.40}{##1}}}
\@namedef{PY@tok@ow}{\let\PY@bf=\textbf\def\PY@tc##1{\textcolor[rgb]{0.67,0.13,1.00}{##1}}}
\@namedef{PY@tok@nb}{\def\PY@tc##1{\textcolor[rgb]{0.00,0.50,0.00}{##1}}}
\@namedef{PY@tok@nf}{\def\PY@tc##1{\textcolor[rgb]{0.00,0.00,1.00}{##1}}}
\@namedef{PY@tok@nc}{\let\PY@bf=\textbf\def\PY@tc##1{\textcolor[rgb]{0.00,0.00,1.00}{##1}}}
\@namedef{PY@tok@nn}{\let\PY@bf=\textbf\def\PY@tc##1{\textcolor[rgb]{0.00,0.00,1.00}{##1}}}
\@namedef{PY@tok@ne}{\let\PY@bf=\textbf\def\PY@tc##1{\textcolor[rgb]{0.80,0.25,0.22}{##1}}}
\@namedef{PY@tok@nv}{\def\PY@tc##1{\textcolor[rgb]{0.10,0.09,0.49}{##1}}}
\@namedef{PY@tok@no}{\def\PY@tc##1{\textcolor[rgb]{0.53,0.00,0.00}{##1}}}
\@namedef{PY@tok@nl}{\def\PY@tc##1{\textcolor[rgb]{0.46,0.46,0.00}{##1}}}
\@namedef{PY@tok@ni}{\let\PY@bf=\textbf\def\PY@tc##1{\textcolor[rgb]{0.44,0.44,0.44}{##1}}}
\@namedef{PY@tok@na}{\def\PY@tc##1{\textcolor[rgb]{0.41,0.47,0.13}{##1}}}
\@namedef{PY@tok@nt}{\let\PY@bf=\textbf\def\PY@tc##1{\textcolor[rgb]{0.00,0.50,0.00}{##1}}}
\@namedef{PY@tok@nd}{\def\PY@tc##1{\textcolor[rgb]{0.67,0.13,1.00}{##1}}}
\@namedef{PY@tok@s}{\def\PY@tc##1{\textcolor[rgb]{0.73,0.13,0.13}{##1}}}
\@namedef{PY@tok@sd}{\let\PY@it=\textit\def\PY@tc##1{\textcolor[rgb]{0.73,0.13,0.13}{##1}}}
\@namedef{PY@tok@si}{\let\PY@bf=\textbf\def\PY@tc##1{\textcolor[rgb]{0.64,0.35,0.47}{##1}}}
\@namedef{PY@tok@se}{\let\PY@bf=\textbf\def\PY@tc##1{\textcolor[rgb]{0.67,0.36,0.12}{##1}}}
\@namedef{PY@tok@sr}{\def\PY@tc##1{\textcolor[rgb]{0.64,0.35,0.47}{##1}}}
\@namedef{PY@tok@ss}{\def\PY@tc##1{\textcolor[rgb]{0.10,0.09,0.49}{##1}}}
\@namedef{PY@tok@sx}{\def\PY@tc##1{\textcolor[rgb]{0.00,0.50,0.00}{##1}}}
\@namedef{PY@tok@m}{\def\PY@tc##1{\textcolor[rgb]{0.40,0.40,0.40}{##1}}}
\@namedef{PY@tok@gh}{\let\PY@bf=\textbf\def\PY@tc##1{\textcolor[rgb]{0.00,0.00,0.50}{##1}}}
\@namedef{PY@tok@gu}{\let\PY@bf=\textbf\def\PY@tc##1{\textcolor[rgb]{0.50,0.00,0.50}{##1}}}
\@namedef{PY@tok@gd}{\def\PY@tc##1{\textcolor[rgb]{0.63,0.00,0.00}{##1}}}
\@namedef{PY@tok@gi}{\def\PY@tc##1{\textcolor[rgb]{0.00,0.52,0.00}{##1}}}
\@namedef{PY@tok@gr}{\def\PY@tc##1{\textcolor[rgb]{0.89,0.00,0.00}{##1}}}
\@namedef{PY@tok@ge}{\let\PY@it=\textit}
\@namedef{PY@tok@gs}{\let\PY@bf=\textbf}
\@namedef{PY@tok@gp}{\let\PY@bf=\textbf\def\PY@tc##1{\textcolor[rgb]{0.00,0.00,0.50}{##1}}}
\@namedef{PY@tok@go}{\def\PY@tc##1{\textcolor[rgb]{0.44,0.44,0.44}{##1}}}
\@namedef{PY@tok@gt}{\def\PY@tc##1{\textcolor[rgb]{0.00,0.27,0.87}{##1}}}
\@namedef{PY@tok@err}{\def\PY@bc##1{{\setlength{\fboxsep}{\string -\fboxrule}\fcolorbox[rgb]{1.00,0.00,0.00}{1,1,1}{\strut ##1}}}}
\@namedef{PY@tok@kc}{\let\PY@bf=\textbf\def\PY@tc##1{\textcolor[rgb]{0.00,0.50,0.00}{##1}}}
\@namedef{PY@tok@kd}{\let\PY@bf=\textbf\def\PY@tc##1{\textcolor[rgb]{0.00,0.50,0.00}{##1}}}
\@namedef{PY@tok@kn}{\let\PY@bf=\textbf\def\PY@tc##1{\textcolor[rgb]{0.00,0.50,0.00}{##1}}}
\@namedef{PY@tok@kr}{\let\PY@bf=\textbf\def\PY@tc##1{\textcolor[rgb]{0.00,0.50,0.00}{##1}}}
\@namedef{PY@tok@bp}{\def\PY@tc##1{\textcolor[rgb]{0.00,0.50,0.00}{##1}}}
\@namedef{PY@tok@fm}{\def\PY@tc##1{\textcolor[rgb]{0.00,0.00,1.00}{##1}}}
\@namedef{PY@tok@vc}{\def\PY@tc##1{\textcolor[rgb]{0.10,0.09,0.49}{##1}}}
\@namedef{PY@tok@vg}{\def\PY@tc##1{\textcolor[rgb]{0.10,0.09,0.49}{##1}}}
\@namedef{PY@tok@vi}{\def\PY@tc##1{\textcolor[rgb]{0.10,0.09,0.49}{##1}}}
\@namedef{PY@tok@vm}{\def\PY@tc##1{\textcolor[rgb]{0.10,0.09,0.49}{##1}}}
\@namedef{PY@tok@sa}{\def\PY@tc##1{\textcolor[rgb]{0.73,0.13,0.13}{##1}}}
\@namedef{PY@tok@sb}{\def\PY@tc##1{\textcolor[rgb]{0.73,0.13,0.13}{##1}}}
\@namedef{PY@tok@sc}{\def\PY@tc##1{\textcolor[rgb]{0.73,0.13,0.13}{##1}}}
\@namedef{PY@tok@dl}{\def\PY@tc##1{\textcolor[rgb]{0.73,0.13,0.13}{##1}}}
\@namedef{PY@tok@s2}{\def\PY@tc##1{\textcolor[rgb]{0.73,0.13,0.13}{##1}}}
\@namedef{PY@tok@sh}{\def\PY@tc##1{\textcolor[rgb]{0.73,0.13,0.13}{##1}}}
\@namedef{PY@tok@s1}{\def\PY@tc##1{\textcolor[rgb]{0.73,0.13,0.13}{##1}}}
\@namedef{PY@tok@mb}{\def\PY@tc##1{\textcolor[rgb]{0.40,0.40,0.40}{##1}}}
\@namedef{PY@tok@mf}{\def\PY@tc##1{\textcolor[rgb]{0.40,0.40,0.40}{##1}}}
\@namedef{PY@tok@mh}{\def\PY@tc##1{\textcolor[rgb]{0.40,0.40,0.40}{##1}}}
\@namedef{PY@tok@mi}{\def\PY@tc##1{\textcolor[rgb]{0.40,0.40,0.40}{##1}}}
\@namedef{PY@tok@il}{\def\PY@tc##1{\textcolor[rgb]{0.40,0.40,0.40}{##1}}}
\@namedef{PY@tok@mo}{\def\PY@tc##1{\textcolor[rgb]{0.40,0.40,0.40}{##1}}}
\@namedef{PY@tok@ch}{\let\PY@it=\textit\def\PY@tc##1{\textcolor[rgb]{0.24,0.48,0.48}{##1}}}
\@namedef{PY@tok@cm}{\let\PY@it=\textit\def\PY@tc##1{\textcolor[rgb]{0.24,0.48,0.48}{##1}}}
\@namedef{PY@tok@cpf}{\let\PY@it=\textit\def\PY@tc##1{\textcolor[rgb]{0.24,0.48,0.48}{##1}}}
\@namedef{PY@tok@c1}{\let\PY@it=\textit\def\PY@tc##1{\textcolor[rgb]{0.24,0.48,0.48}{##1}}}
\@namedef{PY@tok@cs}{\let\PY@it=\textit\def\PY@tc##1{\textcolor[rgb]{0.24,0.48,0.48}{##1}}}

\def\PYZbs{\char`\\}
\def\PYZus{\char`\_}
\def\PYZob{\char`\{}
\def\PYZcb{\char`\}}
\def\PYZca{\char`\^}
\def\PYZam{\char`\&}
\def\PYZlt{\char`\<}
\def\PYZgt{\char`\>}
\def\PYZsh{\char`\#}
\def\PYZpc{\char`\%}
\def\PYZdl{\char`\$}
\def\PYZhy{\char`\-}
\def\PYZsq{\char`\'}
\def\PYZdq{\char`\"}
\def\PYZti{\char`\~}
% for compatibility with earlier versions
\def\PYZat{@}
\def\PYZlb{[}
\def\PYZrb{]}
\makeatother


    % For linebreaks inside Verbatim environment from package fancyvrb. 
    \makeatletter
        \newbox\Wrappedcontinuationbox 
        \newbox\Wrappedvisiblespacebox 
        \newcommand*\Wrappedvisiblespace {\textcolor{red}{\textvisiblespace}} 
        \newcommand*\Wrappedcontinuationsymbol {\textcolor{red}{\llap{\tiny$\m@th\hookrightarrow$}}} 
        \newcommand*\Wrappedcontinuationindent {3ex } 
        \newcommand*\Wrappedafterbreak {\kern\Wrappedcontinuationindent\copy\Wrappedcontinuationbox} 
        % Take advantage of the already applied Pygments mark-up to insert 
        % potential linebreaks for TeX processing. 
        %        {, <, #, %, $, ' and ": go to next line. 
        %        _, }, ^, &, >, - and ~: stay at end of broken line. 
        % Use of \textquotesingle for straight quote. 
        \newcommand*\Wrappedbreaksatspecials {% 
            \def\PYGZus{\discretionary{\char`\_}{\Wrappedafterbreak}{\char`\_}}% 
            \def\PYGZob{\discretionary{}{\Wrappedafterbreak\char`\{}{\char`\{}}% 
            \def\PYGZcb{\discretionary{\char`\}}{\Wrappedafterbreak}{\char`\}}}% 
            \def\PYGZca{\discretionary{\char`\^}{\Wrappedafterbreak}{\char`\^}}% 
            \def\PYGZam{\discretionary{\char`\&}{\Wrappedafterbreak}{\char`\&}}% 
            \def\PYGZlt{\discretionary{}{\Wrappedafterbreak\char`\<}{\char`\<}}% 
            \def\PYGZgt{\discretionary{\char`\>}{\Wrappedafterbreak}{\char`\>}}% 
            \def\PYGZsh{\discretionary{}{\Wrappedafterbreak\char`\#}{\char`\#}}% 
            \def\PYGZpc{\discretionary{}{\Wrappedafterbreak\char`\%}{\char`\%}}% 
            \def\PYGZdl{\discretionary{}{\Wrappedafterbreak\char`\$}{\char`\$}}% 
            \def\PYGZhy{\discretionary{\char`\-}{\Wrappedafterbreak}{\char`\-}}% 
            \def\PYGZsq{\discretionary{}{\Wrappedafterbreak\textquotesingle}{\textquotesingle}}% 
            \def\PYGZdq{\discretionary{}{\Wrappedafterbreak\char`\"}{\char`\"}}% 
            \def\PYGZti{\discretionary{\char`\~}{\Wrappedafterbreak}{\char`\~}}% 
        } 
        % Some characters . , ; ? ! / are not pygmentized. 
        % This macro makes them "active" and they will insert potential linebreaks 
        \newcommand*\Wrappedbreaksatpunct {% 
            \lccode`\~`\.\lowercase{\def~}{\discretionary{\hbox{\char`\.}}{\Wrappedafterbreak}{\hbox{\char`\.}}}% 
            \lccode`\~`\,\lowercase{\def~}{\discretionary{\hbox{\char`\,}}{\Wrappedafterbreak}{\hbox{\char`\,}}}% 
            \lccode`\~`\;\lowercase{\def~}{\discretionary{\hbox{\char`\;}}{\Wrappedafterbreak}{\hbox{\char`\;}}}% 
            \lccode`\~`\:\lowercase{\def~}{\discretionary{\hbox{\char`\:}}{\Wrappedafterbreak}{\hbox{\char`\:}}}% 
            \lccode`\~`\?\lowercase{\def~}{\discretionary{\hbox{\char`\?}}{\Wrappedafterbreak}{\hbox{\char`\?}}}% 
            \lccode`\~`\!\lowercase{\def~}{\discretionary{\hbox{\char`\!}}{\Wrappedafterbreak}{\hbox{\char`\!}}}% 
            \lccode`\~`\/\lowercase{\def~}{\discretionary{\hbox{\char`\/}}{\Wrappedafterbreak}{\hbox{\char`\/}}}% 
            \catcode`\.\active
            \catcode`\,\active 
            \catcode`\;\active
            \catcode`\:\active
            \catcode`\?\active
            \catcode`\!\active
            \catcode`\/\active 
            \lccode`\~`\~ 	
        }
    \makeatother

    \let\OriginalVerbatim=\Verbatim
    \makeatletter
    \renewcommand{\Verbatim}[1][1]{%
        %\parskip\z@skip
        \sbox\Wrappedcontinuationbox {\Wrappedcontinuationsymbol}%
        \sbox\Wrappedvisiblespacebox {\FV@SetupFont\Wrappedvisiblespace}%
        \def\FancyVerbFormatLine ##1{\hsize\linewidth
            \vtop{\raggedright\hyphenpenalty\z@\exhyphenpenalty\z@
                \doublehyphendemerits\z@\finalhyphendemerits\z@
                \strut ##1\strut}%
        }%
        % If the linebreak is at a space, the latter will be displayed as visible
        % space at end of first line, and a continuation symbol starts next line.
        % Stretch/shrink are however usually zero for typewriter font.
        \def\FV@Space {%
            \nobreak\hskip\z@ plus\fontdimen3\font minus\fontdimen4\font
            \discretionary{\copy\Wrappedvisiblespacebox}{\Wrappedafterbreak}
            {\kern\fontdimen2\font}%
        }%
        
        % Allow breaks at special characters using \PYG... macros.
        \Wrappedbreaksatspecials
        % Breaks at punctuation characters . , ; ? ! and / need catcode=\active 	
        \OriginalVerbatim[#1,codes*=\Wrappedbreaksatpunct]%
    }
    \makeatother

    % Exact colors from NB
    \definecolor{incolor}{HTML}{303F9F}
    \definecolor{outcolor}{HTML}{D84315}
    \definecolor{cellborder}{HTML}{CFCFCF}
    \definecolor{cellbackground}{HTML}{F7F7F7}
    
    % prompt
    \makeatletter
    \newcommand{\boxspacing}{\kern\kvtcb@left@rule\kern\kvtcb@boxsep}
    \makeatother
    \newcommand{\prompt}[4]{
        {\ttfamily\llap{{\color{#2}[#3]:\hspace{3pt}#4}}\vspace{-\baselineskip}}
    }
    

    
    % Prevent overflowing lines due to hard-to-break entities
    \sloppy 
    % Setup hyperref package
    \hypersetup{
      breaklinks=true,  % so long urls are correctly broken across lines
      colorlinks=true,
      urlcolor=urlcolor,
      linkcolor=linkcolor,
      citecolor=citecolor,
      }
    % Slightly bigger margins than the latex defaults
    
    \geometry{verbose,tmargin=1in,bmargin=1in,lmargin=1in,rmargin=1in}

    % UNIR
    \usepackage[spanish,mexico]{babel}
    \makeatletter
    \let\newtitle\@title
    \makeatother
    \usepackage{amsmath}
    \usepackage{multirow}
    \definecolor{UnirLight}{HTML}{E6F4F9}
    \definecolor{UnirDark}{HTML}{0098CD}
    \arrayrulecolor{UnirDark}
    \usepackage{titlesec}
    \titleformat*{\section}{\color{UnirDark}\normalsize\bfseries}
    \titleformat*{\subsection}{\color{UnirDark}\normalsize\bfseries}
    \titleformat*{\subsubsection}{\color{UnirDark}\normalsize\bfseries}
    \usepackage{fancyhdr}
    \pagestyle{fancy}
    \renewcommand{\headrulewidth}{0pt}
    \headheight=56pt
    \setlength{\footskip}{64pt}
    \lhead{}
    \chead{
    \begin{tabular}{|c|l|c|}
     \hline
     \rowcolor{UnirLight}
     \textcolor{UnirDark}{Asignatura} & \textcolor{UnirDark}{Datos del alumno} & \textcolor{UnirDark}{Fecha} \\
     \hline
     & Bernal Castillo Aldo Alberto & \\
     \textbf{Percepción computacional} & Calderón Zetter María Inés & \today \\
     & Domínguez Espinoza Edgar Uriel & \\
     \hline
    \end{tabular}}
    \rhead{}
    \lfoot{}
    \cfoot{}
    \rfoot{\makebox(70,56)[t]{\textcolor{UnirDark}{Actividades}}
        \colorbox{UnirDark}{
            \makebox(10,56)[t]{
                \textcolor{white}{\thepage}}}}
    \usepackage[color={[gray]{0.5}}, angle=90,fontsize=9pt,anchor=lb,pos={0.03\paperwidth,0.95\paperheight}]{draftwatermark}
    \SetWatermarkText{{\copyright} Universidad Internacional de La Rioja en México (UNIR)}
    \hypersetup{
      pdfauthor={Aldo Alberto Bernal Castillo and María Inés Calderón Zetter
María Inés Calderón Zetter and Edgar Uriel Domínguez Espinoza},
      pdftitle={Actividad grupal: Usos reales de filtros espaciales y morfológicos},
      pdfkeywords={filtro, Percepción Computacional, imágenes, morfología},
      pdfsubject={Percepción computacional},
      pdfcreator={Emacs 27.2},
      pdflang={Spanish}}
    \usepackage{natbib}

    

\begin{document}
    
    

    
    \hypertarget{actividad-grupal-usos-reales-de-filtros-espaciales-y-morfoluxf3gicos}{%
\textcolor{UnirDark}{\Large\bfseries\newtitle}\label{actividad-grupal-usos-reales-de-filtros-espaciales-y-morfoluxf3gicos}}

\hypertarget{introduccion}{%
\subsection*{Introducción}\label{introduccion}}

La morfología es un estudio de la forma y la estructura. En el
procesamiento de imágenes, se utiliza para analizar y modificar
propiedades geométricas de una imagen probándola con diferentes formas.
Las características geométicas de estas formas, llamadas elementos
estructurales, conduce a medidas cuantitativas que son útiles en la
visión informática. El proceso es similar a la convolución lineal y la
correlación, excepto que las operaciones lógicas AND, OR, y NOT se
utilizan en un área en lugar de operaciones aritméticas. Los píxeles se
añaden a un objeto o se eliminan de él. Hay que definir la extensión del
área, y los elementos de estructurales, los cuales pueden ser rotados
por 180 grados.\cite{Sundararajan_2017}

En la operación de la convolución, con máscaras hechas de diferentes
tipos de respuestas de impulso, podemos procesar señales con diferentes
filtros como paso bajo y paso alto. De manera similar, con diferentes
tipos de elementos de estructurales y la realización de la convolución
con operadores lógicos, podemos realizar diversos tipos de análisis de
objetos. Aunque su uso primario es con imágenes binarias, la morfología
también se extiende a las imágenes a escala gris.\cite{Sundararajan_2017}

En el presente caso utilizamos imagenes de rayos X, que por la
naturaleza de las mismas presentan alto contraste y escalas de grises.
Estas imágenes corresponden directamente a consulta médica donde existió
un agente extraño en el cuerpo humano y con las operaciones de filtros
morfológicos pretendemos hacer más sencilla su determinación de posible
diagnóstico y tratamiento médico. Las imágenes presentaron resultados
favorables para las operaciones en cuestión.

    \hypertarget{librerias-a-utilizar}{%
\subsection*{Librerias a utilizar}\label{librerias-a-utilizar}}

    \begin{tcolorbox}[breakable, size=fbox, boxrule=1pt, pad at break*=1mm,colback=cellbackground, colframe=cellborder]
\prompt{In}{incolor}{1}{\boxspacing}
\begin{Verbatim}[commandchars=\\\{\}]
\PY{k+kn}{import} \PY{n+nn}{cv2} \PY{k}{as} \PY{n+nn}{cv}
\PY{k+kn}{import} \PY{n+nn}{numpy} \PY{k}{as} \PY{n+nn}{np}
\PY{k+kn}{from} \PY{n+nn}{matplotlib} \PY{k+kn}{import} \PY{n}{pyplot} \PY{k}{as} \PY{n}{plt}
\end{Verbatim}
\end{tcolorbox}

    \hypertarget{imuxe1genes-usadas}{%
\subsection*{Imágenes usadas}\label{imuxe1genes-usadas}}

En estas tres imágenes \texttt{img\_0} representa una imagen de control,
mientras que \texttt{img\_1} y \texttt{img\_2} corresponden imágenes a
rayos X.\cite{Children-Dental-Ark, Ortodoncis_2013, Radiologia-Etomatologica_2017}

    \begin{tcolorbox}[breakable, size=fbox, boxrule=1pt, pad at break*=1mm,colback=cellbackground, colframe=cellborder]
\prompt{In}{incolor}{2}{\boxspacing}
\begin{Verbatim}[commandchars=\\\{\}]
\PY{n}{img\PYZus{}0} \PY{o}{=} \PY{n}{cv}\PY{o}{.}\PY{n}{imread}\PY{p}{(}\PY{l+s+s1}{\PYZsq{}}\PY{l+s+s1}{im/unir\PYZhy{}1.jpg}\PY{l+s+s1}{\PYZsq{}}\PY{p}{,} \PY{n}{cv}\PY{o}{.}\PY{n}{IMREAD\PYZus{}GRAYSCALE}\PY{p}{)}
\PY{n}{img\PYZus{}1} \PY{o}{=} \PY{n}{cv}\PY{o}{.}\PY{n}{imread}\PY{p}{(}\PY{l+s+s1}{\PYZsq{}}\PY{l+s+s1}{im/dental\PYZhy{}1.jpg}\PY{l+s+s1}{\PYZsq{}}\PY{p}{,} \PY{l+m+mi}{0}\PY{p}{)}
\PY{n}{img\PYZus{}2} \PY{o}{=} \PY{n}{cv}\PY{o}{.}\PY{n}{imread}\PY{p}{(}\PY{l+s+s1}{\PYZsq{}}\PY{l+s+s1}{im/dental\PYZhy{}2.jpg}\PY{l+s+s1}{\PYZsq{}}\PY{p}{,} \PY{l+m+mi}{0}\PY{p}{)}
\end{Verbatim}
\end{tcolorbox}

    \hypertarget{implementaciuxf3n-de-operadores-morfoluxf3gicos}{%
\subsection*{Implementación de operadores
morfológicos}\label{implementaciuxf3n-de-operadores-morfoluxf3gicos}}

La dilatación suele generar un efecto de ampliación del objeto en una
imágen, llegando a eliminar algunos detalles que se consideren mínimos o
ajenos al objeto que conforma la mayor parte de la composición de la
misma. La erosión genera un efecto contrario, el cual amplia los
detalles mínimos encontrados ajenos al agente principal de la imagen,
minimizando el objeto que tenga mayor presencia en la misma.\cite{Northwestern-University_2003}

Ambas operaciones se pueden ciclar e iterar en la forma que convenga
para obtener un efecto deseado en determinado tipo de imágenes.

En las operaciones de apertura y dilatación, la imagen es sujeta a
dilatación y erosión. La diferencia es el orden de estas operaciones. La
operación de apertura abre pequeñas brechas entre tocar objetos en una
imagen mientras la operación de cierre cierra pequeñas brechas en un
objeto.\cite{Northwestern-University_2003}

Se llama apertura si la dilatación es precedida por la erosión. Si la
dilatación es seguida de la erosión se llama cierre.

    \begin{tcolorbox}[breakable, size=fbox, boxrule=1pt, pad at break*=1mm,colback=cellbackground, colframe=cellborder]
\prompt{In}{incolor}{3}{\boxspacing}
\begin{Verbatim}[commandchars=\\\{\}]
\PY{k}{def} \PY{n+nf}{padding}\PY{p}{(}\PY{n}{originalImg}\PY{p}{,} \PY{n}{padSize}\PY{p}{)}\PY{p}{:}
    \PY{n}{padImg} \PY{o}{=} \PY{n}{np}\PY{o}{.}\PY{n}{zeros}\PY{p}{(}\PY{p}{(}\PY{n}{rows}\PY{o}{+}\PY{l+m+mi}{2}\PY{o}{*}\PY{n}{padSize}\PY{p}{,} \PY{n}{columns}\PY{o}{+}\PY{l+m+mi}{2}\PY{o}{*}\PY{n}{padSize}\PY{p}{)}\PY{p}{,} \PY{n}{dtype}\PY{o}{=}\PY{n}{np}\PY{o}{.}\PY{n}{uint8}\PY{p}{)}
    \PY{c+c1}{\PYZsh{} recortando}
    \PY{n}{padImg}\PY{p}{[}\PY{n}{padSize}\PY{p}{:}\PY{n}{rows}\PY{o}{+}\PY{n}{padSize}\PY{p}{,} \PY{n}{padSize}\PY{p}{:}\PY{n}{columns}\PY{o}{+}\PY{n}{padSize}\PY{p}{]} \PY{o}{=} \PY{n}{originalImg}
    \PY{k}{return} \PY{n}{padImg}
\end{Verbatim}
\end{tcolorbox}

    \begin{tcolorbox}[breakable, size=fbox, boxrule=1pt, pad at break*=1mm,colback=cellbackground, colframe=cellborder]
\prompt{In}{incolor}{4}{\boxspacing}
\begin{Verbatim}[commandchars=\\\{\}]
\PY{k}{def} \PY{n+nf}{Erosion}\PY{p}{(}\PY{n}{padImg}\PY{p}{,} \PY{n}{kernel}\PY{p}{,} \PY{n}{size}\PY{p}{)}\PY{p}{:}
    \PY{n}{output} \PY{o}{=} \PY{n}{np}\PY{o}{.}\PY{n}{zeros}\PY{p}{(}\PY{p}{(}\PY{n}{rows}\PY{p}{,} \PY{n}{columns}\PY{p}{)}\PY{p}{,} \PY{n}{dtype}\PY{o}{=}\PY{n}{np}\PY{o}{.}\PY{n}{uint8}\PY{p}{)}
    \PY{k}{for} \PY{n}{i} \PY{o+ow}{in} \PY{n+nb}{range}\PY{p}{(}\PY{l+m+mi}{0}\PY{p}{,} \PY{n}{rows}\PY{p}{)}\PY{p}{:}
        \PY{k}{for} \PY{n}{j} \PY{o+ow}{in} \PY{n+nb}{range}\PY{p}{(}\PY{l+m+mi}{0}\PY{p}{,} \PY{n}{columns}\PY{p}{)}\PY{p}{:}
            \PY{c+c1}{\PYZsh{} recortando}
            \PY{n}{portion} \PY{o}{=} \PY{n}{padImg}\PY{p}{[}\PY{n}{i}\PY{p}{:}\PY{n}{i}\PY{o}{+}\PY{n}{size}\PY{p}{,} \PY{n}{j}\PY{p}{:}\PY{n}{j}\PY{o}{+}\PY{n}{size}\PY{p}{]}
            \PY{c+c1}{\PYZsh{} se suma el elemento estructural y la ventana}
            \PY{n}{portion1} \PY{o}{=} \PY{n}{portion}\PY{o}{.}\PY{n}{flatten}\PY{p}{(}\PY{p}{)}
            \PY{n}{portion2} \PY{o}{=} \PY{n}{kernel}\PY{o}{.}\PY{n}{flatten}\PY{p}{(}\PY{p}{)}
            \PY{n}{p1} \PY{o}{=} \PY{p}{(}\PY{n}{np}\PY{o}{.}\PY{n}{sum}\PY{p}{(}\PY{n}{portion1}\PY{p}{)}\PY{p}{)}
            \PY{n}{p2} \PY{o}{=} \PY{p}{(}\PY{n}{np}\PY{o}{.}\PY{n}{sum}\PY{p}{(}\PY{n}{portion2}\PY{p}{)}\PY{p}{)}\PY{o}{*}\PY{l+m+mi}{255}
            \PY{c+c1}{\PYZsh{} la condicional para que no revase el limite}
            \PY{k}{if} \PY{n}{p1} \PY{o}{==} \PY{n}{p2}\PY{p}{:}
                \PY{n}{output}\PY{p}{[}\PY{n}{i}\PY{p}{,} \PY{n}{j}\PY{p}{]} \PY{o}{=} \PY{l+m+mi}{255}
            \PY{k}{else}\PY{p}{:}
                \PY{n}{output}\PY{p}{[}\PY{n}{i}\PY{p}{,} \PY{n}{j}\PY{p}{]} \PY{o}{=} \PY{n}{np}\PY{o}{.}\PY{n}{min}\PY{p}{(}\PY{n}{portion1}\PY{p}{)}
    \PY{k}{return} \PY{n}{output}
\end{Verbatim}
\end{tcolorbox}

    \begin{tcolorbox}[breakable, size=fbox, boxrule=1pt, pad at break*=1mm,colback=cellbackground, colframe=cellborder]
\prompt{In}{incolor}{5}{\boxspacing}
\begin{Verbatim}[commandchars=\\\{\}]
\PY{k}{def} \PY{n+nf}{Dilatacion}\PY{p}{(}\PY{n}{padImg}\PY{p}{,} \PY{n}{size}\PY{p}{)}\PY{p}{:}
    \PY{n}{output} \PY{o}{=} \PY{n}{np}\PY{o}{.}\PY{n}{zeros}\PY{p}{(}\PY{p}{(}\PY{n}{rows}\PY{p}{,} \PY{n}{columns}\PY{p}{)}\PY{p}{,} \PY{n}{dtype}\PY{o}{=}\PY{n}{np}\PY{o}{.}\PY{n}{uint8}\PY{p}{)}
    \PY{k}{for} \PY{n}{i} \PY{o+ow}{in} \PY{n+nb}{range}\PY{p}{(}\PY{l+m+mi}{0}\PY{p}{,} \PY{n}{rows}\PY{p}{)}\PY{p}{:}
        \PY{k}{for} \PY{n}{j} \PY{o+ow}{in} \PY{n+nb}{range}\PY{p}{(}\PY{l+m+mi}{0}\PY{p}{,} \PY{n}{columns}\PY{p}{)}\PY{p}{:}
            \PY{c+c1}{\PYZsh{} recortando}
            \PY{n}{portion} \PY{o}{=} \PY{n}{padImg}\PY{p}{[}\PY{n}{i}\PY{p}{:}\PY{n}{i}\PY{o}{+}\PY{n}{size}\PY{p}{,} \PY{n}{j}\PY{p}{:}\PY{n}{j}\PY{o}{+}\PY{n}{size}\PY{p}{]}
            \PY{n}{portion1} \PY{o}{=} \PY{n}{portion}\PY{o}{.}\PY{n}{flatten}\PY{p}{(}\PY{p}{)}
            \PY{c+c1}{\PYZsh{} la condicional para que no revase el limite}
            \PY{k}{if} \PY{l+m+mi}{255} \PY{o+ow}{in} \PY{n}{portion1}\PY{p}{:}
                \PY{n}{output}\PY{p}{[}\PY{n}{i}\PY{p}{,} \PY{n}{j}\PY{p}{]} \PY{o}{=} \PY{l+m+mi}{255}
            \PY{k}{else}\PY{p}{:}
                \PY{n}{output}\PY{p}{[}\PY{n}{i}\PY{p}{,} \PY{n}{j}\PY{p}{]} \PY{o}{=} \PY{n}{np}\PY{o}{.}\PY{n}{max}\PY{p}{(}\PY{n}{portion1}\PY{p}{)}
    \PY{k}{return} \PY{n}{output}
\end{Verbatim}
\end{tcolorbox}

    \begin{tcolorbox}[breakable, size=fbox, boxrule=1pt, pad at break*=1mm,colback=cellbackground, colframe=cellborder]
\prompt{In}{incolor}{6}{\boxspacing}
\begin{Verbatim}[commandchars=\\\{\}]
\PY{k}{def} \PY{n+nf}{opening}\PY{p}{(}\PY{n}{padImg}\PY{p}{,} \PY{n}{kernel}\PY{p}{,} \PY{n}{size}\PY{p}{)}\PY{p}{:}
    \PY{c+c1}{\PYZsh{} Se aplica la erosion}
    \PY{n}{erosion} \PY{o}{=} \PY{n}{Erosion}\PY{p}{(}\PY{n}{padImg}\PY{p}{,} \PY{n}{kernel} \PY{p}{,} \PY{n}{size}\PY{p}{)}
    \PY{n}{padImg2} \PY{o}{=} \PY{n}{padding}\PY{p}{(}\PY{n}{erosion}\PY{p}{,} \PY{n}{size}\PY{o}{/}\PY{o}{/}\PY{l+m+mi}{2}\PY{p}{)}
    \PY{c+c1}{\PYZsh{} Se aplica al dilatacion}
    \PY{n}{output} \PY{o}{=} \PY{n}{Dilatacion}\PY{p}{(}\PY{n}{padImg2}\PY{p}{,} \PY{n}{size}\PY{p}{)}
    \PY{k}{return} \PY{n}{output}
\end{Verbatim}
\end{tcolorbox}

    \begin{tcolorbox}[breakable, size=fbox, boxrule=1pt, pad at break*=1mm,colback=cellbackground, colframe=cellborder]
\prompt{In}{incolor}{7}{\boxspacing}
\begin{Verbatim}[commandchars=\\\{\}]
\PY{k}{def} \PY{n+nf}{closing}\PY{p}{(}\PY{n}{padImg}\PY{p}{,}\PY{n}{kernel}\PY{p}{,} \PY{n}{size}\PY{p}{)}\PY{p}{:}
    \PY{c+c1}{\PYZsh{} Se aplica al dilatacion}
    \PY{n}{dilation} \PY{o}{=} \PY{n}{Dilatacion}\PY{p}{(}\PY{n}{padImg}\PY{p}{,} \PY{n}{size}\PY{p}{)}
    \PY{n}{padImg2} \PY{o}{=} \PY{n}{padding}\PY{p}{(}\PY{n}{dilation}\PY{p}{,} \PY{n}{size}\PY{o}{/}\PY{o}{/}\PY{l+m+mi}{2}\PY{p}{)}
    \PY{c+c1}{\PYZsh{} Se aplica la erosion}
    \PY{n}{output} \PY{o}{=} \PY{n}{Erosion}\PY{p}{(}\PY{n}{padImg2}\PY{p}{,} \PY{n}{kernel}\PY{p}{,} \PY{n}{size}\PY{p}{)}
    \PY{k}{return} \PY{n}{output}
\end{Verbatim}
\end{tcolorbox}

    \hypertarget{uso}{%
\subsection*{Uso}\label{uso}}

\hypertarget{definiciuxf3n-de-paruxe1metros}{%
\subsubsection*{Definición de
parámetros}\label{definiciuxf3n-de-paruxe1metros}}

    \begin{tcolorbox}[breakable, size=fbox, boxrule=1pt, pad at break*=1mm,colback=cellbackground, colframe=cellborder]
\prompt{In}{incolor}{8}{\boxspacing}
\begin{Verbatim}[commandchars=\\\{\}]
\PY{c+c1}{\PYZsh{}Numero de elementos de la mascara a utilizar}
\PY{n}{size} \PY{o}{=} \PY{l+m+mi}{9}
\PY{c+c1}{\PYZsh{}Elemento estructural}
\PY{n}{kernel} \PY{o}{=} \PY{n}{np}\PY{o}{.}\PY{n}{ones}\PY{p}{(}\PY{p}{(}\PY{n}{size}\PY{p}{,} \PY{n}{size}\PY{p}{)}\PY{p}{,} \PY{n}{np}\PY{o}{.}\PY{n}{uint8}\PY{p}{)}
\PY{c+c1}{\PYZsh{}Tamaño del padding}
\PY{n}{p\PYZus{}size} \PY{o}{=} \PY{n}{size}\PY{o}{/}\PY{o}{/}\PY{l+m+mi}{2}
\end{Verbatim}
\end{tcolorbox}

    \hypertarget{operaciones}{%
\subsubsection*{Operaciones}\label{operaciones}}

    \begin{tcolorbox}[breakable, size=fbox, boxrule=1pt, pad at break*=1mm,colback=cellbackground, colframe=cellborder]
\prompt{In}{incolor}{10}{\boxspacing}
\begin{Verbatim}[commandchars=\\\{\}]
\PY{c+c1}{\PYZsh{} Selección de la imagen a utilizar, es posible cambiarla por otras}
\PY{n}{orginalImg} \PY{o}{=} \PY{n}{img\PYZus{}0}
\PY{c+c1}{\PYZsh{}Se obtiene los tamaños de la imagen original}
\PY{n}{rows} \PY{o}{=} \PY{n}{orginalImg}\PY{o}{.}\PY{n}{shape}\PY{p}{[}\PY{l+m+mi}{0}\PY{p}{]}
\PY{n}{columns} \PY{o}{=} \PY{n}{orginalImg}\PY{o}{.}\PY{n}{shape}\PY{p}{[}\PY{l+m+mi}{1}\PY{p}{]}
\PY{c+c1}{\PYZsh{}Se obtiene el padding inicial}
\PY{n}{padImg} \PY{o}{=} \PY{n}{padding}\PY{p}{(}\PY{n}{orginalImg}\PY{p}{,} \PY{n}{p\PYZus{}size}\PY{p}{)}
\PY{c+c1}{\PYZsh{}Aplicamos las operaciones morfológicas}
\PY{n}{Dil} \PY{o}{=} \PY{n}{Dilatacion}\PY{p}{(}\PY{n}{padImg}\PY{p}{,} \PY{n}{size}\PY{p}{)}
\PY{n}{Ero} \PY{o}{=} \PY{n}{Erosion}\PY{p}{(}\PY{n}{padImg}\PY{p}{,} \PY{n}{kernel}\PY{p}{,} \PY{n}{size}\PY{p}{)}
\PY{n}{Clo} \PY{o}{=} \PY{n}{closing}\PY{p}{(}\PY{n}{padImg}\PY{p}{,} \PY{n}{kernel}\PY{p}{,} \PY{n}{size}\PY{p}{)}
\PY{n}{Opn} \PY{o}{=} \PY{n}{opening}\PY{p}{(}\PY{n}{padImg}\PY{p}{,} \PY{n}{kernel}\PY{p}{,} \PY{n}{size}\PY{p}{)}
\end{Verbatim}
\end{tcolorbox}

    \begin{tcolorbox}[breakable, size=fbox, boxrule=1pt, pad at break*=1mm,colback=cellbackground, colframe=cellborder]
\prompt{In}{incolor}{11}{\boxspacing}
\begin{Verbatim}[commandchars=\\\{\}]
\PY{n}{orginalImg} \PY{o}{=} \PY{n}{cv}\PY{o}{.}\PY{n}{cvtColor}\PY{p}{(}\PY{n}{orginalImg}\PY{p}{,} \PY{n}{cv}\PY{o}{.}\PY{n}{COLOR\PYZus{}BGR2RGB}\PY{p}{)}
\PY{n}{Ero} \PY{o}{=} \PY{n}{cv}\PY{o}{.}\PY{n}{cvtColor}\PY{p}{(}\PY{n}{Ero}\PY{p}{,} \PY{n}{cv}\PY{o}{.}\PY{n}{COLOR\PYZus{}BGR2RGB}\PY{p}{)}
\PY{n}{Dil} \PY{o}{=} \PY{n}{cv}\PY{o}{.}\PY{n}{cvtColor}\PY{p}{(}\PY{n}{Dil}\PY{p}{,} \PY{n}{cv}\PY{o}{.}\PY{n}{COLOR\PYZus{}BGR2RGB}\PY{p}{)}
\PY{n}{Clo} \PY{o}{=} \PY{n}{cv}\PY{o}{.}\PY{n}{cvtColor}\PY{p}{(}\PY{n}{Clo}\PY{p}{,} \PY{n}{cv}\PY{o}{.}\PY{n}{COLOR\PYZus{}BGR2RGB}\PY{p}{)}
\PY{n}{Opn} \PY{o}{=} \PY{n}{cv}\PY{o}{.}\PY{n}{cvtColor}\PY{p}{(}\PY{n}{Opn}\PY{p}{,} \PY{n}{cv}\PY{o}{.}\PY{n}{COLOR\PYZus{}BGR2RGB}\PY{p}{)}
\PY{n}{plt}\PY{o}{.}\PY{n}{subplot}\PY{p}{(}\PY{l+m+mi}{231}\PY{p}{)}
\PY{n}{plt}\PY{o}{.}\PY{n}{imshow}\PY{p}{(}\PY{n}{orginalImg}\PY{p}{)}
\PY{n}{plt}\PY{o}{.}\PY{n}{title}\PY{p}{(}\PY{l+s+s1}{\PYZsq{}}\PY{l+s+s1}{Imagen Original}\PY{l+s+s1}{\PYZsq{}}\PY{p}{)}\PY{p}{,} \PY{n}{plt}\PY{o}{.}\PY{n}{xticks}\PY{p}{(}\PY{p}{[}\PY{p}{]}\PY{p}{)}\PY{p}{,} \PY{n}{plt}\PY{o}{.}\PY{n}{yticks}\PY{p}{(}\PY{p}{[}\PY{p}{]}\PY{p}{)}
\PY{n}{plt}\PY{o}{.}\PY{n}{subplot}\PY{p}{(}\PY{l+m+mi}{232}\PY{p}{)}
\PY{n}{plt}\PY{o}{.}\PY{n}{imshow}\PY{p}{(}\PY{n}{Ero}\PY{p}{)}
\PY{n}{plt}\PY{o}{.}\PY{n}{title}\PY{p}{(}\PY{l+s+s1}{\PYZsq{}}\PY{l+s+s1}{Erosión}\PY{l+s+s1}{\PYZsq{}}\PY{p}{)}\PY{p}{,} \PY{n}{plt}\PY{o}{.}\PY{n}{xticks}\PY{p}{(}\PY{p}{[}\PY{p}{]}\PY{p}{)}\PY{p}{,} \PY{n}{plt}\PY{o}{.}\PY{n}{yticks}\PY{p}{(}\PY{p}{[}\PY{p}{]}\PY{p}{)}
\PY{n}{plt}\PY{o}{.}\PY{n}{subplot}\PY{p}{(}\PY{l+m+mi}{233}\PY{p}{)}
\PY{n}{plt}\PY{o}{.}\PY{n}{imshow}\PY{p}{(}\PY{n}{Dil}\PY{p}{)}
\PY{n}{plt}\PY{o}{.}\PY{n}{title}\PY{p}{(}\PY{l+s+s1}{\PYZsq{}}\PY{l+s+s1}{Dilatación}\PY{l+s+s1}{\PYZsq{}}\PY{p}{)}\PY{p}{,} \PY{n}{plt}\PY{o}{.}\PY{n}{xticks}\PY{p}{(}\PY{p}{[}\PY{p}{]}\PY{p}{)}\PY{p}{,} \PY{n}{plt}\PY{o}{.}\PY{n}{yticks}\PY{p}{(}\PY{p}{[}\PY{p}{]}\PY{p}{)}
\PY{n}{plt}\PY{o}{.}\PY{n}{subplot}\PY{p}{(}\PY{l+m+mi}{234}\PY{p}{)}
\PY{n}{plt}\PY{o}{.}\PY{n}{imshow}\PY{p}{(}\PY{n}{Clo}\PY{p}{)}
\PY{n}{plt}\PY{o}{.}\PY{n}{title}\PY{p}{(}\PY{l+s+s1}{\PYZsq{}}\PY{l+s+s1}{Cierre}\PY{l+s+s1}{\PYZsq{}}\PY{p}{)}\PY{p}{,} \PY{n}{plt}\PY{o}{.}\PY{n}{xticks}\PY{p}{(}\PY{p}{[}\PY{p}{]}\PY{p}{)}\PY{p}{,} \PY{n}{plt}\PY{o}{.}\PY{n}{yticks}\PY{p}{(}\PY{p}{[}\PY{p}{]}\PY{p}{)}
\PY{n}{plt}\PY{o}{.}\PY{n}{subplot}\PY{p}{(}\PY{l+m+mi}{236}\PY{p}{)}
\PY{n}{plt}\PY{o}{.}\PY{n}{imshow}\PY{p}{(}\PY{n}{Opn}\PY{p}{)}
\PY{n}{plt}\PY{o}{.}\PY{n}{title}\PY{p}{(}\PY{l+s+s1}{\PYZsq{}}\PY{l+s+s1}{Apertura}\PY{l+s+s1}{\PYZsq{}}\PY{p}{)}\PY{p}{,} \PY{n}{plt}\PY{o}{.}\PY{n}{xticks}\PY{p}{(}\PY{p}{[}\PY{p}{]}\PY{p}{)}\PY{p}{,} \PY{n}{plt}\PY{o}{.}\PY{n}{yticks}\PY{p}{(}\PY{p}{[}\PY{p}{]}\PY{p}{)}
\PY{n}{plt}\PY{o}{.}\PY{n}{show}\PY{p}{(}\PY{p}{)}
\end{Verbatim}
\end{tcolorbox}

    \begin{center}
    \adjustimage{max size={0.9\linewidth}{0.9\paperheight}}{output_15_0.png}
    \end{center}
    { \hspace*{\fill} \\}
    
    \hypertarget{aplicaciuxf3n}{%
\subsection*{Aplicación}\label{aplicaciuxf3n}}

Una vez que se ha observado el efecto sobre una imagen de control, se
aplicaron los filtros a las imágenes de Rayos X de interés. Sin embargo,
solo se obtuvieron resultados parcialmente favorables. Las operaciones
eliminaron elementos que no son de interés pero no permitian ver con
claridad la profundidad de otros elementos.

    \begin{tcolorbox}[breakable, size=fbox, boxrule=1pt, pad at break*=1mm,colback=cellbackground, colframe=cellborder]
\prompt{In}{incolor}{12}{\boxspacing}
\begin{Verbatim}[commandchars=\\\{\}]
\PY{c+c1}{\PYZsh{} Selección de la imagen a utilizar, es posible cambiarla por otras}
\PY{n}{orginalImg} \PY{o}{=} \PY{n}{img\PYZus{}2}
\PY{c+c1}{\PYZsh{}Se obtiene los tamaños de la imagen original}
\PY{n}{rows} \PY{o}{=} \PY{n}{orginalImg}\PY{o}{.}\PY{n}{shape}\PY{p}{[}\PY{l+m+mi}{0}\PY{p}{]}
\PY{n}{columns} \PY{o}{=} \PY{n}{orginalImg}\PY{o}{.}\PY{n}{shape}\PY{p}{[}\PY{l+m+mi}{1}\PY{p}{]}
\PY{c+c1}{\PYZsh{}Se obtiene el padding inicial}
\PY{n}{padImg} \PY{o}{=} \PY{n}{padding}\PY{p}{(}\PY{n}{orginalImg}\PY{p}{,} \PY{n}{p\PYZus{}size}\PY{p}{)}
\PY{c+c1}{\PYZsh{}Aplicamos las operaciones morfológicas}
\PY{n}{Dil} \PY{o}{=} \PY{n}{Dilatacion}\PY{p}{(}\PY{n}{padImg}\PY{p}{,} \PY{n}{size}\PY{p}{)}
\PY{n}{Ero} \PY{o}{=} \PY{n}{Erosion}\PY{p}{(}\PY{n}{padImg}\PY{p}{,} \PY{n}{kernel}\PY{p}{,} \PY{n}{size}\PY{p}{)}
\PY{n}{Clo} \PY{o}{=} \PY{n}{closing}\PY{p}{(}\PY{n}{padImg}\PY{p}{,} \PY{n}{kernel}\PY{p}{,} \PY{n}{size}\PY{p}{)}
\PY{n}{Opn} \PY{o}{=} \PY{n}{opening}\PY{p}{(}\PY{n}{padImg}\PY{p}{,} \PY{n}{kernel}\PY{p}{,} \PY{n}{size}\PY{p}{)}
\end{Verbatim}
\end{tcolorbox}

    \begin{tcolorbox}[breakable, size=fbox, boxrule=1pt, pad at break*=1mm,colback=cellbackground, colframe=cellborder]
\prompt{In}{incolor}{13}{\boxspacing}
\begin{Verbatim}[commandchars=\\\{\}]
\PY{n}{orginalImg} \PY{o}{=} \PY{n}{cv}\PY{o}{.}\PY{n}{cvtColor}\PY{p}{(}\PY{n}{orginalImg}\PY{p}{,} \PY{n}{cv}\PY{o}{.}\PY{n}{COLOR\PYZus{}BGR2RGB}\PY{p}{)}
\PY{n}{Ero} \PY{o}{=} \PY{n}{cv}\PY{o}{.}\PY{n}{cvtColor}\PY{p}{(}\PY{n}{Ero}\PY{p}{,} \PY{n}{cv}\PY{o}{.}\PY{n}{COLOR\PYZus{}BGR2RGB}\PY{p}{)}
\PY{n}{Dil} \PY{o}{=} \PY{n}{cv}\PY{o}{.}\PY{n}{cvtColor}\PY{p}{(}\PY{n}{Dil}\PY{p}{,} \PY{n}{cv}\PY{o}{.}\PY{n}{COLOR\PYZus{}BGR2RGB}\PY{p}{)}
\PY{n}{Clo} \PY{o}{=} \PY{n}{cv}\PY{o}{.}\PY{n}{cvtColor}\PY{p}{(}\PY{n}{Clo}\PY{p}{,} \PY{n}{cv}\PY{o}{.}\PY{n}{COLOR\PYZus{}BGR2RGB}\PY{p}{)}
\PY{n}{Opn} \PY{o}{=} \PY{n}{cv}\PY{o}{.}\PY{n}{cvtColor}\PY{p}{(}\PY{n}{Opn}\PY{p}{,} \PY{n}{cv}\PY{o}{.}\PY{n}{COLOR\PYZus{}BGR2RGB}\PY{p}{)}
\PY{n}{plt}\PY{o}{.}\PY{n}{subplot}\PY{p}{(}\PY{l+m+mi}{231}\PY{p}{)}
\PY{n}{plt}\PY{o}{.}\PY{n}{imshow}\PY{p}{(}\PY{n}{orginalImg}\PY{p}{)}
\PY{n}{plt}\PY{o}{.}\PY{n}{title}\PY{p}{(}\PY{l+s+s1}{\PYZsq{}}\PY{l+s+s1}{Imagen Original}\PY{l+s+s1}{\PYZsq{}}\PY{p}{)}\PY{p}{,} \PY{n}{plt}\PY{o}{.}\PY{n}{xticks}\PY{p}{(}\PY{p}{[}\PY{p}{]}\PY{p}{)}\PY{p}{,} \PY{n}{plt}\PY{o}{.}\PY{n}{yticks}\PY{p}{(}\PY{p}{[}\PY{p}{]}\PY{p}{)}
\PY{n}{plt}\PY{o}{.}\PY{n}{subplot}\PY{p}{(}\PY{l+m+mi}{232}\PY{p}{)}
\PY{n}{plt}\PY{o}{.}\PY{n}{imshow}\PY{p}{(}\PY{n}{Ero}\PY{p}{)}
\PY{n}{plt}\PY{o}{.}\PY{n}{title}\PY{p}{(}\PY{l+s+s1}{\PYZsq{}}\PY{l+s+s1}{Erosión}\PY{l+s+s1}{\PYZsq{}}\PY{p}{)}\PY{p}{,} \PY{n}{plt}\PY{o}{.}\PY{n}{xticks}\PY{p}{(}\PY{p}{[}\PY{p}{]}\PY{p}{)}\PY{p}{,} \PY{n}{plt}\PY{o}{.}\PY{n}{yticks}\PY{p}{(}\PY{p}{[}\PY{p}{]}\PY{p}{)}
\PY{n}{plt}\PY{o}{.}\PY{n}{subplot}\PY{p}{(}\PY{l+m+mi}{233}\PY{p}{)}
\PY{n}{plt}\PY{o}{.}\PY{n}{imshow}\PY{p}{(}\PY{n}{Dil}\PY{p}{)}
\PY{n}{plt}\PY{o}{.}\PY{n}{title}\PY{p}{(}\PY{l+s+s1}{\PYZsq{}}\PY{l+s+s1}{Dilatación}\PY{l+s+s1}{\PYZsq{}}\PY{p}{)}\PY{p}{,} \PY{n}{plt}\PY{o}{.}\PY{n}{xticks}\PY{p}{(}\PY{p}{[}\PY{p}{]}\PY{p}{)}\PY{p}{,} \PY{n}{plt}\PY{o}{.}\PY{n}{yticks}\PY{p}{(}\PY{p}{[}\PY{p}{]}\PY{p}{)}
\PY{n}{plt}\PY{o}{.}\PY{n}{subplot}\PY{p}{(}\PY{l+m+mi}{234}\PY{p}{)}
\PY{n}{plt}\PY{o}{.}\PY{n}{imshow}\PY{p}{(}\PY{n}{Clo}\PY{p}{)}
\PY{n}{plt}\PY{o}{.}\PY{n}{title}\PY{p}{(}\PY{l+s+s1}{\PYZsq{}}\PY{l+s+s1}{Cierre}\PY{l+s+s1}{\PYZsq{}}\PY{p}{)}\PY{p}{,} \PY{n}{plt}\PY{o}{.}\PY{n}{xticks}\PY{p}{(}\PY{p}{[}\PY{p}{]}\PY{p}{)}\PY{p}{,} \PY{n}{plt}\PY{o}{.}\PY{n}{yticks}\PY{p}{(}\PY{p}{[}\PY{p}{]}\PY{p}{)}
\PY{n}{plt}\PY{o}{.}\PY{n}{subplot}\PY{p}{(}\PY{l+m+mi}{236}\PY{p}{)}
\PY{n}{plt}\PY{o}{.}\PY{n}{imshow}\PY{p}{(}\PY{n}{Opn}\PY{p}{)}
\PY{n}{plt}\PY{o}{.}\PY{n}{title}\PY{p}{(}\PY{l+s+s1}{\PYZsq{}}\PY{l+s+s1}{Apertura}\PY{l+s+s1}{\PYZsq{}}\PY{p}{)}\PY{p}{,} \PY{n}{plt}\PY{o}{.}\PY{n}{xticks}\PY{p}{(}\PY{p}{[}\PY{p}{]}\PY{p}{)}\PY{p}{,} \PY{n}{plt}\PY{o}{.}\PY{n}{yticks}\PY{p}{(}\PY{p}{[}\PY{p}{]}\PY{p}{)}
\PY{n}{plt}\PY{o}{.}\PY{n}{show}\PY{p}{(}\PY{p}{)}
\end{Verbatim}
\end{tcolorbox}

    \begin{center}
    \adjustimage{max size={0.9\linewidth}{0.9\paperheight}}{output_18_0.png}
    \end{center}
    { \hspace*{\fill} \\}
    
    \hypertarget{gradiante-morfoluxf3gico}{%
\subsubsection*{Gradiante morfológico}\label{gradiante-morfoluxf3gico}}

El gradiante morfológico es resultado de restar la imagen erosionada a
la imágen dilatada. El resultado es una imagen usada en la segmentación
debido a que cada valor de píxel (por lo general no negativo) indica la
intensidad de contraste en la vecindad de ese píxel\cite{rivest1993morphological}. Con esta operación
se ha conseguido el efecto deseado.

    \begin{tcolorbox}[breakable, size=fbox, boxrule=1pt, pad at break*=1mm,colback=cellbackground, colframe=cellborder]
\prompt{In}{incolor}{14}{\boxspacing}
\begin{Verbatim}[commandchars=\\\{\}]
\PY{c+c1}{\PYZsh{} Selección de la imagen a utilizar, es posible cambiarla por otras}
\PY{n}{orginalImg} \PY{o}{=} \PY{n}{img\PYZus{}1}
\PY{c+c1}{\PYZsh{}Se obtiene los tamaños de la imagen original}
\PY{n}{rows} \PY{o}{=} \PY{n}{orginalImg}\PY{o}{.}\PY{n}{shape}\PY{p}{[}\PY{l+m+mi}{0}\PY{p}{]}
\PY{n}{columns} \PY{o}{=} \PY{n}{orginalImg}\PY{o}{.}\PY{n}{shape}\PY{p}{[}\PY{l+m+mi}{1}\PY{p}{]}
\PY{c+c1}{\PYZsh{}Se obtiene el padding inicial}
\PY{n}{padImg} \PY{o}{=} \PY{n}{padding}\PY{p}{(}\PY{n}{orginalImg}\PY{p}{,} \PY{n}{p\PYZus{}size}\PY{p}{)}
\PY{c+c1}{\PYZsh{}Aplicamos las operaciones morfológicas}
\PY{n}{Dil} \PY{o}{=} \PY{n}{Dilatacion}\PY{p}{(}\PY{n}{padImg}\PY{p}{,} \PY{n}{size}\PY{p}{)}
\PY{n}{Ero} \PY{o}{=} \PY{n}{Erosion}\PY{p}{(}\PY{n}{padImg}\PY{p}{,} \PY{n}{kernel}\PY{p}{,} \PY{n}{size}\PY{p}{)}
\PY{n}{Gradiante} \PY{o}{=} \PY{n}{Dil} \PY{o}{\PYZhy{}} \PY{n}{Ero}
\end{Verbatim}
\end{tcolorbox}

    \begin{tcolorbox}[breakable, size=fbox, boxrule=1pt, pad at break*=1mm,colback=cellbackground, colframe=cellborder]
\prompt{In}{incolor}{15}{\boxspacing}
\begin{Verbatim}[commandchars=\\\{\}]
\PY{n}{orginalImg} \PY{o}{=} \PY{n}{cv}\PY{o}{.}\PY{n}{cvtColor}\PY{p}{(}\PY{n}{orginalImg}\PY{p}{,} \PY{n}{cv}\PY{o}{.}\PY{n}{COLOR\PYZus{}BGR2RGB}\PY{p}{)}
\PY{n}{Ero} \PY{o}{=} \PY{n}{cv}\PY{o}{.}\PY{n}{cvtColor}\PY{p}{(}\PY{n}{Ero}\PY{p}{,} \PY{n}{cv}\PY{o}{.}\PY{n}{COLOR\PYZus{}BGR2RGB}\PY{p}{)}
\PY{n}{Dil} \PY{o}{=} \PY{n}{cv}\PY{o}{.}\PY{n}{cvtColor}\PY{p}{(}\PY{n}{Dil}\PY{p}{,} \PY{n}{cv}\PY{o}{.}\PY{n}{COLOR\PYZus{}BGR2RGB}\PY{p}{)}
\PY{n}{Gradiante} \PY{o}{=} \PY{n}{cv}\PY{o}{.}\PY{n}{cvtColor}\PY{p}{(}\PY{n}{Gradiante}\PY{p}{,} \PY{n}{cv}\PY{o}{.}\PY{n}{COLOR\PYZus{}BGR2RGB}\PY{p}{)}
\PY{n}{plt}\PY{o}{.}\PY{n}{subplot}\PY{p}{(}\PY{l+m+mi}{121}\PY{p}{)}
\PY{n}{plt}\PY{o}{.}\PY{n}{imshow}\PY{p}{(}\PY{n}{orginalImg}\PY{p}{)}
\PY{n}{plt}\PY{o}{.}\PY{n}{title}\PY{p}{(}\PY{l+s+s1}{\PYZsq{}}\PY{l+s+s1}{Imagen Original}\PY{l+s+s1}{\PYZsq{}}\PY{p}{)}\PY{p}{,} \PY{n}{plt}\PY{o}{.}\PY{n}{xticks}\PY{p}{(}\PY{p}{[}\PY{p}{]}\PY{p}{)}\PY{p}{,} \PY{n}{plt}\PY{o}{.}\PY{n}{yticks}\PY{p}{(}\PY{p}{[}\PY{p}{]}\PY{p}{)}
\PY{n}{plt}\PY{o}{.}\PY{n}{subplot}\PY{p}{(}\PY{l+m+mi}{122}\PY{p}{)}
\PY{n}{plt}\PY{o}{.}\PY{n}{imshow}\PY{p}{(}\PY{n}{Gradiante}\PY{p}{)}
\PY{n}{plt}\PY{o}{.}\PY{n}{title}\PY{p}{(}\PY{l+s+s1}{\PYZsq{}}\PY{l+s+s1}{Gradiante}\PY{l+s+s1}{\PYZsq{}}\PY{p}{)}\PY{p}{,} \PY{n}{plt}\PY{o}{.}\PY{n}{xticks}\PY{p}{(}\PY{p}{[}\PY{p}{]}\PY{p}{)}\PY{p}{,} \PY{n}{plt}\PY{o}{.}\PY{n}{yticks}\PY{p}{(}\PY{p}{[}\PY{p}{]}\PY{p}{)}
\PY{n}{plt}\PY{o}{.}\PY{n}{show}\PY{p}{(}\PY{p}{)}
\end{Verbatim}
\end{tcolorbox}

    \begin{center}
    \adjustimage{max size={0.9\linewidth}{0.9\paperheight}}{output_21_0.png}
    \end{center}
    { \hspace*{\fill} \\}
    
    \hypertarget{conclusiuxf3n}{%
\subsection*{Conclusión}\label{conclusiuxf3n}}

Las operaciones morfológicas básicas son dos: erosión y dilatación.
Estas operaciones son se pueden combinar a conveniencia para obtener
operaciones secundarias, como la apertura y el cierre, que pueden ser
usados dependiendo del objetivo o necesidades del caso particular.

En la aplicación de los filtros puede verse que el gradiante morfológico
permite distingir qué segmentos se encuentran más cercanos de otros en
una imagen de rayos X, además de definir el área de dichos segmentos.
Esto es importante por el efecto de transparencia que llegan a tener las
imágenes de rayos X.


    % Add a bibliography block to the postdoc
    \bibliographystyle{ieeetr}
    \bibliography{main}
    
    
\end{document}
