\documentclass[11pt]{article}

    \usepackage[breakable]{tcolorbox}
    \usepackage{parskip} % Stop auto-indenting (to mimic markdown behaviour)
    
    \usepackage{iftex}
    \ifPDFTeX
    	\usepackage[T1]{fontenc}
    	\usepackage{mathpazo}
    \else
    	\usepackage{fontspec}
    \fi

    % Basic figure setup, for now with no caption control since it's done
    % automatically by Pandoc (which extracts ![](path) syntax from Markdown).
    \usepackage{graphicx}
    % Maintain compatibility with old templates. Remove in nbconvert 6.0
    \let\Oldincludegraphics\includegraphics
    % Ensure that by default, figures have no caption (until we provide a
    % proper Figure object with a Caption API and a way to capture that
    % in the conversion process - todo).
    \usepackage{caption}
    \DeclareCaptionFormat{nocaption}{}
    \captionsetup{format=nocaption,aboveskip=0pt,belowskip=0pt}

    \usepackage{float}
    \floatplacement{figure}{H} % forces figures to be placed at the correct location
    \usepackage{xcolor} % Allow colors to be defined
    \usepackage{enumerate} % Needed for markdown enumerations to work
    \usepackage{geometry} % Used to adjust the document margins
    \usepackage{amsmath} % Equations
    \usepackage{amssymb} % Equations
    \usepackage{textcomp} % defines textquotesingle
    % Hack from http://tex.stackexchange.com/a/47451/13684:
    \AtBeginDocument{%
        \def\PYZsq{\textquotesingle}% Upright quotes in Pygmentized code
    }
    \usepackage{upquote} % Upright quotes for verbatim code
    \usepackage{eurosym} % defines \euro
    \usepackage[mathletters]{ucs} % Extended unicode (utf-8) support
    \usepackage{fancyvrb} % verbatim replacement that allows latex
    \usepackage{grffile} % extends the file name processing of package graphics 
                         % to support a larger range
    \makeatletter % fix for old versions of grffile with XeLaTeX
    \@ifpackagelater{grffile}{2019/11/01}
    {
      % Do nothing on new versions
    }
    {
      \def\Gread@@xetex#1{%
        \IfFileExists{"\Gin@base".bb}%
        {\Gread@eps{\Gin@base.bb}}%
        {\Gread@@xetex@aux#1}%
      }
    }
    \makeatother
    \usepackage[Export]{adjustbox} % Used to constrain images to a maximum size
    \adjustboxset{max size={0.9\linewidth}{0.9\paperheight}}

    % The hyperref package gives us a pdf with properly built
    % internal navigation ('pdf bookmarks' for the table of contents,
    % internal cross-reference links, web links for URLs, etc.)
    \usepackage{hyperref}
    % The default LaTeX title has an obnoxious amount of whitespace. By default,
    % titling removes some of it. It also provides customization options.
    \usepackage{titling}
    \usepackage{longtable} % longtable support required by pandoc >1.10
    \usepackage{booktabs}  % table support for pandoc > 1.12.2
    \usepackage[inline]{enumitem} % IRkernel/repr support (it uses the enumerate* environment)
    \usepackage[normalem]{ulem} % ulem is needed to support strikethroughs (\sout)
                                % normalem makes italics be italics, not underlines
    \usepackage{mathrsfs}
    

    
    % Colors for the hyperref package
    \definecolor{urlcolor}{rgb}{0,.145,.698}
    \definecolor{linkcolor}{rgb}{.71,0.21,0.01}
    \definecolor{citecolor}{rgb}{.12,.54,.11}

    % ANSI colors
    \definecolor{ansi-black}{HTML}{3E424D}
    \definecolor{ansi-black-intense}{HTML}{282C36}
    \definecolor{ansi-red}{HTML}{E75C58}
    \definecolor{ansi-red-intense}{HTML}{B22B31}
    \definecolor{ansi-green}{HTML}{00A250}
    \definecolor{ansi-green-intense}{HTML}{007427}
    \definecolor{ansi-yellow}{HTML}{DDB62B}
    \definecolor{ansi-yellow-intense}{HTML}{B27D12}
    \definecolor{ansi-blue}{HTML}{208FFB}
    \definecolor{ansi-blue-intense}{HTML}{0065CA}
    \definecolor{ansi-magenta}{HTML}{D160C4}
    \definecolor{ansi-magenta-intense}{HTML}{A03196}
    \definecolor{ansi-cyan}{HTML}{60C6C8}
    \definecolor{ansi-cyan-intense}{HTML}{258F8F}
    \definecolor{ansi-white}{HTML}{C5C1B4}
    \definecolor{ansi-white-intense}{HTML}{A1A6B2}
    \definecolor{ansi-default-inverse-fg}{HTML}{FFFFFF}
    \definecolor{ansi-default-inverse-bg}{HTML}{000000}

    % common color for the border for error outputs.
    \definecolor{outerrorbackground}{HTML}{FFDFDF}

    % commands and environments needed by pandoc snippets
    % extracted from the output of `pandoc -s`
    \providecommand{\tightlist}{%
      \setlength{\itemsep}{0pt}\setlength{\parskip}{0pt}}
    \DefineVerbatimEnvironment{Highlighting}{Verbatim}{commandchars=\\\{\}}
    % Add ',fontsize=\small' for more characters per line
    \newenvironment{Shaded}{}{}
    \newcommand{\KeywordTok}[1]{\textcolor[rgb]{0.00,0.44,0.13}{\textbf{{#1}}}}
    \newcommand{\DataTypeTok}[1]{\textcolor[rgb]{0.56,0.13,0.00}{{#1}}}
    \newcommand{\DecValTok}[1]{\textcolor[rgb]{0.25,0.63,0.44}{{#1}}}
    \newcommand{\BaseNTok}[1]{\textcolor[rgb]{0.25,0.63,0.44}{{#1}}}
    \newcommand{\FloatTok}[1]{\textcolor[rgb]{0.25,0.63,0.44}{{#1}}}
    \newcommand{\CharTok}[1]{\textcolor[rgb]{0.25,0.44,0.63}{{#1}}}
    \newcommand{\StringTok}[1]{\textcolor[rgb]{0.25,0.44,0.63}{{#1}}}
    \newcommand{\CommentTok}[1]{\textcolor[rgb]{0.38,0.63,0.69}{\textit{{#1}}}}
    \newcommand{\OtherTok}[1]{\textcolor[rgb]{0.00,0.44,0.13}{{#1}}}
    \newcommand{\AlertTok}[1]{\textcolor[rgb]{1.00,0.00,0.00}{\textbf{{#1}}}}
    \newcommand{\FunctionTok}[1]{\textcolor[rgb]{0.02,0.16,0.49}{{#1}}}
    \newcommand{\RegionMarkerTok}[1]{{#1}}
    \newcommand{\ErrorTok}[1]{\textcolor[rgb]{1.00,0.00,0.00}{\textbf{{#1}}}}
    \newcommand{\NormalTok}[1]{{#1}}
    
    % Additional commands for more recent versions of Pandoc
    \newcommand{\ConstantTok}[1]{\textcolor[rgb]{0.53,0.00,0.00}{{#1}}}
    \newcommand{\SpecialCharTok}[1]{\textcolor[rgb]{0.25,0.44,0.63}{{#1}}}
    \newcommand{\VerbatimStringTok}[1]{\textcolor[rgb]{0.25,0.44,0.63}{{#1}}}
    \newcommand{\SpecialStringTok}[1]{\textcolor[rgb]{0.73,0.40,0.53}{{#1}}}
    \newcommand{\ImportTok}[1]{{#1}}
    \newcommand{\DocumentationTok}[1]{\textcolor[rgb]{0.73,0.13,0.13}{\textit{{#1}}}}
    \newcommand{\AnnotationTok}[1]{\textcolor[rgb]{0.38,0.63,0.69}{\textbf{\textit{{#1}}}}}
    \newcommand{\CommentVarTok}[1]{\textcolor[rgb]{0.38,0.63,0.69}{\textbf{\textit{{#1}}}}}
    \newcommand{\VariableTok}[1]{\textcolor[rgb]{0.10,0.09,0.49}{{#1}}}
    \newcommand{\ControlFlowTok}[1]{\textcolor[rgb]{0.00,0.44,0.13}{\textbf{{#1}}}}
    \newcommand{\OperatorTok}[1]{\textcolor[rgb]{0.40,0.40,0.40}{{#1}}}
    \newcommand{\BuiltInTok}[1]{{#1}}
    \newcommand{\ExtensionTok}[1]{{#1}}
    \newcommand{\PreprocessorTok}[1]{\textcolor[rgb]{0.74,0.48,0.00}{{#1}}}
    \newcommand{\AttributeTok}[1]{\textcolor[rgb]{0.49,0.56,0.16}{{#1}}}
    \newcommand{\InformationTok}[1]{\textcolor[rgb]{0.38,0.63,0.69}{\textbf{\textit{{#1}}}}}
    \newcommand{\WarningTok}[1]{\textcolor[rgb]{0.38,0.63,0.69}{\textbf{\textit{{#1}}}}}
    
    
    % Define a nice break command that doesn't care if a line doesn't already
    % exist.
    \def\br{\hspace*{\fill} \\* }
    % Math Jax compatibility definitions
    \def\gt{>}
    \def\lt{<}
    \let\Oldtex\TeX
    \let\Oldlatex\LaTeX
    \renewcommand{\TeX}{\textrm{\Oldtex}}
    \renewcommand{\LaTeX}{\textrm{\Oldlatex}}
    % Document parameters
    % Document title
    \title{main}
    
    
    
    
    
% Pygments definitions
\makeatletter
\def\PY@reset{\let\PY@it=\relax \let\PY@bf=\relax%
    \let\PY@ul=\relax \let\PY@tc=\relax%
    \let\PY@bc=\relax \let\PY@ff=\relax}
\def\PY@tok#1{\csname PY@tok@#1\endcsname}
\def\PY@toks#1+{\ifx\relax#1\empty\else%
    \PY@tok{#1}\expandafter\PY@toks\fi}
\def\PY@do#1{\PY@bc{\PY@tc{\PY@ul{%
    \PY@it{\PY@bf{\PY@ff{#1}}}}}}}
\def\PY#1#2{\PY@reset\PY@toks#1+\relax+\PY@do{#2}}

\@namedef{PY@tok@w}{\def\PY@tc##1{\textcolor[rgb]{0.73,0.73,0.73}{##1}}}
\@namedef{PY@tok@c}{\let\PY@it=\textit\def\PY@tc##1{\textcolor[rgb]{0.24,0.48,0.48}{##1}}}
\@namedef{PY@tok@cp}{\def\PY@tc##1{\textcolor[rgb]{0.61,0.40,0.00}{##1}}}
\@namedef{PY@tok@k}{\let\PY@bf=\textbf\def\PY@tc##1{\textcolor[rgb]{0.00,0.50,0.00}{##1}}}
\@namedef{PY@tok@kp}{\def\PY@tc##1{\textcolor[rgb]{0.00,0.50,0.00}{##1}}}
\@namedef{PY@tok@kt}{\def\PY@tc##1{\textcolor[rgb]{0.69,0.00,0.25}{##1}}}
\@namedef{PY@tok@o}{\def\PY@tc##1{\textcolor[rgb]{0.40,0.40,0.40}{##1}}}
\@namedef{PY@tok@ow}{\let\PY@bf=\textbf\def\PY@tc##1{\textcolor[rgb]{0.67,0.13,1.00}{##1}}}
\@namedef{PY@tok@nb}{\def\PY@tc##1{\textcolor[rgb]{0.00,0.50,0.00}{##1}}}
\@namedef{PY@tok@nf}{\def\PY@tc##1{\textcolor[rgb]{0.00,0.00,1.00}{##1}}}
\@namedef{PY@tok@nc}{\let\PY@bf=\textbf\def\PY@tc##1{\textcolor[rgb]{0.00,0.00,1.00}{##1}}}
\@namedef{PY@tok@nn}{\let\PY@bf=\textbf\def\PY@tc##1{\textcolor[rgb]{0.00,0.00,1.00}{##1}}}
\@namedef{PY@tok@ne}{\let\PY@bf=\textbf\def\PY@tc##1{\textcolor[rgb]{0.80,0.25,0.22}{##1}}}
\@namedef{PY@tok@nv}{\def\PY@tc##1{\textcolor[rgb]{0.10,0.09,0.49}{##1}}}
\@namedef{PY@tok@no}{\def\PY@tc##1{\textcolor[rgb]{0.53,0.00,0.00}{##1}}}
\@namedef{PY@tok@nl}{\def\PY@tc##1{\textcolor[rgb]{0.46,0.46,0.00}{##1}}}
\@namedef{PY@tok@ni}{\let\PY@bf=\textbf\def\PY@tc##1{\textcolor[rgb]{0.44,0.44,0.44}{##1}}}
\@namedef{PY@tok@na}{\def\PY@tc##1{\textcolor[rgb]{0.41,0.47,0.13}{##1}}}
\@namedef{PY@tok@nt}{\let\PY@bf=\textbf\def\PY@tc##1{\textcolor[rgb]{0.00,0.50,0.00}{##1}}}
\@namedef{PY@tok@nd}{\def\PY@tc##1{\textcolor[rgb]{0.67,0.13,1.00}{##1}}}
\@namedef{PY@tok@s}{\def\PY@tc##1{\textcolor[rgb]{0.73,0.13,0.13}{##1}}}
\@namedef{PY@tok@sd}{\let\PY@it=\textit\def\PY@tc##1{\textcolor[rgb]{0.73,0.13,0.13}{##1}}}
\@namedef{PY@tok@si}{\let\PY@bf=\textbf\def\PY@tc##1{\textcolor[rgb]{0.64,0.35,0.47}{##1}}}
\@namedef{PY@tok@se}{\let\PY@bf=\textbf\def\PY@tc##1{\textcolor[rgb]{0.67,0.36,0.12}{##1}}}
\@namedef{PY@tok@sr}{\def\PY@tc##1{\textcolor[rgb]{0.64,0.35,0.47}{##1}}}
\@namedef{PY@tok@ss}{\def\PY@tc##1{\textcolor[rgb]{0.10,0.09,0.49}{##1}}}
\@namedef{PY@tok@sx}{\def\PY@tc##1{\textcolor[rgb]{0.00,0.50,0.00}{##1}}}
\@namedef{PY@tok@m}{\def\PY@tc##1{\textcolor[rgb]{0.40,0.40,0.40}{##1}}}
\@namedef{PY@tok@gh}{\let\PY@bf=\textbf\def\PY@tc##1{\textcolor[rgb]{0.00,0.00,0.50}{##1}}}
\@namedef{PY@tok@gu}{\let\PY@bf=\textbf\def\PY@tc##1{\textcolor[rgb]{0.50,0.00,0.50}{##1}}}
\@namedef{PY@tok@gd}{\def\PY@tc##1{\textcolor[rgb]{0.63,0.00,0.00}{##1}}}
\@namedef{PY@tok@gi}{\def\PY@tc##1{\textcolor[rgb]{0.00,0.52,0.00}{##1}}}
\@namedef{PY@tok@gr}{\def\PY@tc##1{\textcolor[rgb]{0.89,0.00,0.00}{##1}}}
\@namedef{PY@tok@ge}{\let\PY@it=\textit}
\@namedef{PY@tok@gs}{\let\PY@bf=\textbf}
\@namedef{PY@tok@gp}{\let\PY@bf=\textbf\def\PY@tc##1{\textcolor[rgb]{0.00,0.00,0.50}{##1}}}
\@namedef{PY@tok@go}{\def\PY@tc##1{\textcolor[rgb]{0.44,0.44,0.44}{##1}}}
\@namedef{PY@tok@gt}{\def\PY@tc##1{\textcolor[rgb]{0.00,0.27,0.87}{##1}}}
\@namedef{PY@tok@err}{\def\PY@bc##1{{\setlength{\fboxsep}{\string -\fboxrule}\fcolorbox[rgb]{1.00,0.00,0.00}{1,1,1}{\strut ##1}}}}
\@namedef{PY@tok@kc}{\let\PY@bf=\textbf\def\PY@tc##1{\textcolor[rgb]{0.00,0.50,0.00}{##1}}}
\@namedef{PY@tok@kd}{\let\PY@bf=\textbf\def\PY@tc##1{\textcolor[rgb]{0.00,0.50,0.00}{##1}}}
\@namedef{PY@tok@kn}{\let\PY@bf=\textbf\def\PY@tc##1{\textcolor[rgb]{0.00,0.50,0.00}{##1}}}
\@namedef{PY@tok@kr}{\let\PY@bf=\textbf\def\PY@tc##1{\textcolor[rgb]{0.00,0.50,0.00}{##1}}}
\@namedef{PY@tok@bp}{\def\PY@tc##1{\textcolor[rgb]{0.00,0.50,0.00}{##1}}}
\@namedef{PY@tok@fm}{\def\PY@tc##1{\textcolor[rgb]{0.00,0.00,1.00}{##1}}}
\@namedef{PY@tok@vc}{\def\PY@tc##1{\textcolor[rgb]{0.10,0.09,0.49}{##1}}}
\@namedef{PY@tok@vg}{\def\PY@tc##1{\textcolor[rgb]{0.10,0.09,0.49}{##1}}}
\@namedef{PY@tok@vi}{\def\PY@tc##1{\textcolor[rgb]{0.10,0.09,0.49}{##1}}}
\@namedef{PY@tok@vm}{\def\PY@tc##1{\textcolor[rgb]{0.10,0.09,0.49}{##1}}}
\@namedef{PY@tok@sa}{\def\PY@tc##1{\textcolor[rgb]{0.73,0.13,0.13}{##1}}}
\@namedef{PY@tok@sb}{\def\PY@tc##1{\textcolor[rgb]{0.73,0.13,0.13}{##1}}}
\@namedef{PY@tok@sc}{\def\PY@tc##1{\textcolor[rgb]{0.73,0.13,0.13}{##1}}}
\@namedef{PY@tok@dl}{\def\PY@tc##1{\textcolor[rgb]{0.73,0.13,0.13}{##1}}}
\@namedef{PY@tok@s2}{\def\PY@tc##1{\textcolor[rgb]{0.73,0.13,0.13}{##1}}}
\@namedef{PY@tok@sh}{\def\PY@tc##1{\textcolor[rgb]{0.73,0.13,0.13}{##1}}}
\@namedef{PY@tok@s1}{\def\PY@tc##1{\textcolor[rgb]{0.73,0.13,0.13}{##1}}}
\@namedef{PY@tok@mb}{\def\PY@tc##1{\textcolor[rgb]{0.40,0.40,0.40}{##1}}}
\@namedef{PY@tok@mf}{\def\PY@tc##1{\textcolor[rgb]{0.40,0.40,0.40}{##1}}}
\@namedef{PY@tok@mh}{\def\PY@tc##1{\textcolor[rgb]{0.40,0.40,0.40}{##1}}}
\@namedef{PY@tok@mi}{\def\PY@tc##1{\textcolor[rgb]{0.40,0.40,0.40}{##1}}}
\@namedef{PY@tok@il}{\def\PY@tc##1{\textcolor[rgb]{0.40,0.40,0.40}{##1}}}
\@namedef{PY@tok@mo}{\def\PY@tc##1{\textcolor[rgb]{0.40,0.40,0.40}{##1}}}
\@namedef{PY@tok@ch}{\let\PY@it=\textit\def\PY@tc##1{\textcolor[rgb]{0.24,0.48,0.48}{##1}}}
\@namedef{PY@tok@cm}{\let\PY@it=\textit\def\PY@tc##1{\textcolor[rgb]{0.24,0.48,0.48}{##1}}}
\@namedef{PY@tok@cpf}{\let\PY@it=\textit\def\PY@tc##1{\textcolor[rgb]{0.24,0.48,0.48}{##1}}}
\@namedef{PY@tok@c1}{\let\PY@it=\textit\def\PY@tc##1{\textcolor[rgb]{0.24,0.48,0.48}{##1}}}
\@namedef{PY@tok@cs}{\let\PY@it=\textit\def\PY@tc##1{\textcolor[rgb]{0.24,0.48,0.48}{##1}}}

\def\PYZbs{\char`\\}
\def\PYZus{\char`\_}
\def\PYZob{\char`\{}
\def\PYZcb{\char`\}}
\def\PYZca{\char`\^}
\def\PYZam{\char`\&}
\def\PYZlt{\char`\<}
\def\PYZgt{\char`\>}
\def\PYZsh{\char`\#}
\def\PYZpc{\char`\%}
\def\PYZdl{\char`\$}
\def\PYZhy{\char`\-}
\def\PYZsq{\char`\'}
\def\PYZdq{\char`\"}
\def\PYZti{\char`\~}
% for compatibility with earlier versions
\def\PYZat{@}
\def\PYZlb{[}
\def\PYZrb{]}
\makeatother


    % For linebreaks inside Verbatim environment from package fancyvrb. 
    \makeatletter
        \newbox\Wrappedcontinuationbox 
        \newbox\Wrappedvisiblespacebox 
        \newcommand*\Wrappedvisiblespace {\textcolor{red}{\textvisiblespace}} 
        \newcommand*\Wrappedcontinuationsymbol {\textcolor{red}{\llap{\tiny$\m@th\hookrightarrow$}}} 
        \newcommand*\Wrappedcontinuationindent {3ex } 
        \newcommand*\Wrappedafterbreak {\kern\Wrappedcontinuationindent\copy\Wrappedcontinuationbox} 
        % Take advantage of the already applied Pygments mark-up to insert 
        % potential linebreaks for TeX processing. 
        %        {, <, #, %, $, ' and ": go to next line. 
        %        _, }, ^, &, >, - and ~: stay at end of broken line. 
        % Use of \textquotesingle for straight quote. 
        \newcommand*\Wrappedbreaksatspecials {% 
            \def\PYGZus{\discretionary{\char`\_}{\Wrappedafterbreak}{\char`\_}}% 
            \def\PYGZob{\discretionary{}{\Wrappedafterbreak\char`\{}{\char`\{}}% 
            \def\PYGZcb{\discretionary{\char`\}}{\Wrappedafterbreak}{\char`\}}}% 
            \def\PYGZca{\discretionary{\char`\^}{\Wrappedafterbreak}{\char`\^}}% 
            \def\PYGZam{\discretionary{\char`\&}{\Wrappedafterbreak}{\char`\&}}% 
            \def\PYGZlt{\discretionary{}{\Wrappedafterbreak\char`\<}{\char`\<}}% 
            \def\PYGZgt{\discretionary{\char`\>}{\Wrappedafterbreak}{\char`\>}}% 
            \def\PYGZsh{\discretionary{}{\Wrappedafterbreak\char`\#}{\char`\#}}% 
            \def\PYGZpc{\discretionary{}{\Wrappedafterbreak\char`\%}{\char`\%}}% 
            \def\PYGZdl{\discretionary{}{\Wrappedafterbreak\char`\$}{\char`\$}}% 
            \def\PYGZhy{\discretionary{\char`\-}{\Wrappedafterbreak}{\char`\-}}% 
            \def\PYGZsq{\discretionary{}{\Wrappedafterbreak\textquotesingle}{\textquotesingle}}% 
            \def\PYGZdq{\discretionary{}{\Wrappedafterbreak\char`\"}{\char`\"}}% 
            \def\PYGZti{\discretionary{\char`\~}{\Wrappedafterbreak}{\char`\~}}% 
        } 
        % Some characters . , ; ? ! / are not pygmentized. 
        % This macro makes them "active" and they will insert potential linebreaks 
        \newcommand*\Wrappedbreaksatpunct {% 
            \lccode`\~`\.\lowercase{\def~}{\discretionary{\hbox{\char`\.}}{\Wrappedafterbreak}{\hbox{\char`\.}}}% 
            \lccode`\~`\,\lowercase{\def~}{\discretionary{\hbox{\char`\,}}{\Wrappedafterbreak}{\hbox{\char`\,}}}% 
            \lccode`\~`\;\lowercase{\def~}{\discretionary{\hbox{\char`\;}}{\Wrappedafterbreak}{\hbox{\char`\;}}}% 
            \lccode`\~`\:\lowercase{\def~}{\discretionary{\hbox{\char`\:}}{\Wrappedafterbreak}{\hbox{\char`\:}}}% 
            \lccode`\~`\?\lowercase{\def~}{\discretionary{\hbox{\char`\?}}{\Wrappedafterbreak}{\hbox{\char`\?}}}% 
            \lccode`\~`\!\lowercase{\def~}{\discretionary{\hbox{\char`\!}}{\Wrappedafterbreak}{\hbox{\char`\!}}}% 
            \lccode`\~`\/\lowercase{\def~}{\discretionary{\hbox{\char`\/}}{\Wrappedafterbreak}{\hbox{\char`\/}}}% 
            \catcode`\.\active
            \catcode`\,\active 
            \catcode`\;\active
            \catcode`\:\active
            \catcode`\?\active
            \catcode`\!\active
            \catcode`\/\active 
            \lccode`\~`\~ 	
        }
    \makeatother

    \let\OriginalVerbatim=\Verbatim
    \makeatletter
    \renewcommand{\Verbatim}[1][1]{%
        %\parskip\z@skip
        \sbox\Wrappedcontinuationbox {\Wrappedcontinuationsymbol}%
        \sbox\Wrappedvisiblespacebox {\FV@SetupFont\Wrappedvisiblespace}%
        \def\FancyVerbFormatLine ##1{\hsize\linewidth
            \vtop{\raggedright\hyphenpenalty\z@\exhyphenpenalty\z@
                \doublehyphendemerits\z@\finalhyphendemerits\z@
                \strut ##1\strut}%
        }%
        % If the linebreak is at a space, the latter will be displayed as visible
        % space at end of first line, and a continuation symbol starts next line.
        % Stretch/shrink are however usually zero for typewriter font.
        \def\FV@Space {%
            \nobreak\hskip\z@ plus\fontdimen3\font minus\fontdimen4\font
            \discretionary{\copy\Wrappedvisiblespacebox}{\Wrappedafterbreak}
            {\kern\fontdimen2\font}%
        }%
        
        % Allow breaks at special characters using \PYG... macros.
        \Wrappedbreaksatspecials
        % Breaks at punctuation characters . , ; ? ! and / need catcode=\active 	
        \OriginalVerbatim[#1,codes*=\Wrappedbreaksatpunct]%
    }
    \makeatother

    % Exact colors from NB
    \definecolor{incolor}{HTML}{303F9F}
    \definecolor{outcolor}{HTML}{D84315}
    \definecolor{cellborder}{HTML}{CFCFCF}
    \definecolor{cellbackground}{HTML}{F7F7F7}
    
    % prompt
    \makeatletter
    \newcommand{\boxspacing}{\kern\kvtcb@left@rule\kern\kvtcb@boxsep}
    \makeatother
    \newcommand{\prompt}[4]{
        {\ttfamily\llap{{\color{#2}[#3]:\hspace{3pt}#4}}\vspace{-\baselineskip}}
    }
    

    
    % Prevent overflowing lines due to hard-to-break entities
    \sloppy 
    % Setup hyperref package
    \hypersetup{
      breaklinks=true,  % so long urls are correctly broken across lines
      colorlinks=true,
      urlcolor=urlcolor,
      linkcolor=linkcolor,
      citecolor=citecolor,
      }
    % Slightly bigger margins than the latex defaults
    
    \geometry{verbose,tmargin=1in,bmargin=1in,lmargin=1in,rmargin=1in}
    
    

\begin{document}
    
    \maketitle
    
    

    
    \hypertarget{evaluaciuxf3n-de-la-segmentaciuxf3n}{%
\section{Evaluación de la
segmentación}\label{evaluaciuxf3n-de-la-segmentaciuxf3n}}

    \begin{tcolorbox}[breakable, size=fbox, boxrule=1pt, pad at break*=1mm,colback=cellbackground, colframe=cellborder]
\prompt{In}{incolor}{1}{\boxspacing}
\begin{Verbatim}[commandchars=\\\{\}]
\PY{k+kn}{import} \PY{n+nn}{cv2} \PY{k}{as} \PY{n+nn}{cv} \PY{c+c1}{\PYZsh{} ver. 4.5.5}
\PY{k+kn}{import} \PY{n+nn}{numpy} \PY{k}{as} \PY{n+nn}{np}
\PY{k+kn}{from} \PY{n+nn}{matplotlib} \PY{k+kn}{import} \PY{n}{pyplot} \PY{k}{as} \PY{n}{plt}
\end{Verbatim}
\end{tcolorbox}

    \hypertarget{imuxe1genes-a-utilizar}{%
\subsection{Imágenes a utilizar}\label{imuxe1genes-a-utilizar}}

En esta ocasión se usarán cartas de póquer para establecer la
segmentación de objetos. Las imágenes \texttt{img\_07c},
\texttt{img\_09d}, \texttt{img\_10c} y \texttt{img\_10t} contienen una
sola carta, en este caso se pretenderá identificar cuantos elementos la
constituyen: números y figuras. Por su porte, las imágenes
\texttt{img\_all} y \texttt{img\_4a6} contienen varias cartas, por lo
que se pretenderá identificar cuántas cartas hay en la imagen. Todas las
imágenes tienen licencia Creative Commons y fueron obtenidas con el
buscador
\href{https://duckduckgo.com/?q=cartas+poker\&t=brave\&iax=images\&ia=images\&iaf=license\%3AAny}{DuckDuckGo}.

    \begin{tcolorbox}[breakable, size=fbox, boxrule=1pt, pad at break*=1mm,colback=cellbackground, colframe=cellborder]
\prompt{In}{incolor}{2}{\boxspacing}
\begin{Verbatim}[commandchars=\\\{\}]
\PY{n}{img\PYZus{}07c} \PY{o}{=} \PY{n}{cv}\PY{o}{.}\PY{n}{imread}\PY{p}{(}\PY{l+s+s1}{\PYZsq{}}\PY{l+s+s1}{im/07c.jpg}\PY{l+s+s1}{\PYZsq{}}\PY{p}{)}
\PY{n}{img\PYZus{}09d} \PY{o}{=} \PY{n}{cv}\PY{o}{.}\PY{n}{imread}\PY{p}{(}\PY{l+s+s1}{\PYZsq{}}\PY{l+s+s1}{im/09d.png}\PY{l+s+s1}{\PYZsq{}}\PY{p}{)}
\PY{n}{img\PYZus{}10c} \PY{o}{=} \PY{n}{cv}\PY{o}{.}\PY{n}{imread}\PY{p}{(}\PY{l+s+s1}{\PYZsq{}}\PY{l+s+s1}{im/10c.jpg}\PY{l+s+s1}{\PYZsq{}}\PY{p}{)}
\PY{n}{img\PYZus{}10t} \PY{o}{=} \PY{n}{cv}\PY{o}{.}\PY{n}{imread}\PY{p}{(}\PY{l+s+s1}{\PYZsq{}}\PY{l+s+s1}{im/10t.jpg}\PY{l+s+s1}{\PYZsq{}}\PY{p}{)}
\PY{n}{img\PYZus{}all} \PY{o}{=} \PY{n}{cv}\PY{o}{.}\PY{n}{imread}\PY{p}{(}\PY{l+s+s1}{\PYZsq{}}\PY{l+s+s1}{im/all.png}\PY{l+s+s1}{\PYZsq{}}\PY{p}{)}
\PY{n}{img\PYZus{}4a6} \PY{o}{=} \PY{n}{cv}\PY{o}{.}\PY{n}{imread}\PY{p}{(}\PY{l+s+s1}{\PYZsq{}}\PY{l+s+s1}{im/4a6.jpg}\PY{l+s+s1}{\PYZsq{}}\PY{p}{)}
\end{Verbatim}
\end{tcolorbox}

    \hypertarget{segmentaciuxf3n-con-canny}{%
\subsection{Segmentación con Canny}\label{segmentaciuxf3n-con-canny}}

Primero seleccionaremos una de las seis imágenes de trabajo

    \begin{tcolorbox}[breakable, size=fbox, boxrule=1pt, pad at break*=1mm,colback=cellbackground, colframe=cellborder]
\prompt{In}{incolor}{3}{\boxspacing}
\begin{Verbatim}[commandchars=\\\{\}]
\PY{n}{im} \PY{o}{=} \PY{n}{img\PYZus{}4a6}
\PY{n}{im} \PY{o}{=} \PY{n}{cv}\PY{o}{.}\PY{n}{cvtColor}\PY{p}{(}\PY{n}{im}\PY{p}{,} \PY{n}{cv}\PY{o}{.}\PY{n}{COLOR\PYZus{}BGR2GRAY}\PY{p}{)}
\PY{n}{im} \PY{o}{=} \PY{n}{cv}\PY{o}{.}\PY{n}{cvtColor}\PY{p}{(}\PY{n}{im}\PY{p}{,} \PY{n}{cv}\PY{o}{.}\PY{n}{COLOR\PYZus{}BGR2RGB}\PY{p}{)}
\PY{n}{plt}\PY{o}{.}\PY{n}{imshow}\PY{p}{(}\PY{n}{im}\PY{p}{)}
\PY{n}{plt}\PY{o}{.}\PY{n}{title}\PY{p}{(}\PY{l+s+s2}{\PYZdq{}}\PY{l+s+s2}{Imagen original}\PY{l+s+s2}{\PYZdq{}}\PY{p}{)}\PY{p}{,} \PY{n}{plt}\PY{o}{.}\PY{n}{xticks}\PY{p}{(}\PY{p}{[}\PY{p}{]}\PY{p}{)}\PY{p}{,} \PY{n}{plt}\PY{o}{.}\PY{n}{yticks}\PY{p}{(}\PY{p}{[}\PY{p}{]}\PY{p}{)}
\PY{n}{plt}\PY{o}{.}\PY{n}{show}\PY{p}{(}\PY{p}{)}
\end{Verbatim}
\end{tcolorbox}

    \begin{center}
    \adjustimage{max size={0.9\linewidth}{0.9\paperheight}}{output_5_0.png}
    \end{center}
    { \hspace*{\fill} \\}
    
    Una vez que seleccionamos una imagen en escala de grises, es posible
obtener una imagen binaria con los bordes claramente identificados
gracias al algoritmo de Canny.

    \begin{tcolorbox}[breakable, size=fbox, boxrule=1pt, pad at break*=1mm,colback=cellbackground, colframe=cellborder]
\prompt{In}{incolor}{4}{\boxspacing}
\begin{Verbatim}[commandchars=\\\{\}]
\PY{c+c1}{\PYZsh{} Los números son las intensidades min y máx para identificar el borde}
\PY{n}{canny} \PY{o}{=} \PY{n}{cv}\PY{o}{.}\PY{n}{Canny}\PY{p}{(}\PY{n}{im}\PY{p}{,} \PY{l+m+mi}{25}\PY{p}{,} \PY{l+m+mi}{100}\PY{p}{)}
\PY{n}{cv}\PY{o}{.}\PY{n}{imwrite}\PY{p}{(}\PY{l+s+s1}{\PYZsq{}}\PY{l+s+s1}{out/canny.jpg}\PY{l+s+s1}{\PYZsq{}}\PY{p}{,} \PY{n}{canny}\PY{p}{)}
\end{Verbatim}
\end{tcolorbox}

            \begin{tcolorbox}[breakable, size=fbox, boxrule=.5pt, pad at break*=1mm, opacityfill=0]
\prompt{Out}{outcolor}{4}{\boxspacing}
\begin{Verbatim}[commandchars=\\\{\}]
True
\end{Verbatim}
\end{tcolorbox}
        
    Ahora procederemos a la segmentación de la imagen. El método
\texttt{cv.findContours()} encuentra los contornos en una imagen binaria
(primer parámetro). Debe especificarse el algoritmo que usará para
encontrar los contornos (segundo parámetro):

\begin{itemize}
\tightlist
\item
  \texttt{RETR\_EXTERNAL}: Detecta solo los bordes más externos.
\item
  \texttt{RETR\_LIST}: Detecta todos los contornos.
\item
  \texttt{RETR\_CCOMP} y \texttt{RETR\_TREE}: Detectan todos los bordes
  y establecen alguna jerarquía entre ellos.
\end{itemize}

El último parámetro establece el método de aproximación para el
contorno, los principales son:

\begin{itemize}
\tightlist
\item
  \texttt{CHAIN\_APPROX\_NONE}: Almacena absolutamente todos los puntos
  de contorno.
\item
  \texttt{CHAIN\_APPROX\_SIMPLE}: Comprime segmentos horizontales,
  verticales y diagonales y deja sólo sus puntos finales.
\end{itemize}

Al cambiar este último parámetro no se han observado cambios en el
resultado.

Por su parte el método \texttt{drawContours()} dibujará los contornos
sobre la imagen original (primer parámetro); para ello debe acceder a
los contornos previamente obtenidos (segundo parámetro). El tercer
parámetro es aquel contorno específico a dibujar, \texttt{-1} significa
todos. Los parámetros siguientes serán el color del contorno y su
grosor.\cite{Madhav2019Image, OpenCV2021OpenCV}

    \begin{tcolorbox}[breakable, size=fbox, boxrule=1pt, pad at break*=1mm,colback=cellbackground, colframe=cellborder]
\prompt{In}{incolor}{5}{\boxspacing}
\begin{Verbatim}[commandchars=\\\{\}]
\PY{n}{contours}\PY{p}{,} \PY{n}{hierarchy} \PY{o}{=} \PY{n}{cv}\PY{o}{.}\PY{n}{findContours}\PY{p}{(}\PY{n}{canny}\PY{p}{,} \PY{n}{cv}\PY{o}{.}\PY{n}{RETR\PYZus{}EXTERNAL}\PY{p}{,} \PY{n}{cv}\PY{o}{.}\PY{n}{CHAIN\PYZus{}APPROX\PYZus{}NONE}\PY{p}{)}
\PY{n}{out} \PY{o}{=} \PY{n}{cv}\PY{o}{.}\PY{n}{drawContours}\PY{p}{(}\PY{n}{im}\PY{p}{,} \PY{n}{contours}\PY{p}{,} \PY{o}{\PYZhy{}}\PY{l+m+mi}{1}\PY{p}{,} \PY{p}{(}\PY{l+m+mi}{64}\PY{p}{,}\PY{l+m+mi}{255}\PY{p}{,}\PY{l+m+mi}{64}\PY{p}{)}\PY{p}{,} \PY{l+m+mi}{1}\PY{p}{)}
\PY{n}{cv}\PY{o}{.}\PY{n}{imwrite}\PY{p}{(}\PY{l+s+s1}{\PYZsq{}}\PY{l+s+s1}{out/out.jpg}\PY{l+s+s1}{\PYZsq{}}\PY{p}{,} \PY{n}{out}\PY{p}{)}
\PY{n+nb}{print}\PY{p}{(}\PY{l+s+s1}{\PYZsq{}}\PY{l+s+s1}{Número de contornos encontrados: }\PY{l+s+s1}{\PYZsq{}}\PY{p}{,} \PY{n+nb}{len}\PY{p}{(}\PY{n}{contours}\PY{p}{)}\PY{p}{)}
\end{Verbatim}
\end{tcolorbox}

    \begin{Verbatim}[commandchars=\\\{\}]
Número de contornos encontrados:  4
    \end{Verbatim}

    \begin{tcolorbox}[breakable, size=fbox, boxrule=1pt, pad at break*=1mm,colback=cellbackground, colframe=cellborder]
\prompt{In}{incolor}{6}{\boxspacing}
\begin{Verbatim}[commandchars=\\\{\}]
\PY{n}{canny} \PY{o}{=} \PY{n}{cv}\PY{o}{.}\PY{n}{cvtColor}\PY{p}{(}\PY{n}{canny}\PY{p}{,} \PY{n}{cv}\PY{o}{.}\PY{n}{COLOR\PYZus{}BGR2RGB}\PY{p}{)}
\PY{n}{out} \PY{o}{=} \PY{n}{cv}\PY{o}{.}\PY{n}{cvtColor}\PY{p}{(}\PY{n}{out}\PY{p}{,} \PY{n}{cv}\PY{o}{.}\PY{n}{COLOR\PYZus{}BGR2RGB}\PY{p}{)}
\PY{n}{plt}\PY{o}{.}\PY{n}{subplot}\PY{p}{(}\PY{l+m+mi}{211}\PY{p}{)}
\PY{n}{plt}\PY{o}{.}\PY{n}{imshow}\PY{p}{(}\PY{n}{canny}\PY{p}{)}
\PY{n}{plt}\PY{o}{.}\PY{n}{title}\PY{p}{(}\PY{l+s+s1}{\PYZsq{}}\PY{l+s+s1}{Imagen Canny}\PY{l+s+s1}{\PYZsq{}}\PY{p}{)}\PY{p}{,} \PY{n}{plt}\PY{o}{.}\PY{n}{xticks}\PY{p}{(}\PY{p}{[}\PY{p}{]}\PY{p}{)}\PY{p}{,} \PY{n}{plt}\PY{o}{.}\PY{n}{yticks}\PY{p}{(}\PY{p}{[}\PY{p}{]}\PY{p}{)}
\PY{n}{plt}\PY{o}{.}\PY{n}{subplot}\PY{p}{(}\PY{l+m+mi}{212}\PY{p}{)}
\PY{n}{plt}\PY{o}{.}\PY{n}{imshow}\PY{p}{(}\PY{n}{out}\PY{p}{)}
\PY{n}{plt}\PY{o}{.}\PY{n}{title}\PY{p}{(}\PY{l+s+s1}{\PYZsq{}}\PY{l+s+s1}{Imagen segmentada}\PY{l+s+s1}{\PYZsq{}}\PY{p}{)}\PY{p}{,} \PY{n}{plt}\PY{o}{.}\PY{n}{xticks}\PY{p}{(}\PY{p}{[}\PY{p}{]}\PY{p}{)}\PY{p}{,} \PY{n}{plt}\PY{o}{.}\PY{n}{yticks}\PY{p}{(}\PY{p}{[}\PY{p}{]}\PY{p}{)}
\PY{n}{plt}\PY{o}{.}\PY{n}{show}\PY{p}{(}\PY{p}{)}
\end{Verbatim}
\end{tcolorbox}

    \begin{center}
    \adjustimage{max size={0.9\linewidth}{0.9\paperheight}}{output_10_0.png}
    \end{center}
    { \hspace*{\fill} \\}
    
    Como puede observarse en la imagen, los contornos son dibujados
propiamente, sin embargo, el número de contornos puede no coincidir con
el número de objetos en la imagen. Por ejemplo, los tréboles debido a lo
angosto de la parte media del dibujo, pueden ser identificados como dos
objetos y no como uno. A continuación se presenta una lista con la
relación de objetos y contornos detectados:

\begin{longtable}[]{@{}lll@{}}
\toprule
Imagen & Objetos observables & Segmentos detectados \\
\midrule
\endhead
img\_07c & 11 & 11 \\
img\_09d & 13 & 14 \\
img\_10c & 16 & 16 \\
img\_10t & 16 & 18 \\
img\_all & 52 & 56 \\
img\_4a6 & 4 & 4 \\
\bottomrule
\end{longtable}

Es bien sabido que el algoritmo de Canny aplica una reducción de ruido
con un filtro gausiano como primer paso. Parece ser que no es suficiente
en todos los casos.

    \hypertarget{segmentaciuxf3n-con-umbralado}{%
\subsection{Segmentación con
umbralado}\label{segmentaciuxf3n-con-umbralado}}

Es posible usar un umbralado directo para generar una imagen en binario.
En este caso usaremos solo las imágenes que presentaron fallas en la
sección anterior: \texttt{img\_09d}, \texttt{img\_10t} y
\texttt{img\_all}.

Este método mejora el resultado siempre y cuando la variable
\texttt{thresh} tenga el valor ideal\cite{Madhav2019Image}. Aquí se
coloca el ejemplo que funciona con la imagen \texttt{img\_09d} donde
\texttt{thresh\ =\ 32}, para corregir la imagen \texttt{img\_all} es
necesario establecer \texttt{thresh\ =\ 160}.

Se hizo la prueba cambiando el algoritmo de umbralado a
\texttt{THRESH\_OTSU}, pero no se observó ninguna mejora.

    \begin{tcolorbox}[breakable, size=fbox, boxrule=1pt, pad at break*=1mm,colback=cellbackground, colframe=cellborder]
\prompt{In}{incolor}{7}{\boxspacing}
\begin{Verbatim}[commandchars=\\\{\}]
\PY{n}{im} \PY{o}{=} \PY{n}{img\PYZus{}09d}
\PY{n}{im} \PY{o}{=} \PY{n}{cv}\PY{o}{.}\PY{n}{cvtColor}\PY{p}{(}\PY{n}{im}\PY{p}{,} \PY{n}{cv}\PY{o}{.}\PY{n}{COLOR\PYZus{}BGR2GRAY}\PY{p}{)}
\PY{c+c1}{\PYZsh{} Imagen en binario}
\PY{n}{thresh} \PY{o}{=} \PY{l+m+mi}{32} \PY{c+c1}{\PYZsh{}Los bits del umbralado}
\PY{n}{binary} \PY{o}{=} \PY{n}{cv}\PY{o}{.}\PY{n}{threshold}\PY{p}{(}\PY{n}{im}\PY{p}{,} \PY{n}{thresh}\PY{p}{,} \PY{l+m+mi}{255}\PY{p}{,} \PY{n}{cv}\PY{o}{.}\PY{n}{THRESH\PYZus{}BINARY}\PY{p}{)}\PY{p}{[}\PY{l+m+mi}{1}\PY{p}{]}
\end{Verbatim}
\end{tcolorbox}

    \begin{tcolorbox}[breakable, size=fbox, boxrule=1pt, pad at break*=1mm,colback=cellbackground, colframe=cellborder]
\prompt{In}{incolor}{8}{\boxspacing}
\begin{Verbatim}[commandchars=\\\{\}]
\PY{n}{im} \PY{o}{=} \PY{n}{cv}\PY{o}{.}\PY{n}{cvtColor}\PY{p}{(}\PY{n}{im}\PY{p}{,} \PY{n}{cv}\PY{o}{.}\PY{n}{COLOR\PYZus{}BGR2RGB}\PY{p}{)}
\PY{n}{plt}\PY{o}{.}\PY{n}{imshow}\PY{p}{(}\PY{n}{im}\PY{p}{)}
\PY{n}{plt}\PY{o}{.}\PY{n}{title}\PY{p}{(}\PY{l+s+s2}{\PYZdq{}}\PY{l+s+s2}{Imagen original}\PY{l+s+s2}{\PYZdq{}}\PY{p}{)}\PY{p}{,} \PY{n}{plt}\PY{o}{.}\PY{n}{xticks}\PY{p}{(}\PY{p}{[}\PY{p}{]}\PY{p}{)}\PY{p}{,} \PY{n}{plt}\PY{o}{.}\PY{n}{yticks}\PY{p}{(}\PY{p}{[}\PY{p}{]}\PY{p}{)}
\PY{n}{plt}\PY{o}{.}\PY{n}{show}\PY{p}{(}\PY{p}{)}
\end{Verbatim}
\end{tcolorbox}

    \begin{center}
    \adjustimage{max size={0.9\linewidth}{0.9\paperheight}}{output_14_0.png}
    \end{center}
    { \hspace*{\fill} \\}
    
    \begin{tcolorbox}[breakable, size=fbox, boxrule=1pt, pad at break*=1mm,colback=cellbackground, colframe=cellborder]
\prompt{In}{incolor}{9}{\boxspacing}
\begin{Verbatim}[commandchars=\\\{\}]
\PY{n}{contours}\PY{p}{,} \PY{n}{hierarchy} \PY{o}{=} \PY{n}{cv}\PY{o}{.}\PY{n}{findContours}\PY{p}{(}\PY{n}{binary}\PY{p}{,} \PY{n}{cv}\PY{o}{.}\PY{n}{RETR\PYZus{}EXTERNAL}\PY{p}{,} \PY{n}{cv}\PY{o}{.}\PY{n}{CHAIN\PYZus{}APPROX\PYZus{}SIMPLE}\PY{p}{)}
\PY{n}{out} \PY{o}{=} \PY{n}{cv}\PY{o}{.}\PY{n}{drawContours}\PY{p}{(}\PY{n}{im}\PY{p}{,} \PY{n}{contours}\PY{p}{,} \PY{o}{\PYZhy{}}\PY{l+m+mi}{1}\PY{p}{,} \PY{p}{(}\PY{l+m+mi}{64}\PY{p}{,}\PY{l+m+mi}{255}\PY{p}{,}\PY{l+m+mi}{64}\PY{p}{)}\PY{p}{,} \PY{l+m+mi}{3}\PY{p}{)}
\PY{n}{cv}\PY{o}{.}\PY{n}{imwrite}\PY{p}{(}\PY{l+s+s1}{\PYZsq{}}\PY{l+s+s1}{out/bin.jpg}\PY{l+s+s1}{\PYZsq{}}\PY{p}{,} \PY{n}{binary}\PY{p}{)}
\PY{n}{cv}\PY{o}{.}\PY{n}{imwrite}\PY{p}{(}\PY{l+s+s1}{\PYZsq{}}\PY{l+s+s1}{out/bin\PYZus{}out.jpg}\PY{l+s+s1}{\PYZsq{}}\PY{p}{,} \PY{n}{out}\PY{p}{)}
\PY{n+nb}{print}\PY{p}{(}\PY{l+s+s1}{\PYZsq{}}\PY{l+s+s1}{Número de contornos encontrados: }\PY{l+s+s1}{\PYZsq{}}\PY{p}{,} \PY{n+nb}{len}\PY{p}{(}\PY{n}{contours}\PY{p}{)}\PY{p}{)}
\end{Verbatim}
\end{tcolorbox}

    \begin{Verbatim}[commandchars=\\\{\}]
Número de contornos encontrados:  13
    \end{Verbatim}

    \begin{tcolorbox}[breakable, size=fbox, boxrule=1pt, pad at break*=1mm,colback=cellbackground, colframe=cellborder]
\prompt{In}{incolor}{10}{\boxspacing}
\begin{Verbatim}[commandchars=\\\{\}]
\PY{n}{binary} \PY{o}{=} \PY{n}{cv}\PY{o}{.}\PY{n}{cvtColor}\PY{p}{(}\PY{n}{binary}\PY{p}{,} \PY{n}{cv}\PY{o}{.}\PY{n}{COLOR\PYZus{}BGR2RGB}\PY{p}{)}
\PY{n}{plt}\PY{o}{.}\PY{n}{subplot}\PY{p}{(}\PY{l+m+mi}{121}\PY{p}{)}
\PY{n}{plt}\PY{o}{.}\PY{n}{imshow}\PY{p}{(}\PY{n}{binary}\PY{p}{)}
\PY{n}{plt}\PY{o}{.}\PY{n}{title}\PY{p}{(}\PY{l+s+s1}{\PYZsq{}}\PY{l+s+s1}{Imagen binaria}\PY{l+s+s1}{\PYZsq{}}\PY{p}{)}\PY{p}{,} \PY{n}{plt}\PY{o}{.}\PY{n}{xticks}\PY{p}{(}\PY{p}{[}\PY{p}{]}\PY{p}{)}\PY{p}{,} \PY{n}{plt}\PY{o}{.}\PY{n}{yticks}\PY{p}{(}\PY{p}{[}\PY{p}{]}\PY{p}{)}
\PY{n}{plt}\PY{o}{.}\PY{n}{subplot}\PY{p}{(}\PY{l+m+mi}{122}\PY{p}{)}
\PY{n}{plt}\PY{o}{.}\PY{n}{imshow}\PY{p}{(}\PY{n}{out}\PY{p}{)}
\PY{n}{plt}\PY{o}{.}\PY{n}{title}\PY{p}{(}\PY{l+s+s1}{\PYZsq{}}\PY{l+s+s1}{Imagen segmentada}\PY{l+s+s1}{\PYZsq{}}\PY{p}{)}\PY{p}{,} \PY{n}{plt}\PY{o}{.}\PY{n}{xticks}\PY{p}{(}\PY{p}{[}\PY{p}{]}\PY{p}{)}\PY{p}{,} \PY{n}{plt}\PY{o}{.}\PY{n}{yticks}\PY{p}{(}\PY{p}{[}\PY{p}{]}\PY{p}{)}
\PY{n}{plt}\PY{o}{.}\PY{n}{show}\PY{p}{(}\PY{p}{)}
\end{Verbatim}
\end{tcolorbox}

    \begin{center}
    \adjustimage{max size={0.9\linewidth}{0.9\paperheight}}{output_16_0.png}
    \end{center}
    { \hspace*{\fill} \\}
    
    Es posible observar los siguientes resultados mediante este método:

\begin{longtable}[]{@{}lll@{}}
\toprule
Carta & Objetos observables & Segmentos detectados (Umbral) \\
\midrule
\endhead
img\_09d & 13 & 13 (32) \\
img\_10t & 16 & 1 \\
img\_all & 52 & 52 (160) \\
\bottomrule
\end{longtable}

Así com sucede en el caso de la imagen \texttt{img\_10t}, es posible que
este método encuentre como contorno más externo los límites de la
imagen, esto es posible solucionando si el algoritmo del método
\texttt{cv.findContours()} cambia a \texttt{RETR\_LIST}. Esto mejorará
el comportamiento porque será posible identificar los contornos
internos, no solo los externos. Pese a esto, en la mayoría de los casos
eleva demasiado el número de bordes encontrados en la imagen.

    \hypertarget{segmentaciuxf3n-con-color}{%
\subsection{Segmentación con color}\label{segmentaciuxf3n-con-color}}

Otra forma de identificar los segmentos es usando el color en el
proceso. La ventaja de este método es que permitirá identificar objetos
que se encuentren en un rango de color.\cite{Maindola20214}

    \begin{tcolorbox}[breakable, size=fbox, boxrule=1pt, pad at break*=1mm,colback=cellbackground, colframe=cellborder]
\prompt{In}{incolor}{11}{\boxspacing}
\begin{Verbatim}[commandchars=\\\{\}]
\PY{n}{im} \PY{o}{=} \PY{n}{img\PYZus{}10t}
\PY{n}{im} \PY{o}{=} \PY{n}{cv}\PY{o}{.}\PY{n}{cvtColor}\PY{p}{(}\PY{n}{im}\PY{p}{,} \PY{n}{cv}\PY{o}{.}\PY{n}{COLOR\PYZus{}BGR2RGB}\PY{p}{)}
\PY{n}{plt}\PY{o}{.}\PY{n}{imshow}\PY{p}{(}\PY{n}{im}\PY{p}{)}
\PY{n}{plt}\PY{o}{.}\PY{n}{title}\PY{p}{(}\PY{l+s+s2}{\PYZdq{}}\PY{l+s+s2}{Imagen original}\PY{l+s+s2}{\PYZdq{}}\PY{p}{)}\PY{p}{,} \PY{n}{plt}\PY{o}{.}\PY{n}{xticks}\PY{p}{(}\PY{p}{[}\PY{p}{]}\PY{p}{)}\PY{p}{,} \PY{n}{plt}\PY{o}{.}\PY{n}{yticks}\PY{p}{(}\PY{p}{[}\PY{p}{]}\PY{p}{)}
\PY{n}{plt}\PY{o}{.}\PY{n}{show}\PY{p}{(}\PY{p}{)}
\end{Verbatim}
\end{tcolorbox}

    \begin{center}
    \adjustimage{max size={0.9\linewidth}{0.9\paperheight}}{output_19_0.png}
    \end{center}
    { \hspace*{\fill} \\}
    
    Es necesario especificar el rango de color de interés. En el siguiente
código se comentan los valores que encontrarían en una carta roja y se
aplican aquellos que tienen que ver con el negro, propio de la imagen
seleccionada.

    \begin{tcolorbox}[breakable, size=fbox, boxrule=1pt, pad at break*=1mm,colback=cellbackground, colframe=cellborder]
\prompt{In}{incolor}{12}{\boxspacing}
\begin{Verbatim}[commandchars=\\\{\}]
\PY{c+c1}{\PYZsh{}light\PYZus{}red = (205, 0, 0)}
\PY{c+c1}{\PYZsh{}dark\PYZus{}red = (255, 200, 200)}
\PY{c+c1}{\PYZsh{}mask = cv.inRange(im, light\PYZus{}red, dark\PYZus{}red)}
\PY{n}{light\PYZus{}black} \PY{o}{=} \PY{p}{(}\PY{l+m+mi}{205}\PY{p}{,} \PY{l+m+mi}{205}\PY{p}{,} \PY{l+m+mi}{205}\PY{p}{)}
\PY{n}{dark\PYZus{}black} \PY{o}{=} \PY{p}{(}\PY{l+m+mi}{0}\PY{p}{,} \PY{l+m+mi}{0}\PY{p}{,} \PY{l+m+mi}{0}\PY{p}{)}
\PY{n}{mask} \PY{o}{=} \PY{n}{cv}\PY{o}{.}\PY{n}{inRange}\PY{p}{(}\PY{n}{im}\PY{p}{,} \PY{n}{dark\PYZus{}black}\PY{p}{,} \PY{n}{light\PYZus{}black}\PY{p}{)}
\PY{n}{cv}\PY{o}{.}\PY{n}{imwrite}\PY{p}{(}\PY{l+s+s1}{\PYZsq{}}\PY{l+s+s1}{out/color\PYZus{}mask.jpg}\PY{l+s+s1}{\PYZsq{}}\PY{p}{,} \PY{n}{mask}\PY{p}{)}
\PY{n}{plt}\PY{o}{.}\PY{n}{imshow}\PY{p}{(}\PY{n}{mask}\PY{p}{)}
\PY{n}{plt}\PY{o}{.}\PY{n}{title}\PY{p}{(}\PY{l+s+s2}{\PYZdq{}}\PY{l+s+s2}{Máscara de color}\PY{l+s+s2}{\PYZdq{}}\PY{p}{)}\PY{p}{,} \PY{n}{plt}\PY{o}{.}\PY{n}{xticks}\PY{p}{(}\PY{p}{[}\PY{p}{]}\PY{p}{)}\PY{p}{,} \PY{n}{plt}\PY{o}{.}\PY{n}{yticks}\PY{p}{(}\PY{p}{[}\PY{p}{]}\PY{p}{)}
\PY{n}{plt}\PY{o}{.}\PY{n}{show}\PY{p}{(}\PY{p}{)}
\end{Verbatim}
\end{tcolorbox}

    \begin{center}
    \adjustimage{max size={0.9\linewidth}{0.9\paperheight}}{output_21_0.png}
    \end{center}
    { \hspace*{\fill} \\}
    
    \begin{tcolorbox}[breakable, size=fbox, boxrule=1pt, pad at break*=1mm,colback=cellbackground, colframe=cellborder]
\prompt{In}{incolor}{13}{\boxspacing}
\begin{Verbatim}[commandchars=\\\{\}]
\PY{n}{contours}\PY{p}{,} \PY{n}{hierarchy} \PY{o}{=} \PY{n}{cv}\PY{o}{.}\PY{n}{findContours}\PY{p}{(}\PY{n}{mask}\PY{p}{,} \PY{n}{cv}\PY{o}{.}\PY{n}{RETR\PYZus{}EXTERNAL}\PY{p}{,} \PY{n}{cv}\PY{o}{.}\PY{n}{CHAIN\PYZus{}APPROX\PYZus{}SIMPLE}\PY{p}{)}
\PY{n}{out} \PY{o}{=} \PY{n}{cv}\PY{o}{.}\PY{n}{drawContours}\PY{p}{(}\PY{n}{im}\PY{p}{,} \PY{n}{contours}\PY{p}{,} \PY{o}{\PYZhy{}}\PY{l+m+mi}{1}\PY{p}{,} \PY{p}{(}\PY{l+m+mi}{64}\PY{p}{,}\PY{l+m+mi}{255}\PY{p}{,}\PY{l+m+mi}{64}\PY{p}{)}\PY{p}{,} \PY{l+m+mi}{3}\PY{p}{)}
\PY{n}{cv}\PY{o}{.}\PY{n}{imwrite}\PY{p}{(}\PY{l+s+s1}{\PYZsq{}}\PY{l+s+s1}{out/color\PYZus{}out.jpg}\PY{l+s+s1}{\PYZsq{}}\PY{p}{,} \PY{n}{out}\PY{p}{)}
\PY{n+nb}{print}\PY{p}{(}\PY{l+s+s1}{\PYZsq{}}\PY{l+s+s1}{Número de contornos encontrados: }\PY{l+s+s1}{\PYZsq{}}\PY{p}{,} \PY{n+nb}{len}\PY{p}{(}\PY{n}{contours}\PY{p}{)}\PY{p}{)}
\end{Verbatim}
\end{tcolorbox}

    \begin{Verbatim}[commandchars=\\\{\}]
Número de contornos encontrados:  16
    \end{Verbatim}

    \begin{tcolorbox}[breakable, size=fbox, boxrule=1pt, pad at break*=1mm,colback=cellbackground, colframe=cellborder]
\prompt{In}{incolor}{14}{\boxspacing}
\begin{Verbatim}[commandchars=\\\{\}]
\PY{n}{plt}\PY{o}{.}\PY{n}{imshow}\PY{p}{(}\PY{n}{out}\PY{p}{)}
\PY{n}{plt}\PY{o}{.}\PY{n}{title}\PY{p}{(}\PY{l+s+s2}{\PYZdq{}}\PY{l+s+s2}{Segmentación por color}\PY{l+s+s2}{\PYZdq{}}\PY{p}{)}\PY{p}{,} \PY{n}{plt}\PY{o}{.}\PY{n}{xticks}\PY{p}{(}\PY{p}{[}\PY{p}{]}\PY{p}{)}\PY{p}{,} \PY{n}{plt}\PY{o}{.}\PY{n}{yticks}\PY{p}{(}\PY{p}{[}\PY{p}{]}\PY{p}{)}
\PY{n}{plt}\PY{o}{.}\PY{n}{show}\PY{p}{(}\PY{p}{)}
\end{Verbatim}
\end{tcolorbox}

    \begin{center}
    \adjustimage{max size={0.9\linewidth}{0.9\paperheight}}{output_23_0.png}
    \end{center}
    { \hspace*{\fill} \\}
    
    En esta ocasión fue posible obtener el resultado deseado para la última
imagen pendiente: \texttt{img\_10t}. Además este método es preciso con
otras imágenes similares. En caso seleccionar una imagen con varias
cartas, este método encontraría todos los elementos del color de
interés, mas no contaría el número de cartas que aparecen en la imagen.

\hypertarget{conclusiuxf3n}{%
\subsection{Conclusión}\label{conclusiuxf3n}}

La segmentación de una imagen es un trabajo de presición. Es necesario
elegir el curso de acción adecuado para cada situación, debido a que
hasta donde puede verse, no existe una forma que presente un resultado
satisfactorio para todos los casos, incluso si la dificultad que
presentan las imágenes es relativamente sencilla.

Para el presente caso, la segmentación con color parece ser un
procedimiento fácilmente adaptable para imágenes que no exceden los dos
colores, sin embargo podría traer problemas si la imagen a procesar
tiene mayor número de colores.


    % Add a bibliography block to the postdoc
    
    
    
\end{document}
